\subtitle{4F13: Machine Learning}
\author[Ghahramani \& Rasmussen]{Zoubin Ghahramani and Carl Edward Rasmussen}
\institute[CUED]{Department of Engineering, University of Cambridge}

\parskip=1.5ex

\newcommand{\maketilde}{\raisebox{0.4ex}{\tiny $\sim$}}
\newcommand{\bfa}{\mathbf a}
\newcommand{\bfb}{\mathbf b}
\newcommand{\bff}{\mathbf f}
\newcommand{\bfm}{\mathbf m}
\newcommand{\bfp}{\mathbf{p}}
\newcommand{\bfs}{\mathbf s}
\newcommand{\bfu}{\mathbf u}
\newcommand{\bfx}{\mathbf x}
\newcommand{\bfy}{\mathbf y}
\newcommand{\bfv}{\mathbf v}

\newcommand{\bmu}{{\boldsymbol{\mu}}}
\newcommand{\btheta}{\boldsymbol{\theta}}
\newcommand{\balpha}{\boldsymbol{\alpha}}
\newcommand{\bphi}{\boldsymbol{\phi}}
\newcommand{\bpi}{\boldsymbol{\pi}}
\newcommand{\blambda}{\boldsymbol{\lambda}}

\newcommand{\N}{\mathcal{N}}
\newcommand{\R}{\mathbb{R}}
\newcommand{\T}{{\scriptsize^{\top}}}
\newcommand{\D}{\mathcal{D}}
\newcommand{\F}{\mathcal{F}}
\newcommand{\E}{\mathbb{E}}
\newcommand{\V}{\mathbb{V}}
\newcommand{\M}{\mathcal{M}}
\newcommand{\KL}{\mathcal{KL}}
\newcommand{\cut}[1]{}
\newcommand{\trace}{\operatorname{trace}}

\newcommand{\argmax}{\operatorname{argmax}}
\newcommand{\argmin}{\operatorname{argmin}}
\newcommand{\ci}{{\bot\negthickspace\negthickspace\bot}} % conditional indep.
\newcommand{\neigh}{\operatorname{ne}}
\newcommand{\vectr}[2]{  \left[ \!\!\begin{array}{c} #1 \\
      #2 \end{array} \!\!\right]}
\newcommand{\deff}{\stackrel{\mathrm{def}}{=}}
\newcommand{\deldel}[2]{\frac{\partial #1}{\partial #2}}


\definecolor{mypine}{rgb}{0.05,0.45,0.05}
\definecolor{mycyan}{rgb}{0.0,0.4,0.4}
\newcommand{\Red}{\textcolor{red}}
\newcommand{\Blue}{\textcolor{blue}}
\newcommand{\Green}{\textcolor{mypine}}
\newcommand{\PineGreen}{\textcolor{mypine}}
\newcommand{\Magenta}{\textcolor{magenta}}


\begin{document}

\titleslide{Lecture 1: Introduction to Machine Learning}{January 18th, 2008}

%%%%%%%%%%%%%%%%%%%%%%%%%%%%%%%%%%%%%%%%%%%%%%%%%%%%%%%%%%%%%%%%%%%%%%
\begin{frame}
\frametitle{What is machine learning?}

\begin{itemize}
\item {\it \Blue{Machine learning} is an interdisciplinary field
focusing on both the mathematical foundations and practical
applications of systems that learn, reason and act.}\\[1ex]

\item Other related terms: \Blue{Pattern Recognition}, \Blue{Neural
  Networks}, \Blue{Data Mining}, \Blue{Statistical Modelling} ...

\item Using ideas from: 
\Magenta{Statistics}, \Magenta{Computer Science}, \Magenta{Engineering}, 
\Magenta{Applied Mathematics}, \Magenta{Cognitive Science},
\Magenta{Psychology}, \Magenta{Computational Neuroscience},
\Magenta{Economics}  

\item \Blue{The goal of these lectures:} to introduce important
concepts, models and algorithms in machine learning.

\item \Blue{For more:} We have organised a ``Tutorial Lecture Series on
  Machine Learning'' with a series of guest lecturers (Thursdays,
  4-6pm in LT2). Go to
  {\tt talks.cam.ac.uk}, search for ``Machine Learning'' for various local reading
  groups, lectures, and seminars.

\end{itemize}
\end{frame}

%%%%%%%%%%%%%%%%%%%%%%%%%%%%%%%%%%%%%%%%%%%%%%%%%%%%%%%%%%%%%%%%%%%%%%
\begin{frame}
\frametitle{Warning!}

\vspace{0.5in}

\centerline{\framebox{\parbox{0.8\textwidth}{Lecture 1 will overlap somewhat with
      my lectures in 3f3: Pattern Processing---but don't despair, a lot of new
      material later!}}}

\end{frame}



%%%%%%%%%%%%%%%%%%%%%%%%%%%%%%%%%%%%%%%%%%%%%%%%%%%%%%%%%%%%%%%%%%%%%%
\begin{frame}
\frametitle{What is machine learning useful for?}
\end{frame}

%%%%%%%%%%%%%%%%%%%%%%%%%%%%%%%%%%%%%%%%%%%%%%%%%%%%%%%%%%%%%%%%%%%%%%
\begin{frame}
\frametitle{Automatic speech recognition}

\centerline{\includegraphics[width=0.5\textwidth]{speech-viavoice}}
\end{frame}

%%%%%%%%%%%%%%%%%%%%%%%%%%%%%%%%%%%%%%%%%%%%%%%%%%%%%%%%%%%%%%%%%%%%%%
\begin{frame}
\frametitle{Computer vision: e.g. object, face and handwriting recognition}

\centerline{
\includegraphics[width=0.6\textwidth]{trainingdigexamples}
\includegraphics[width=0.4\textwidth]{image-recognition}}

% \centerline{\includegraphics[width=0.7\textwidth]{digit-rec}}

\hfill {\small (NORB image from Yann LeCun)}
\end{frame}
%%%%%%%%%%%%%%%%%%%%%%%%%%%%%%%%%%%%%%%%%%%%%%%%%%%%%%%%%%%%%%%%%%%%%%
\begin{frame}
\frametitle{Information retrieval}

\centerline{
\begin{tabular}{c@{\qquad}c}
\fbox{\includegraphics[height=0.85\textheight]{unsup}}
&\parbox[b]{30ex}{\centerline{\large\bf Web Pages}\vspace{0.2in}
Retrieval\\Categorisation\\ Clustering\\ Relations between pages{\vskip 30ex}}
\end{tabular}}

\end{frame}
%%%%%%%%%%%%%%%%%%%%%%%%%%%%%%%%%%%%%%%%%%%%%%%%%%%%%%%%%%%%%%%%%%%%%%
\begin{frame}
\frametitle{Financial prediction}

\centerline{\includegraphics[height=0.5\textwidth]{finance}}

\end{frame}
%%%%%%%%%%%%%%%%%%%%%%%%%%%%%%%%%%%%%%%%%%%%%%%%%%%%%%%%%%%%%%%%%%%%%%
\begin{frame}
\frametitle{Medical diagnosis}

\centerline{\includegraphics[height=0.4\textwidth]{qmr}}

\centerline{\small (image from Kevin Murphy)}
\end{frame}
%%%%%%%%%%%%%%%%%%%%%%%%%%%%%%%%%%%%%%%%%%%%%%%%%%%%%%%%%%%%%%%%%%%%%%
\begin{frame}
\frametitle{Bioinformatics}

\centerline{\includegraphics[height=0.6\textheight]{microarray} \hspace{2ex}
\includegraphics[height=0.6\textheight]{expression-data}}

\vspace{2ex}

e.g.\ modelling gene microarray data, protein structure prediction

\end{frame}
%%%%%%%%%%%%%%%%%%%%%%%%%%%%%%%%%%%%%%%%%%%%%%%%%%%%%%%%%%%%%%%%%%%%%%
\begin{frame}
\frametitle{Robotics}

\centerline{\includegraphics[height=0.4\textheight]{stanley1}}

\centerline{\includegraphics[height=0.4\textheight]{stanley2}}

% 2 million dollar prize
\vspace{2ex}

DARPA \$2m Grand Challenge

\end{frame}
%%%%%%%%%%%%%%%%%%%%%%%%%%%%%%%%%%%%%%%%%%%%%%%%%%%%%%%%%%%%%%%%%%%%%%
\begin{frame}
\frametitle{Movie recommendation systems}

\centerline{\includegraphics[height=0.6\textheight]{netflix}}

\vspace{1ex}

Challenge: to improve the accuracy of movie preference predictions
 
Netflix \$1m Prize. Competition started Oct 2, 2006 and still ongoing.\\ 

\end{frame}
%%%%%%%%%%%%%%%%%%%%%%%%%%%%%%%%%%%%%%%%%%%%%%%%%%%%%%%%%%%%%%%%%%%%%%
\begin{frame}
\frametitle{Three Types of Learning}

Imagine an organism or machine which experiences a series of sensory inputs: 
%
\[
x_1, x_2, x_3, x_4, \ldots
\] 
%
\Green{Supervised learning:} The machine is also given
\textcolor{red}{desired outputs $y_1, y_2, \ldots$,} and its goal is
to learn to \Blue{produce the correct output} given a new
input. \\[1.5ex]

\Green{Unsupervised learning:} The goal of the machine is to \Blue{build
a model} of $x$ that can be used for reasoning, decision
making, predicting things, communicating etc. \\[1.5ex]

\Green{Reinforcement learning:} The machine can also produce \Red{actions $a_1,
a_2, \ldots$} which affect the state of the world, and receives \Red{rewards
(or punishments) $r_1, r_2, \ldots$}. Its goal is to learn to act in a
way that \Blue{maximises rewards} in the long term.  \\[2ex]

(In this course we'll focus mostly on unsupervised learning and
reinforcement learning.)

% review of probabilistic models, relation to coding terminology: Bayes
% rule, maximum likelihood, MAP, supervised, unsupervised and
% reinforcement learning

\end{frame}
%%%%%%%%%%%%%%%%%%%%%%%%%%%%%%%%%%%%%%%%%%%%%%%%%%%%%%%%%%%%%%%%%%
\begin{frame}
\frametitle{Key Ingredients}

\Red{Data} \\[1ex]

We will represent data by vectors in some vector space\footnote{This
assumption can be relaxed.}\\[1ex]

Let $\bfx$ denote a \Blue{data point} with elements $\bfx=(x_1, x_2, \ldots,
x_D)$\\[1ex] 

The elements of $\bfx$, e.g.\ $x_d$, represent measured (observed)
\Blue{features} of the data point; $D$ denotes the number of measured
features of each point. \\[1ex]

The \Blue{data set} $\D$ consists of $N$ data points:
\[
\D = \{ \bfx^{(1)}, \bfx^{(2)} \ldots, \bfx^{(N)} \}
\]

\end{frame}
%%%%%%%%%%%%%%%%%%%%%%%%%%%%%%%%%%%%%%%%%%%%%%%%%%%%%%%%%%%%%%%%%%
\begin{frame}
\frametitle{Key Ingredients}

\vspace{-1ex}

\Red{Data} \\

Let  $\bfx=(x_1, x_2, \ldots, x_D)$ denote a \Blue{data point}, and $
\D = \{ \bfx^{(1)}, \ldots, \bfx^{(N)} \} $, a \Blue{data
  set}\\[3ex]

\parbox{0.5\textwidth}{
\Red{Predictions} \\[-0.5ex]

We are generally interested in predicting something based on the observed
data. \\

Given $\D$ what can we say about $\bfx^{(N+1)}$?\\

Given $\D$ and $ x^{(N+1)}_1, x^{(N+1)}_2, \ldots,
x^{(N+1)}_{D-1}$, what can we say about $x^{(N+1)}_{D}$?}
\parbox{0.45\textwidth}{\centerline{\includegraphics[width=0.4\textwidth]{flix}}}

\end{frame}
%%%%%%%%%%%%%%%%%%%%%%%%%%%%%%%%%%%%%%%%%%%%%%%%%%%%%%%%%%%%%%%%%%
\begin{frame}
\frametitle{Key Ingredients}

\vspace{-1ex}

\Red{Data} \\

Let  $\bfx=(x_1, x_2, \ldots, x_D)$ be a \Blue{data point}, and $
\D = \{ \bfx^{(1)}, \bfx^{(2)} \ldots, \bfx^{(N)} \} $, a \Blue{data
  set}\\[2ex]

\Red{Predictions} \\

We are interested in predicting something based on the observed
data set. \\

Given $\D$ what can we say about $\bfx^{(N+1)}$?\\

Given $\D$ and $ x^{(N+1)}_1, x^{(N+1)}_2, \ldots,
x^{(N+1)}_{D-1}$, what can we say about $x^{(N+1)}_{D}$? \\[2ex]

\Red{Model}\\

To make predictions, we need to make some \Blue{\em assumptions}. 
We can often express these assumptions in the form of a \Blue{model}, with
some \Blue{parameters}, $\theta$.\\

Given data $\D$, we learn the model parameters $\theta$, from which we
can predict new data points.\\

The model can often be expressed as a \Blue{\emph probability distribution over
data points}

\end{frame}
%%%%%%%%%%%%%%%%%%%%%%%%%%%%%%%%%%%%%%%%%%%%%%%%%%%%%%%%%%%%%%%%%%
\begin{frame}
\frametitle{Basic Rules of Probability}

\vspace*{-2ex}
Let $X$ be a random variable taking values $x$ in some set $\cal X$.\\

Probabilities are non-negative $P(X=x)\geq 0\;\forall x$.\\

Probabilities normalise: $\sum_{x \in {\cal X}} P(X=x)=1$ for distributions if $x$ is a
discrete variable and $\int_{-\infty}^{+\infty} p(x)dx=1$ for
probability densities over continuous variables\\[2ex]

The \Blue{joint probability} of $X=x$ and $Y=y$ is: $P(X=x,Y=y)$.\\

The \Blue{marginal probability} of $X=x$ is: $P(X=x) = \sum_y P(X=x,y)$,
assuming $y$ is discrete. I will generally write $P(x)$ to mean
$P(X=x)$.\\ 

The \Blue{conditional probability} of $x$ given $y$ is:
$P(x|y)=P(x,y)/P(y)$ \\[2ex]

\Magenta{Bayes Rule:} \vspace*{-3ex}
\begin{equation*}
P(x,y)\;=\;P(x)P(y|x)\;=\;P(y)P(x|y)\qquad \Rightarrow \qquad
\fbox{\Red{$\displaystyle P(y|x) = \frac{P(x|y) P(y)}{P(x)}$}}
\end{equation*}

\vspace*{-1ex}

{\small {\bf Warning:} I will not be obsessively careful in my use of $p$
and $P$ for probability density and probability distribution. Should
be obvious from context.} 

\end{frame}
%%%%%%%%%%%%%%%%%%%%%%%%%%%%%%%%%%%%%%%%%%%%%%%%%%%%%%%%%%%%%%%%%%
\begin{frame}
\frametitle{Information, Probability and Entropy}

Information is the \Red{reduction of uncertainty}. How do we measure
uncertainty?\\

Some axioms (informally):
%
\begin{itemize}
\item if something is certain, its uncertainty = 0 \\[-3.5ex]
\item uncertainty should be maximum if all choices are equally
  probable \\[-3.5ex]
\item uncertainty (information) should add for independent sources
\end{itemize}
%
This leads to a discrete random variable $X$ having uncertainty equal
to the \Red{entropy} function:
%
\[
\Red{H(X) = - \sum_{x \in {\cal X}} P(X=x) \; \log P(X=x)}
\]
%
measured in \emph{bits} ({\bf bi}nary digi{\bf ts}) if the base $2$
logarithm is used or \emph{nats} ({\bf na}tural digi{\bf ts})
if the natural (base $e$) logarithm is used.

\end{frame}
%%%%%%%%%%%%%%%%%%%%%%%%%%%%%%%%%%%%%%%%%%%%%%%%%%%%%%%%%%%%%%%%%%
\begin{frame}
\frametitle{Some Definitions Relating to Information Theory}

\begin{itemize}
\item \Blue{Surprise} (for event $X=x$): $- \log P(X=x)$
\item \Blue{Entropy} = average surprise: $H(X) = - \sum_{x \in {\cal X}} P(X=x) \; \log P(X=x)$
\item \Blue{Conditional entropy}
\[
H(X|Y) = - \sum_{x} \sum_{y} P(x,y) \log P(x|y)
\]
\item \Blue{Mutual information}
\[
I(X;Y) = H(X) - H(X|Y) = H(Y) - H(Y|X) = H(X)+H(Y)-H(X,Y)
\]
% \im \Blue{Kullback-Leibler divergence} (relative entropy) 
% \[
% KL(P(X)\|Q(X)) = 
% \sum_{x} P(x) \log \frac{P(x)}{Q(x)}
% \]
% \im Relation between Mutual information and KL: $I(X;Y) = KL(P(X,Y)\|P(X)P(Y))$
\item Independent random variables: $P(x,y) = P(x) P(y) \, \forall x \,
\forall y$
% \im Conditional independence $X\ci Y|Z$ ($X$ conditionally independent of $Y$ given $Z$) \\ 
% \phantom{.} \hfill means  $P(X,Y|Z) = P(X|Z) P(Y|Z)$ and $P(X|Y,Z)=P(X|Z)$
\end{itemize}

How do we relate information theory and probabilistic modelling?

\end{frame}
%%%%%%%%%%%%%%%%%%%%%%%%%%%%%%%%%%%%%%%%%%%%%%%%%%%%%%%%%%%%%%%%%%
\begin{frame}
\frametitle{The source coding problem}

Imagine we have a set of symbols ${\cal X} = \{{\tt a, b, c, d, e, f,
  g, h}\}$. \\

We want to transmit these symbols over some binary communication
channel, i.e.\ using a sequence of \Blue{bits} to represent the
symbols. \\

Since we have 8 symbols, we could use 3 bits per symbol ($2^3 =
8$). For example:\\
${\tt a}=000, \, {\tt b}=001, \, {\tt c}=010, \, \ldots, \, {\tt h}=111$ \\[2ex]

\Red{Is this optimal?} \\[2ex]

What if some symbol, {\tt a}, is much more probable than other
symbols, e.g.\ {\tt f}? \\
Shouldn't we use fewer bits to transmit the
more probable symbols? \\[2ex]

Think of a discrete variable $X$ taking on values in $\cal X$,
having probability distribution $P(X)$. \\[1ex]

How does the probability distribution $P(X)$ relate to the number of
bits we need for each symbol to optimally and losslessly
transmit symbols from 
${\cal X}$? 

\end{frame}
%%%%%%%%%%%%%%%%%%%%%%%%%%%%%%%%%%%%%%%%%%%%%%%%%%%%%%%%%%%%%%%%%%
\begin{frame}
\frametitle{Shannon's Source Coding Theorem}

%\vspace*{-3ex}
A discrete random variable $X$, distributed according to $P(X)$ has
\Red{entropy}:
%
\vspace*{-1ex}
\[
\Red{H(X) = - \sum_{x \in \mathcal{X}} P(x) \log_2 P(x)} 
\] 
%
\Red{\bf Shannon's source coding theorem:} Consider a random variable $X$,
with entropy $H(X)$. A sequence of $n$ independent draws from $X$
can be losslessly compressed into a minimum expected code of length $n
{\cal L}$ bits, where $H(X) \le {\cal L} < H(X)+\frac{1}{n}.$ \\[2ex]

If each symbol is given a code length $l(x) = \Green{- \log_2 Q(x)}$
then the expected per-symbol length ${\cal L}_Q$ of the code is
%
\vspace*{-1ex}
\[
{\cal L}_Q = \sum_x P(x) l(x) = - \sum_x P(x) \log_2 Q(x) = H(X) + KL(P\|Q),
\]
%
\vspace*{-1ex}
where the \Blue{relative-entropy} or \Blue{Kullback-Leibler divergence} is 
%
\[
KL(P\|Q) = \sum_x P(x) \log_2 \frac{P(x)}{Q(x)} \ge 0
\]
%
\Green{\bf Take home message:} better probabilistic models $\equiv$ more
efficient codes

\end{frame}
%%%%%%%%%%%%%%%%%%%%%%%%%%%%%%%%%%%%%%%%%%%%%%%%%%%%%%%%%%%%%%%%%%%%%%
\begin{frame}
\frametitle{Some distributions}

\Blue{Univariate Gaussian density} ($x \in \mathbb{R}$):
\[
p(x|\mu,\sigma) = \frac{1}{\sqrt{2 \pi \sigma^2}} \exp \left\{
- \frac{(x - \mu)^2}{2 \sigma^2} \right\}
\]
\Blue{Multivariate Gaussian density} ($\bfx \in \mathbb{R}^D$):
\[
p(\bfx|\bmu,\Sigma) = |2\pi \Sigma|^{-\frac{1}{2}} \exp \left\{
-\frac{1}{2} (\bfx - \bmu)\T \Sigma^{-1} 
(\bfx - \bmu) \right\}
\]
\Blue{Bernoulli distribution} ($x \in \{0, 1\}$):
\[
p(x|\theta) = \theta^x (1-\theta)^{1-x}
\]
\Blue{Discrete distribution} ($x \in \{1, \dots L\}$):
\[
p(x|\theta) = \prod_{\ell = 1}^L \theta_\ell^{\delta(x,\ell)} 
\]
where $\delta(a,b)=1 $ iff $a=b$, and $\sum_{\ell=1}^L \theta_\ell
=1$ and $\theta_\ell \ge 0 \, \, \forall \ell$.


\end{frame}
%%%%%%%%%%%%%%%%%%%%%%%%%%%%%%%%%%%%%%%%%%%%%%%%%%%%%%%%%%%%%%%%%%%%%%
\begin{frame}
\frametitle{Some distributions (cont)}

\Blue{Uniform} ($x \in [a,b]$):
\[
p(x|a,b) = \left\{ 
\begin{array}{cc} 
\frac{1}{b-a} & \mbox{if} \, \, a\le x \le b \\
0 & \mbox{otherwise}
\end{array}
\right.
\]

\Blue{Gamma} ($x \ge 0$):
\[
p(x|a,b)= \frac{b^a}{\Gamma(a)} x^{a-1} \exp \{ - b x \} 
\]

\Blue{Beta} ($x \in [0,1]$):
\[
p(x|\alpha,\beta) =
\frac{\Gamma(\alpha+\beta)}{\Gamma(\alpha)\Gamma(\beta)} x^{\alpha-1}
(1-x)^{\beta-1} 
\]
where $\Gamma(z) = \int_0^\infty t^{z-1} e^{-t} dt$ is the gamma
function, a generalisation of the factorial: $\Gamma(n) = (n-1)!$. \\

\Blue{Dirichlet} ($\bfp \in \mathbb{R}^D$, $p_d \ge 0$, $\sum_{d=1}^D p_d = 1$):
\[
p(\bfp|\balpha) = \frac{\Gamma(\sum_{d=1}^D \alpha_d)}{\prod_{d=1}^D
\Gamma(\alpha_d)} \prod_{d=1}^D p_d^{\alpha_d -1}
\]

\end{frame}
%%%%%%%%%%%%%%%%%%%%%%%%%%%%%%%%%%%%%%%%%%%%%%%%%%%%%%%%%%%%%%%%%%
\begin{frame}
\frametitle{Dirichlet Distributions}

\vspace*{-1ex}
Examples of Dirichlet distributions over $\bfp=(p_1, p_2, p_3)$ which
can be plotted in 2D since $p_3=1-p_1-p_2$:

\vspace{2ex}

\centerline{
\includegraphics[width=0.2\textwidth]{dir111}
\includegraphics[width=0.2\textwidth]{dir222}
\includegraphics[width=0.2\textwidth]{dir000}
}

\vspace{1ex}

\centerline{
\includegraphics[width=0.2\textwidth]{dir202}
\includegraphics[width=0.2\textwidth]{dir220}
\includegraphics[width=0.2\textwidth]{dir999}
}

\end{frame}
%%%%%%%%%%%%%%%%%%%%%%%%%%%%%%%%%%%%%%%%%%%%%%%%%%%%%%%%%%%%%%%%%%
\begin{frame}
\frametitle{Other distributions you should know about...}

\Blue{Exponential family of distributions}:
\[
P(\bfx|\btheta) = f(\bfx) \; g(\btheta) \exp \left\{
\bphi(\btheta)^\top \bfu(
\bfx) \right\}
\]
where $\bphi(\btheta)$ is the vector of {\em natural parameters},
$\bfu$ are {\em sufficient statistics}


\begin{itemize}
\item Binomial 
\item Multinomial
\item Poisson
\item Student t distribution
\item ...
\end{itemize}

\end{frame}
%%%%%%%%%%%%%%%%%%%%%%%%%%%%%%%%%%%%%%%%%%%%%%%%%%%%%%%%%%%%%%%%%%
\begin{frame}
\frametitle{End Notes}

It is very important that you {\em understand} all the material in the
following cribsheet:\\
{\small \tt http://learning.eng.cam.ac.uk/zoubin/ml06/cribsheet.pdf}\\

Here is a useful statistics / pattern recognition glossary:\\
{\small \tt http://research.microsoft.com/$\sim$minka/statlearn/glossary/}\\

\end{frame}
%%%%%%%%%%%%%%%%%%%%%%%%%%%%%%%%%%%%%%%%%%%%%%%%%%%%%%%%%%%%%%%%%%
\end{document}

\cut{
%%%%%%%%%%%%%%%%%%%%%%%%%%%%%%%%%%%%%%%%%%%%%%%%%%%%%%%%%%%%%%%%%%
\frametitle{Bayesian Learning}

\vspace*{-3ex}
\Blue{Apply the basic rules of probability to learning from data.}\\

\Green{Data set: ${\cal D}=\{x_1, \ldots, x_n\}$ \hspace{2em} 
Models: $m$, $m'$ etc.
\hspace{2em} 
Model parameters: $\theta$} \\
Prior probability of models: $P(m)$, $P(m')$ etc.\\
Prior probabilities of model parameters:  $P(\theta|m)$\\
Model of data given parameters (likelihood model): $P(x|\theta,m)$\\

If the data are independently and identically distributed then:
\vspace*{-2ex}
\[
P({\cal D}|\theta,m) = \prod_{i=1}^n P(x_i|\theta,m)
\]
Posterior probability of model parameters:
\vspace*{-2ex}
\[
P(\theta|{\cal D},m) =\frac{P({\cal D}|\theta,m)P(\theta|m)}{P({\cal
    D}|m)}
\]
Posterior probability of models:
\vspace*{-2ex}
\[
P(m|{\cal D}) = \frac{P(m)P({\cal D}|m)}{P({\cal D})}
\]


}

\end{document}