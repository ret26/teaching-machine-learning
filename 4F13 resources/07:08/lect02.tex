\subtitle{4F13: Machine Learning}
\author[Ghahramani \& Rasmussen]{Zoubin Ghahramani and Carl Edward Rasmussen}
\institute[CUED]{Department of Engineering, University of Cambridge}

\parskip=1.5ex

\newcommand{\maketilde}{\raisebox{0.4ex}{\tiny $\sim$}}
\newcommand{\bfa}{\mathbf a}
\newcommand{\bfb}{\mathbf b}
\newcommand{\bff}{\mathbf f}
\newcommand{\bfm}{\mathbf m}
\newcommand{\bfp}{\mathbf{p}}
\newcommand{\bfs}{\mathbf s}
\newcommand{\bfu}{\mathbf u}
\newcommand{\bfx}{\mathbf x}
\newcommand{\bfy}{\mathbf y}
\newcommand{\bfv}{\mathbf v}

\newcommand{\bmu}{{\boldsymbol{\mu}}}
\newcommand{\btheta}{\boldsymbol{\theta}}
\newcommand{\balpha}{\boldsymbol{\alpha}}
\newcommand{\bphi}{\boldsymbol{\phi}}
\newcommand{\bpi}{\boldsymbol{\pi}}
\newcommand{\blambda}{\boldsymbol{\lambda}}

\newcommand{\N}{\mathcal{N}}
\newcommand{\R}{\mathbb{R}}
\newcommand{\T}{{\scriptsize^{\top}}}
\newcommand{\D}{\mathcal{D}}
\newcommand{\F}{\mathcal{F}}
\newcommand{\E}{\mathbb{E}}
\newcommand{\V}{\mathbb{V}}
\newcommand{\M}{\mathcal{M}}
\newcommand{\KL}{\mathcal{KL}}
\newcommand{\cut}[1]{}
\newcommand{\trace}{\operatorname{trace}}

\newcommand{\argmax}{\operatorname{argmax}}
\newcommand{\argmin}{\operatorname{argmin}}
\newcommand{\ci}{{\bot\negthickspace\negthickspace\bot}} % conditional indep.
\newcommand{\neigh}{\operatorname{ne}}
\newcommand{\vectr}[2]{  \left[ \!\!\begin{array}{c} #1 \\
      #2 \end{array} \!\!\right]}
\newcommand{\deff}{\stackrel{\mathrm{def}}{=}}
\newcommand{\deldel}[2]{\frac{\partial #1}{\partial #2}}


\definecolor{mypine}{rgb}{0.05,0.45,0.05}
\definecolor{mycyan}{rgb}{0.0,0.4,0.4}
\newcommand{\Red}{\textcolor{red}}
\newcommand{\Blue}{\textcolor{blue}}
\newcommand{\Green}{\textcolor{mypine}}
\newcommand{\PineGreen}{\textcolor{mypine}}
\newcommand{\Magenta}{\textcolor{magenta}}


\begin{document}

\titleslide{Lecture 2, 3: PCA, FA and EM}{January 23rd, 25th, 2008}

<<<<<<< .mine
\begin{frame}
\frametitle{Latent Variable Models}

Explain correlations in $\bfy$ by assuming a \Blue{generative model} with
\Blue{latent} (hidden) \Blue{variables} $\bfx$

\vfill 

\begin{minipage}{61mm}
\centerline{\includegraphics[width=61mm]{genmodel}}
\end{minipage}
\begin{minipage}{61mm}
  \begin{gather*}
    \bfx \sim p(\bfx|\theta_x)\\[1ex]
    \bfy|\bfx \sim p(\bfy|\bfx,\theta_y) \\[1ex]
    p(\bfx,\bfy|\theta_x,\theta_y)=p(\bfy|\bfx,\theta_y)p(\bfx|\theta_x)\\
    p(\bfy | \theta_x, \theta_y) =
      \int p(\bfy | \bfx, \theta_y) p(\bfx | \theta_x)d \bfx
  \end{gather*}
\end{minipage}
\end{frame}

\begin{frame}
\frametitle{Inference and Learning; latent variables and parameters}

Some of the variables in our model are termed \Blue{latent variables}
and some are called \Blue{parameters}, why?
\begin{itemize}
\item each example has separate latent values; so there are many
\item we usually try to integrate out latent values
\end{itemize}

The process of finding the distribution over latent variables is called
\Blue{inference}. The process of finding parameters is called \Blue{learning}.

Ideally, we would want to integrate over all unknown quantities
\[
p(\bfy)\;=\;\int\!\!\int p(\bfy|\bfx,\theta)p(\bfx|\theta)p(\theta)d\bfx d\theta,
\]
but unfortunately, this is not trivial --- \Red{but we will show you how to do
this approximately in a few weeks time.}

Today, we concentrate on the simpler task where parameters are treated
as deterministic quantities
\[
p(\bfy)\;\simeq\;p(\bfy|\hat\theta), \text{\ \ where\ \ }
\hat\theta=\argmax\int p(\bfy|\bfx,\theta)p(\bfx|\theta)d\bfx.
\]
\end{frame}

\begin{frame}
\frametitle{Factor Analysis}

Latent variable models are useful even when both latent and observed
variables are Gaussian.

\vfill

\begin{minipage}{43mm}
\centerline{\includegraphics[width=43mm]{fa2col}}
\end{minipage} \hspace{5mm}
\begin{minipage}{70mm}
Linear generative model: $\displaystyle
y_d = \sum_{k=1}^K \Lambda_{dk} \; x_k + \epsilon_d$

$\bullet$ $x_k$ are independent ${\cal N}(0,1)$ Gaussian \Red{factors} \\
$\bullet$ $\epsilon_d$ are
independent ${\cal N}(0,\Psi_{dd})$ Gaussian \Red{noise} \\
$\bullet$ $K\!<\!D$
\end{minipage}

So, $\bfy$ is Gaussian with: $\displaystyle
p(\bfy) = \int p(\bfx) p(\bfy|\bfx) d\bfx = {\cal N}(0,\Lambda \Lambda\T +
\Psi) $ \\
where $\Lambda$ is a $D\times K$ matrix, and $\Psi$ is diagonal.

{\bf Dimensionality Reduction:} Finds a low-dimensional projection of
high dimensional data that captures the \Blue{correlation structure}
of the data.
\end{frame}

\begin{frame}
\frametitle{Inference in the Factor Analysis Model}

Given an observation, $\bfy$, what is the \emph{distribution} of the
latent cause?

Use Bayes' rule:
\[
p(\bfx|\bfy)\;=\;\frac{p(\bfy|\bfx)p(\bfx)}{p(\bfy)}
\;\propto\;p(\bfy|\bfx)p(\bfx).
\]
In detail:
\[
\begin{split}
p(\bfx|\bfy)\;&\propto\;\Red{p(\bfx)}\;\Blue{p(\bfy|\bfx)}\\
&=\;\Red{(2\pi)^{-K/2}\exp\!\big[\!-\tfrac{1}{2}\bfx\T\bfx\big]}\;
\Blue{|2\pi\Psi|^{-1/2}\exp\!\big[\!-\tfrac{1}{2}(\bfy-\Lambda\bfx)\T
\Psi^{-1}(\bfy-\Lambda\bfx)\big]}\\
&\propto\;\exp\!\big(\!-\tfrac{1}{2}[\bfx\T\bfx+
(\bfy-\Lambda\bfx)\T\Psi^{-1}(\bfy-\Lambda \bfx)]\big)\\
&\propto\;\exp\!\big(\!-\tfrac{1}{2}[\bfx\T
\Red{(I+\Lambda\T\Psi^{-1}\Lambda)}\bfx-2\bfx\T
\Blue{\Lambda\T\Psi^{-1}\bfy}]\big)\\
&\propto\;\exp\!\big(\!-\tfrac{1}{2}[\bfx\T
\Red{\Sigma^{-1}}\bfx-2\bfx\T\Blue{\Sigma^{-1}\bmu}
+\bmu\T\Sigma^{-1}\bmu]\big)\;\propto\;{\cal N}(\Blue{\bmu},\Red{\Sigma}),
\end{split}
\]
where $\Red{\Sigma=(I+\Lambda\T\Psi^{-1}\Lambda)^{-1}=I-\beta\Lambda}$, $\Blue{\bmu=\Sigma\Lambda\T\Psi^{-1}\bfy=\beta\bfy}$, and $\beta=\Sigma\Lambda\T\Psi^{-1}$.

Note that $\bmu$ is a linear function of $\bfy$ and $\Sigma$ does not
depend on $\bfy$.


\end{frame}


\begin{frame}
\frametitle{Learning in the Factor Analysis Model}

Integrating over the latent variables, we obtained:
\[
p(\bfy)\;=\;\int p(\bfx) p(\bfy|\bfx) d\bfx = {\cal N}(0,\Lambda \Lambda\T +
\Psi).
\]
Note that Factor Analysis is a restricted form of Gaussian model.

The unrestricted (zero mean) Gaussian model has a closed form solution
\[
p(\bfy)\;\sim\;{\cal N}\big(0,\;\textstyle\frac{1}{n}\sum_c\bfy^{(c)}\bfy^{(c)}\T\big),
\]
but is no closed form solution for the Factor Analysis model.

We could try to use the gradients to do maximum likelihood wrt the parameters
$\Lambda$ and $\Psi$.

It turns out that there is a better algorithm for learning (the EM algorithm).
\end{frame}

\begin{frame}
\frametitle{Eigenvalues and Eigenvectors}

$\lambda$ is an \Red{eigenvalue} and $\bfx$ is an \Red{eigenvector} of $A$ if:
%
\[
A \bfx\;=\;\lambda \bfx
\]
and $\bfx$ is a unit vector ($\bfx^\top \bfx = 1$). \\[1ex]

{\bf Interpretation:} the operation of $A$ in direction $\bfx$ is a scaling
by $\lambda$. \\[1ex]

The $K$ Principal Components are the $K$ eigenvectors with the largest
eigenvalues of the \Blue{data covariance matrix} (i.e. $K$ directions with
the largest variance). \\[1ex]

Note: $\Sigma$ can be decomposed:
\[
\Sigma\;=\; U S U^\top,
\]
where $S$ is diag($\sigma^2_1, \ldots, \sigma^2_D$) and $U$ is a
an orthonormal matrix.
\end{frame}

\begin{frame}
\frametitle{Example of PCA: Eigenfaces}

\centerline{\includegraphics[width=70mm]{eigfacessmall}}

from \texttt{http://vismod.media.mit.edu/vismod/demos/facerec/basic.html}
\end{frame}

\begin{frame}
\frametitle{Principal Components Analysis (PCA)}
\vskip 3mm

\centerline{\includegraphics[width=43mm]{fa2col}}

Noise variable becomes infinitesimal compared to the scale of the
data: $\displaystyle \Psi = \lim_{\sigma^2 \rightarrow 0} \sigma^2 I $

Equivalently: reconstruction cost becomes infinite compared to the
cost of coding the hidden units under the prior.
\[
p(\bfx|\bfy)\;=\;{\cal N}(\beta \bfy, I - \beta \Lambda),
\]
where
\[
\beta\;=\;\lim_{\sigma^2 \rightarrow 0} \Lambda^\top (\Lambda \Lambda^\top +
\sigma^2 I)^{-1}\;=\;(\Lambda^\top \Lambda)^{-1} \Lambda^\top.
\]
\end{frame}

\begin{frame}
\frametitle{From Supervised Learning to PCA: linear autoencoder}

\vspace{4mm}
\centerline{\includegraphics[width=72mm]{autoencoder}}

\vspace{4mm}
A linear autoencoder neural network trained to minimise squared error
learns to perform PCA (Baldi \& Hornik, 1989).
\end{frame}

\begin{frame}
\frametitle{FA vs PCA}
\vfill
\begin{itemize}

\item PCA is rotationally invariant; FA is not

\item FA is measurement scale invariant; PCA is not

\item FA defines a probabilistic model; PCA does not

\item PCA can be computed in closed form; FA can not 
\end{itemize}
\vfill
\end{frame}

\begin{frame}
\frametitle{Jensen's Inequality}

\centerline{\includegraphics[height=0.55\textheight]{jensen_demo}}

\vfill

For $\alpha_i \ge 0$, $\sum \alpha_i = 1$ and any $\{x_i > 0\}$
%
\[
\log \big( \sum_i \alpha_i x_i \big)\;\ge\;\sum_i \alpha_i \log(x_i)
\]
%
Equality if and only if $\alpha_i =  1$ for some $i$ (and therefore
all others are 0).
\end{frame}


\begin{frame}
\frametitle{The Expectation Maximization (EM) algorithm}

Given a set of observed (visible) variables $V$, a set of unobserved
(hidden / latent / missing) variables $H$, and model parameters
$\theta$, optimize the log likelihood:
%
\begin{equation}
{\cal L}(\theta)\;=\;\log p(V|\theta)\;=\;\log \int p(H,V|\theta)dH,
\end{equation}
%
where we have written the marginal for the visibles in terms of an
integral over the joint distribution for hidden and visible variables.

Using \emph{Jensen's inequality} for \Red{any} distribution of hidden
states $q(H)$ we have:
%
\begin{equation}
{\cal L}\;=\;\log\int q(H)\frac{p(H,V|\theta)}{q(H)}dH
\;\geq\;\int q(H)\log\frac{p(H,V|\theta)}{q(H)}dH\;=\;{\cal F}(q,\theta),
\end{equation}
%
defining the ${\cal F}(q,\theta)$ functional, which is a \Blue{lower bound}
on the log likelihood.

In the EM algorithm, we alternately optimize ${\cal F}(q,\theta)$ wrt
$q$ and $\theta$, and we can prove that this will never decrease
${\cal L}$.
\end{frame}

\begin{frame}
\frametitle{The E and M steps of EM}

The lower bound on the log likelihood:
%
\begin{equation}
{\cal F}(q,\theta)\;=\;\int q(H)\log\frac{p(H,V|\theta)}{q(H)}dH\;=\;
\int q(H)\log p(H,V|\theta)dH+\Green{{\cal H}(q)},
\end{equation}
%
where $\Green{{\cal H}(q)=-\displaystyle\int q(H)\log q(H)dH}$ is the
\Green{entropy} of $q$. We iteratively alternate:
%
\Blue{E step:} optimize ${\cal F}(q,\theta)$ wrt the distribution over hidden
variables given the parameters:
\begin{equation}
\Blue{q^{(k)}(H)}:=\underset{q(H)}{\argmax}\;\;{\cal F}\big(\Blue{q(H)},\Red{\theta^{(k-1)}}\big).
\end{equation}
\Blue{M step:} maximize ${\cal F}(q,\theta)$ wrt the parameters given
the hidden distribution:
\begin{equation}
\Blue{\theta^{(k)}}:=\underset{\theta}{\argmax}\;\;
{\cal F}\big(\Red{q^{(k)}(H)},\Blue{\theta}\big)=
\underset{\theta}{\argmax}\;\;\int q^{(k)}(H)\log p(H,V|\theta)dH,
\end{equation}
which is equivalent to optimizing the expected complete-data likelihood
$p(H,V|\theta)$, since the \Green{entropy of $q(H)$} does not depend on
$\theta$.
\end{frame}

\begin{frame}
\frametitle{EM as Coordinate Ascent in ${\cal F}$}

\centerline{\includegraphics[width=100mm]{fqtheta}}
\end{frame}

\begin{frame}
\frametitle{The EM algorithm never decreases the log likelihood}

The difference between the cost functions:
%
\begin{equation*}
\begin{split}
\Green{{\cal L}(\theta)-{\cal F}(q,\theta)}\;=&\;\log p(V|\theta)
-\int q(H)\log\frac{p(H,V|\theta)}{q(H)}dH\\
=&\;\log p(V|\theta)-\int q(H)\log\frac{p(H|V,\theta)p(V|\theta)}{q(H)}dH\\
=&\;-\int q(H)\log\frac{p(H|V,\theta)}{q(H)}dH\;=\;
\Green{{\cal KL}\big(q(H),p(H|V,\theta)\big)},
\end{split}
\end{equation*}
%
is called the Kullback-Liebler divergence; it is non-negative and only
zero if and only if $q(H)=p(H|V,\theta)$ (thus this is the E
step). Although we are working with the \Blue{wrong cost function}, the
likelihood is still increased in every iteration:
%
\begin{equation*}
{\cal L}\big(\theta^{(k-1)}\big)\;
\underset{\Blue{\mbox{E step}}}{=}
\;{\cal F}\big(q^{(k)},\theta^{(k-1)}\big)
\underset{\mbox{\Blue{M step}}}{\leq}
\;{\cal F}\big(q^{(k)},\theta^{(k)}\big)
\underset{\mbox{\Blue{Jensen}}}{\leq}
\;{\cal L}\big(\theta^{(k)}\big),
\end{equation*}
where the first equality holds because of the E step, and the first inequality
comes from the M step and the final inequality from Jensen. Usually EM
converges to a local optimum of ${\cal L}$ (although there are exceptions).
\end{frame}


=======
\begin{frame}
\frametitle{Latent Variable Models}

Explain correlations in $\bfy$ by assuming a \Blue{generative model} with
\Blue{latent} (hidden) \Blue{variables} $\bfx$

\vfill 

\begin{minipage}{61mm}
\centerline{\includegraphics[width=61mm]{genmodel}}
\end{minipage}
\begin{minipage}{61mm}
  \begin{gather*}
    \bfx \sim p(\bfx|\theta_x)\\[1ex]
    \bfy|\bfx \sim p(\bfy|\bfx,\theta_y) \\[1ex]
    p(\bfx,\bfy|\theta_x,\theta_y)=p(\bfy|\bfx,\theta_y)p(\bfx|\theta_x)\\
    p(\bfy | \theta_x, \theta_y) =
      \int p(\bfy | \bfx, \theta_y) p(\bfx | \theta_x)d \bfx
  \end{gather*}
\end{minipage}
\end{frame}

\begin{frame}
\frametitle{Inference and Learning; latent variables and parameters}

Some of the variables in our model are termed \Blue{latent variables}
and some are called \Blue{parameters}, why?
\begin{itemize}
\item each example has separate latent values; so there are many
\item we usually try to integrate out latent values
\end{itemize}

The process of finding the distribution over latent variables is called
\Blue{inference}. The process of finding parameters is called \Blue{learning}.

Ideally, we would want to integrate over all unknown quantities
\[
p(\bfy)\;=\;\int\!\!\int p(\bfy|\bfx,\theta)p(\bfx|\theta)p(\theta)d\bfx d\theta,
\]
but unfortunately, this is not trivial --- \Red{but we will show you how to do
this approximately in a few weeks time.}

Today, we concentrate on the simpler task where parameters are treated
as deterministic quantities
\[
p(\bfy)\;\simeq\;p(\bfy|\hat\theta), \text{\ \ where\ \ }
\hat\theta=\argmax\int p(\bfy|\bfx,\theta)p(\bfx|\theta)d\bfx.
\]
\end{frame}

\begin{frame}
\frametitle{Factor Analysis}

Latent variable models are useful even when both latent and observed
variables are Gaussian.

\vfill

\begin{minipage}{43mm}
\centerline{\includegraphics[width=43mm]{fa2col}}
\end{minipage} \hspace{5mm}
\begin{minipage}{70mm}
Linear generative model: $\displaystyle
y_d = \sum_{k=1}^K \Lambda_{dk} \; x_k + \epsilon_d$

$\bullet$ $x_k$ are independent ${\cal N}(0,1)$ Gaussian \Red{factors} \\
$\bullet$ $\epsilon_d$ are
independent ${\cal N}(0,\Psi_{dd})$ Gaussian \Red{noise} \\
$\bullet$ $K\!<\!D$
\end{minipage}

So, $\bfy$ is Gaussian with: $\displaystyle
p(\bfy) = \int p(\bfx) p(\bfy|\bfx) d\bfx = {\cal N}(0,\Lambda \Lambda\T +
\Psi) $ \\
where $\Lambda$ is a $D\times K$ matrix, and $\Psi$ is diagonal.

{\bf Dimensionality Reduction:} Finds a low-dimensional projection of
high dimensional data that captures the \Blue{correlation structure}
of the data.
\end{frame}

\begin{frame}
\frametitle{Inference in the Factor Analysis Model}

Given an observation, $\bfy$, what is the \emph{distribution} of the
latent cause?

Use Bayes' rule:
\[
p(\bfx|\bfy)\;=\;\frac{p(\bfy|\bfx)p(\bfx)}{p(\bfy)}
\;\propto\;p(\bfy|\bfx)p(\bfx).
\]
In detail:
\[
\begin{split}
p(\bfx|\bfy)\;&\propto\;\Red{p(\bfx)}\;\Blue{p(\bfy|\bfx)}\\
&=\;\Red{(2\pi)^{-K/2}\exp\!\big[\!-\tfrac{1}{2}\bfx\T\bfx\big]}\;
\Blue{|2\pi\Psi|^{-1/2}\exp\!\big[\!-\tfrac{1}{2}(\bfy-\Lambda\bfx)\T
\Psi^{-1}(\bfy-\Lambda\bfx)\big]}\\
&\propto\;\exp\!\big(\!-\tfrac{1}{2}[\bfx\T\bfx+
(\bfy-\Lambda\bfx)\T\Psi^{-1}(\bfy-\Lambda \bfx)]\big)\\
&\propto\;\exp\!\big(\!-\tfrac{1}{2}[\bfx\T
\Red{(I+\Lambda\T\Psi^{-1}\Lambda)}\bfx-2\bfx\T
\Blue{\Lambda\T\Psi^{-1}\bfy}]\big)\\
&\propto\;\exp\!\big(\!-\tfrac{1}{2}[\bfx\T
\Red{\Sigma^{-1}}\bfx-2\bfx\T\Blue{\Sigma^{-1}\bmu}
+\bmu\T\Sigma^{-1}\bmu]\big)\;\propto\;{\cal N}(\Blue{\bmu},\Red{\Sigma}),
\end{split}
\]
where $\Red{\Sigma=(I+\Lambda\T\Psi^{-1}\Lambda)^{-1}=I-\beta\Lambda}$, $\Blue{\bmu=\Sigma\Lambda\T\Psi^{-1}\bfy=\beta\bfy}$, and $\beta=\Sigma\Lambda\T\Psi^{-1}$.

Note that $\bmu$ is a linear function of $\bfy$ and $\Sigma$ does not
depend on $\bfy$.


\end{frame}


\begin{frame}
\frametitle{Learning in the Factor Analysis Model}

Integrating over the latent variables, we obtained:
\[
p(\bfy)\;=\;\int p(\bfx) p(\bfy|\bfx) d\bfx = {\cal N}(0,\Lambda \Lambda\T +
\Psi).
\]
Note that Factor Analysis is a restricted form of Gaussian model.

The unrestricted (zero mean) Gaussian model has a closed form solution
\[
p(\bfy)\;\sim\;{\cal N}\big(0,\;\textstyle\frac{1}{n}\sum_c\bfy^{(c)}\bfy^{(c)}\T\big),
\]
but is no closed form solution for the Factor Analysis model.

We could try to use the gradients to do maximum likelihood wrt the parameters
$\Lambda$ and $\Psi$.

It turns out that there is a better algorithm for learning (the EM algorithm).
\end{frame}

\begin{frame}
\frametitle{Eigenvalues and Eigenvectors}

$\lambda$ is an \Red{eigenvalue} and $\bfx$ is an \Red{eigenvector} of $A$ if:
%
\[
A \bfx\;=\;\lambda \bfx
\]
and $\bfx$ is a unit vector ($\bfx^\top \bfx = 1$). \\[1ex]

{\bf Interpretation:} the operation of $A$ in direction $\bfx$ is a scaling
by $\lambda$. \\[1ex]

The $K$ Principal Components are the $K$ eigenvectors with the largest
eigenvalues of the \Blue{data covariance matrix} (i.e. $K$ directions with
the largest variance). \\[1ex]

Note: $\Sigma$ can be decomposed:
\[
\Sigma\;=\; U S U^\top,
\]
where $S$ is diag($\sigma^2_1, \ldots, \sigma^2_D$) and $U$ is a
an orthonormal matrix.
\end{frame}

\begin{frame}
\frametitle{Example of PCA: Eigenfaces}

\centerline{\includegraphics[width=70mm]{eigfacessmall}}

from \texttt{http://vismod.media.mit.edu/vismod/demos/facerec/basic.html}
\end{frame}

\begin{frame}
\frametitle{Principal Components Analysis (PCA)}
\vskip 3mm

\centerline{\includegraphics[width=43mm]{fa2col}}

Noise variable becomes infinitesimal compared to the scale of the
data: $\displaystyle \Psi = \lim_{\sigma^2 \rightarrow 0} \sigma^2 I $

Equivalently: reconstruction cost becomes infinite compared to the
cost of coding the hidden units under the prior.
\[
p(\bfx|\bfy)\;=\;{\cal N}(\beta \bfy, I - \beta \Lambda),
\]
where
\[
\beta\;=\;\lim_{\sigma^2 \rightarrow 0} \Lambda^\top (\Lambda \Lambda^\top +
\sigma^2 I)^{-1}\;=\;(\Lambda^\top \Lambda)^{-1} \Lambda^\top.
\]
\end{frame}

\begin{frame}
\frametitle{From Supervised Learning to PCA: linear autoencoder}

\vspace{4mm}
\centerline{\includegraphics[width=72mm]{autoencoder}}

\vspace{4mm}
A linear autoencoder neural network trained to minimise squared error
learns to perform PCA (Baldi \& Hornik, 1989).
\end{frame}

\begin{frame}
\frametitle{FA vs PCA}
\vfill
\begin{itemize}

\item PCA is rotationally invariant; FA is not

\item FA is measurement scale invariant; PCA is not

\item FA defines a probabilistic model; PCA does not

\item PCA can be computed in closed form; FA can not 
\end{itemize}
\vfill
\end{frame}

\begin{frame}
\frametitle{Jensen's Inequality}

\centerline{\includegraphics[height=0.55\textheight]{jensen_demo}}

\vfill

For $\alpha_i \ge 0$, $\sum \alpha_i = 1$ and any $\{x_i > 0\}$
%
\[
\log \big( \sum_i \alpha_i x_i \big)\;\ge\;\sum_i \alpha_i \log(x_i)
\]
%
Equality if and only if $\alpha_i =  1$ for some $i$ (and therefore
all others are 0).
\end{frame}


\begin{frame}
\frametitle{The Expectation Maximization (EM) algorithm}

Given a set of observed (visible) variables $V$, a set of unobserved
(hidden / latent / missing) variables $H$, and model parameters
$\theta$, optimize the log likelihood:
%
\begin{equation}
{\cal L}(\theta)\;=\;\log p(V|\theta)\;=\;\log \int p(H,V|\theta)dH,
\end{equation}
%
where we have written the marginal for the visibles in terms of an
integral over the joint distribution for hidden and visible variables.

Using \emph{Jensen's inequality} for \Red{any} distribution of hidden
states $q(H)$ we have:
%
\begin{equation}
{\cal L}\;=\;\log\int q(H)\frac{p(H,V|\theta)}{q(H)}dH
\;\geq\;\int q(H)\log\frac{p(H,V|\theta)}{q(H)}dH\;=\;{\cal F}(q,\theta),
\end{equation}
%
defining the ${\cal F}(q,\theta)$ functional, which is a \Blue{lower bound}
on the log likelihood.

In the EM algorithm, we alternately optimize ${\cal F}(q,\theta)$ wrt
$q$ and $\theta$, and we can prove that this will never decrease
${\cal L}$.
\end{frame}

\begin{frame}
\frametitle{The E and M steps of EM}

The lower bound on the log likelihood:
%
\begin{equation}
{\cal F}(q,\theta)\;=\;\int q(H)\log\frac{p(H,V|\theta)}{q(H)}dH\;=\;
\int q(H)\log p(H,V|\theta)dH+\Green{{\cal H}(q)},
\end{equation}
%
where $\Green{{\cal H}(q)=-\displaystyle\int q(H)\log q(H)dH}$ is the
\Green{entropy} of $q$. We iteratively alternate:
%
\Blue{E step:} optimize ${\cal F}(q,\theta)$ wrt the distribution over hidden
variables given the parameters:
\begin{equation}
\Blue{q^{(k)}(H)}:=\underset{q(H)}{\argmax}\;\;{\cal F}\big(\Blue{q(H)},\Red{\theta^{(k-1)}}\big).
\end{equation}
\Blue{M step:} maximize ${\cal F}(q,\theta)$ wrt the parameters given
the hidden distribution:
\begin{equation}
\Blue{\theta^{(k)}}:=\underset{\theta}{\argmax}\;\;
{\cal F}\big(\Red{q^{(k)}(H)},\Blue{\theta}\big)=
\underset{\theta}{\argmax}\;\;\int q^{(k)}(H)\log p(H,V|\theta)dH,
\end{equation}
which is equivalent to optimizing the expected complete-data likelihood
$p(H,V|\theta)$, since the \Green{entropy of $q(H)$} does not depend on
$\theta$.
\end{frame}

\begin{frame}
\frametitle{EM as Coordinate Ascent in ${\cal F}$}

\centerline{\includegraphics[width=100mm]{fqtheta}}
\end{frame}

\begin{frame}
\frametitle{The EM algorithm never decreases the log likelihood}

The difference between the cost functions:
%
\begin{equation*}
\begin{split}
\Green{{\cal L}(\theta)-{\cal F}(q,\theta)}\;=&\;\log p(V|\theta)
-\int q(H)\log\frac{p(H,V|\theta)}{q(H)}dH\\
=&\;\log p(V|\theta)-\int q(H)\log\frac{p(H|V,\theta)p(V|\theta)}{q(H)}dH\\
=&\;-\int q(H)\log\frac{p(H|V,\theta)}{q(H)}dH\;=\;
\Green{{\cal KL}\big(q(H),p(H|V,\theta)\big)},
\end{split}
\end{equation*}
%
is called the Kullback-Liebler divergence; it is non-negative and only
zero if and only if $q(H)=p(H|V,\theta)$ (thus this is the E
step). Although we are working with the \Blue{wrong cost function}, the
likelihood is still increased in every iteration:
%
\begin{equation*}
{\cal L}\big(\theta^{(k-1)}\big)\;
\underset{\Blue{\mbox{E step}}}{=}
\;{\cal F}\big(q^{(k)},\theta^{(k-1)}\big)
\underset{\mbox{\Blue{M step}}}{\leq}
\;{\cal F}\big(q^{(k)},\theta^{(k)}\big)
\underset{\mbox{\Blue{Jensen}}}{\leq}
\;{\cal L}\big(\theta^{(k)}\big),
\end{equation*}
where the first equality holds because of the E step, and the first inequality
comes from the M step and the final inequality from Jensen. Usually EM
converges to a local optimum of ${\cal L}$ (although there are exceptions).
\end{frame}


\begin{frame}
\frametitle{EM for Factor Analysis}

\begin{minipage}{36mm}
\centerline{\includegraphics[width=30mm]{fa2col}}
\end{minipage} \hspace{6mm}
\begin{minipage}{80mm}
The model for $\bfy$: \\
$ p(\bfy|\theta) = \int p(\bfx|\theta) p(\bfy|\bfx,\theta) d\bfx = {\cal
  N}(0,\Lambda \Lambda^\top + \Psi) $ \\
Model parameters: $\theta = \{ \Lambda, \Psi \}$.
\end{minipage}

\vspace{0.3in}

{\bf E step:} For each data point $\bfy_c$, compute the posterior
distribution of hidden factors given the observed data:
$q_c(\bfx_c)=p(\bfx_c|\bfy_c,\theta^{(t)})$. \\[1ex]

{\bf M step:} Find the $\theta^{(t+1)}$ that maximises ${\cal F}(q,\theta)$:
\[
\begin{split}
{\cal F}(q,\theta)\;&=\;\sum_c  \int q_c(\bfx_c)\Big[\log p(\bfx_c|\theta)+
\log p(\bfy_c|\bfx_c,\theta) - \log q_c(\bfx_c) \Big]  d\bfx_c \\
&=\;\sum_c \int q_c(\bfx) \Big[ \log p(\bfx_c|\theta) +
\log p(\bfy_c|\bfx_c,\theta) \Big] d\bfx + \text{const.}
\end{split}
\]
\end{frame}

\begin{frame}
\frametitle{The M step for Factor Analysis}

{\bf M step:} Find $\theta^{(t+1)}$ maximising ${\cal F} = \sum_c \int
q_c(\bfx)\left[\log p(\bfx|\theta)+\log p(\bfy_c|\bfx,\theta) \right] d\bfx$.
%
\[
\begin{split}
\Red{\log p(\bfx|\theta)}&+\Blue{\log p(\bfy_c|\bfx,\theta)}\\
&=\;\mbox{const}-\Red{\tfrac{1}{2}\bfx_c^\top\bfx_c}-\Blue{\tfrac{1}{2}\log|\Psi|
-\tfrac{1}{2}(\bfy_c-\Lambda\bfx_c)^\top\Psi^{-1}(\bfy_c-\Lambda \bfx_c)} \\
&=\;\mbox{const'}-\tfrac{1}{2}\log|\Psi|-\tfrac{1}{2}
[\bfy_c^\top\Psi^{-1}\bfy_c-2\bfy_c^\top\Psi^{-1}\Lambda\bfx_c+
\bfx_c^\top\Lambda^\top\Psi^{-1}\Lambda\bfx_c] \\
&=\;\mbox{const'}-\tfrac{1}{2}\log|\Psi|-\tfrac{1}{2}[\bfy_c^\top\Psi^{-1}
\bfy_c-2\bfy_c^\top\Psi^{-1}\Lambda\bfx_c+
\trace{\Lambda^\top\Psi^{-1}\Lambda \bfx_c\bfx_c^\top}]
\end{split}
\]
%
Taking expectations over $q_c(\bfx_c)$\ldots
%
\[
=\;\mbox{const'}-\tfrac{1}{2}\log|\Psi|-\tfrac{1}{2}\big[\bfy_c^\top\Psi^{-1}
\bfy_c-2\bfy_c^\top\Psi^{-1}\Lambda\bmu_c+\trace{\Lambda^\top
\Psi^{-1} \Lambda (\bmu_c \bmu_c^\top + \Sigma)}\big].
\]
%
Note that we don't need to know everything about $q$, just the
expectations of $\bfx$ and $\bfx\bfx^\top$ under $q$ (i.e.\ the expected
sufficient statistics).

\end{frame}

\begin{frame}
\frametitle{The M step for Factor Analysis (cont.)}

$\displaystyle{\cal F}=\mbox{const'}-\frac{N}{2} \log |\Psi| -
\tfrac{1}{2} \sum_c \left[ \bfy_c^\top \Psi^{-1} \bfy_c - 2 \bfy_c^\top \Psi^{-1}
  \Lambda \bmu_c  + \operatorname{trace}(\Lambda^\top \Psi^{-1} \Lambda (\bmu_c
  \bmu_c^\top + \Sigma))\right] $ \\
Taking derivatives w.r.t. $\Lambda$ and $\Psi^{-1}$, using
$\tfrac{\partial\trace{AB}}{\partial B}=A^\top$ and
$\tfrac{\partial\log|A|}{\partial A}=A^{-\top}$:
%
\[
\begin{split}
\frac{\partial{\cal F}}{\partial\Lambda}\;&=\;
\Psi^{-1}\sum_c\bfy_c\bmu_c^\top -
\Psi^{-1} \Lambda \Big(N \Sigma + \sum_c \bmu_c \bmu_c^\top \Big)\;=\;0 \\
\Blue{\Lambda}\;&\Blue{=\;\Big(}\Red{N\Sigma+}
\Blue{\sum_c \bmu_c \bmu_c^\top \Big)^{-1}\sum_c \bfy_c \bmu_c^\top}\\
\frac{\partial{\cal F}}{\partial\Psi^{-1}}\;&=\;\frac{N}{2}\Psi
-\frac{1}{2}\sum_c\left[ \bfy_c \bfy_c^\top - \Lambda \bmu_c \bfy_c^\top - \bfy
_c \bmu_c^\top \Lambda^\top + \Lambda (\bmu_c
 \bmu_c^\top + \Sigma ) \Lambda^\top\right] \\
\Psi\;&=\;\frac{1}{N} \sum_c \left[ \bfy_c \bfy_c^\top - \Lambda \bmu_c
  \bfy_c^\top - \bfy_c \bmu_c^\top \Lambda^\top + \Lambda (\bmu_c \bmu_c^\top
 + \Sigma ) \Lambda^\top\right] \\
&\Blue{=\;\Red{\Lambda \Sigma \Lambda^\top +}\;\frac{1}{N} \sum_c
(\bfy_c - \Lambda
\bmu_c) (\bfy_c - \Lambda \bmu_c)^\top} \hspace{0.6in} \mbox{(squared residual
s)}
\end{split}
\]
%
Note: we should actually only take derivarives w.r.t.\ $\Psi_{dd}$ since
$\Psi$ is diagonal. \\[-0.3ex] When $\Red{\Sigma \rightarrow 0}$ these become
the equations for linear regression!
\end{frame}

\begin{frame}
\frametitle{Limitations and Mixtures}

So far, we have assumed that the data is reasonably well approximated by a
Gaussian.

\centerline{\includegraphics[width=\textwidth]{mog-approximations}}

A mixture distribution has a single discrete latent variable:
\[
s_i\;\sim\;{\rm Discrete}[\boldsymbol\pi] \hspace{5mm}
\bfy_i | s_i\;\sim\;p(\bfy_i|\theta_{s_i})
\]

Mixtures arise naturally when observations from different sources have
been collated.

They can also be used to \emph{approximate} arbitrary distributions.
\end{frame}

\begin{frame}
\frametitle{Clustering with MoG}

\centerline{\includegraphics[height=0.85\textheight]{mog-cluster-data}}
\end{frame}

\begin{frame}
\frametitle{Clustering with MoG}

\centerline{\includegraphics[height=0.85\textheight]{mog-cluster-ellipses}}
\end{frame}



\begin{frame}
\frametitle{Likelihood for the Mixture of Gaussians model}

Likelihood:
%
\[
\begin{split}
p(\bfy_1,\ldots,\bfy_n|\theta)\;=&\;\prod_{c=1}^n\sum_{i=1}^k\pi_i\;{\cal N}
(\bfy_c|\bmu_i,\Sigma_i)\\
=&\;\prod_{c=1}^n\sum_{i=1}^k\pi_i\;|2\pi\Sigma_i|^{-1/2}\exp\big(-
(\bfy_c-\bmu_i)^\top\Sigma_i^{-1}(\bfy_c-\bmu_i)\big).
\end{split}
\]
Here, $\pi_i$ are the \Blue{mixing proportions}, (non-negative, sum to one) and
the parameters are collected in
$\theta=(\boldsymbol{\pi},\bmu_1,\ldots,\bmu_k,\Sigma_1,\ldots,\Sigma_k)$.
\end{frame}

\begin{frame}
\frametitle{Mixture of Gaussians, E-step}

The likelihood (for a single data point) in the mixture of Gaussians model is
%
\[
p(\bfy|\theta)\;=\;\sum_{j=1}^kp(\bfy,h|\theta)\;=\;
\sum_{j=1}^kp(h=j|\theta)p(\bfy|h=j,\theta),
\]
%
where
$\theta=(\pi_1,\ldots,\pi_k,\bmu_1,\ldots,\bmu_k,\Sigma_1,\ldots,\Sigma_k)$
are the parameters, and $h$ is the hidden, or latent, variable.

In the E-step we maximize the lower bound functional ${\cal
  F}(\Blue{q(H)},\Red{\theta})$ wrt $\Blue{q(H)}$ for fixed $\Red{\theta}$. We
have seen that this is equivalent to setting $\Blue{q(H)}$ equal to the
posterior:
%
\[
\Blue{q(H)}\;=\;p(h|\bfy,\theta)\;=\;
\frac{p(\bfy|h,\theta)p(h|\theta)}{p(\bfy|\theta)}
\;\propto\;p(\bfy|h,\theta)p(h|\theta).
\]
%
Thus, the \emph{responsibilities} are:
%
\[
\Blue{r_j}\;=\;p(h=j|\bfy,\theta)\;\propto\;\pi_j|\Sigma_j|^{1/2}
\exp\Big(\!-(\bfy-\bmu_j)^\top\Sigma_j^{-1}(\bfy-\bmu_j)/2\Big),
\]
%
normalized such that $\sum_jr_j=1$. For multiple data points, responsibilities
are computed analogously: $r_{cj}=p(h_c=j|\bfy_c,\theta)$.
\end{frame}

\begin{frame}
\frametitle{Mixture of Gaussians, M-step}

In the M-step, we maximize the lower bound functional ${\cal
  F}(\Red{q(H)},\Blue{\theta})$ wrt. $\Blue{\theta}$. Recall, that this is 
equivalent to maximizing
%
\[
E(\theta)\;=\;\int q(H)\log\big(p(H,V|\theta)\big)dH\;=
\;\int q(H)\sum_c\log\big(p(H_c,V_c|\theta)\big)dH,
\]
%
for a fixed $q(H)$. For mixtures of Gaussians:
%
\[
E(\theta)\;=\;\sum_{c,j}r_{cj}\Big[\log(\pi_j)-\tfrac{1}{2}\log|\Sigma_j|
-(\bfy_c-\bmu_j)^\top\Sigma_j^{-1}(\bfy_c-\bmu_j)/2\Big].
\]
\end{frame}

\begin{frame}
\frametitle{Mixture of Gaussians, M-step, cont.}

Optimizing wrt the $\Red{\theta}$ parameters (and $S_j=\Sigma_j^{-1}$):
%
\[
\begin{split}
\frac{\partial E(\theta)}{\partial \bmu_j}\;&=\;\sum_c
r_{cj}\Sigma_j^{-1}(\bfy_c-\bmu)\;=\;{\bf 0}\;\;\Longrightarrow\;\;
\bmu_j\;=\;\sum_c r_{cj}\bfy_c/(\sum_c r_{cj})\\
\frac{\partial E(\theta)}{\partial
  S_j}\;&=\;\frac{1}{2}\sum_cr_{cj}\Big[S_j^{-1}-(\bfy_c-\bmu_j)(\bfy_c-\bmu_j
)^\top\Big]\;=\;{\bf  0}\\
&\hspace{20mm}\Longrightarrow\;\;
\Sigma_j\;=\;\sum_{c=1}^nr_{cj}(\bfy_c-\bmu_j)(\bfy_c-\bmu_j)^\top/(\sum_c r_{
cj}),\\
\frac{\partial E(\theta)}{\partial \pi_j}\;&=\;\sum_c\frac{r_{cj}}{\pi_j}-
\lambda\;=0\;\;\Longrightarrow\;\;\pi_j\;=\;\sum_cr_{cj}/(\sum_{c,j}r_{cj}),
\end{split}
\]
%
where $\lambda$ is the Lagrange multiplier which ensures that the mixture
normalizes $\sum_j\pi_j=1$.
\end{frame}

\begin{frame}
\frametitle{Mixture of Gaussian issues}

The EM algorithm converges to a \emph{local} maximum of the likelihood.
There may be many local maxima.

There could be many bad local maxima, ie with low values of the likelihood.

\Blue{In fact, we are not interested in the global optimum!}

Another problem: ``How many mixture components?''
\end{frame}






>>>>>>> .r224
\end{document}