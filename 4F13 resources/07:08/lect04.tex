\subtitle{4F13: Machine Learning}
\author[Ghahramani \& Rasmussen]{Zoubin Ghahramani and Carl Edward Rasmussen}
\institute[CUED]{Department of Engineering, University of Cambridge}

\parskip=1.5ex

\newcommand{\maketilde}{\raisebox{0.4ex}{\tiny $\sim$}}
\newcommand{\bfa}{\mathbf a}
\newcommand{\bfb}{\mathbf b}
\newcommand{\bff}{\mathbf f}
\newcommand{\bfm}{\mathbf m}
\newcommand{\bfp}{\mathbf{p}}
\newcommand{\bfs}{\mathbf s}
\newcommand{\bfu}{\mathbf u}
\newcommand{\bfx}{\mathbf x}
\newcommand{\bfy}{\mathbf y}
\newcommand{\bfv}{\mathbf v}

\newcommand{\bmu}{{\boldsymbol{\mu}}}
\newcommand{\btheta}{\boldsymbol{\theta}}
\newcommand{\balpha}{\boldsymbol{\alpha}}
\newcommand{\bphi}{\boldsymbol{\phi}}
\newcommand{\bpi}{\boldsymbol{\pi}}
\newcommand{\blambda}{\boldsymbol{\lambda}}

\newcommand{\N}{\mathcal{N}}
\newcommand{\R}{\mathbb{R}}
\newcommand{\T}{{\scriptsize^{\top}}}
\newcommand{\D}{\mathcal{D}}
\newcommand{\F}{\mathcal{F}}
\newcommand{\E}{\mathbb{E}}
\newcommand{\V}{\mathbb{V}}
\newcommand{\M}{\mathcal{M}}
\newcommand{\KL}{\mathcal{KL}}
\newcommand{\cut}[1]{}
\newcommand{\trace}{\operatorname{trace}}

\newcommand{\argmax}{\operatorname{argmax}}
\newcommand{\argmin}{\operatorname{argmin}}
\newcommand{\ci}{{\bot\negthickspace\negthickspace\bot}} % conditional indep.
\newcommand{\neigh}{\operatorname{ne}}
\newcommand{\vectr}[2]{  \left[ \!\!\begin{array}{c} #1 \\
      #2 \end{array} \!\!\right]}
\newcommand{\deff}{\stackrel{\mathrm{def}}{=}}
\newcommand{\deldel}[2]{\frac{\partial #1}{\partial #2}}


\definecolor{mypine}{rgb}{0.05,0.45,0.05}
\definecolor{mycyan}{rgb}{0.0,0.4,0.4}
\newcommand{\Red}{\textcolor{red}}
\newcommand{\Blue}{\textcolor{blue}}
\newcommand{\Green}{\textcolor{mypine}}
\newcommand{\PineGreen}{\textcolor{mypine}}
\newcommand{\Magenta}{\textcolor{magenta}}


\begin{document}

% divide all sizes by around 2.2

\titleslide{Lecture 4: Graphical Models}{January 30th, 2008}

%%%%%%%%%%%%%%%%%%%%%%%%%%%%%%%%%%%%%%%%%%%%%%%%%%%%%%%%%%%%%%%%%%%%%%
\begin{frame}
\frametitle{Three main kinds of graphical models}

\begin{center}
\begin{tabular}{ccc}
\includegraphics[width=35mm]{factorgraphb} &
\includegraphics[width=35mm]{undir2} &
\includegraphics[width=35mm]{dir2} \\
factor graph & undirected graph & directed graph 
\end{tabular}
\end{center}

\vspace*{2em}

\begin{itemize}
\item Nodes correspond to random variables
\item Edges represent statistical dependencies between the variables
\end{itemize}

\end{frame}

%%%%%%%%%%%%%%%%%%%%%%%%%%%%%%%%%%%%%%%%%%%%%%%%%%%%%%%%%%%%%%%%%%%%%%
\begin{frame}
\frametitle{Why do we need graphical models?}

\begin{itemize}
\item Graphs are an \Blue{\bf intuitive} way of representing and visualising the
  relationships   between many variables. (Examples: family trees,
  electric circuit diagrams, neural networks)
\item A graph allows us to abstract out the \Blue{\bf conditional independence}
  relationships between the variables from the details of their
  parametric forms. Thus we can answer questions like: ``Is $A$ dependent
  on $B$ given that we know the value of $C$ ?'' just by looking at the
  graph. 
\item Graphical models allow us to define general \Blue{\bf
    message-passing algorithms} that implement probabilistic inference
    efficiently. Thus we can answer queries like ``What is
    $P(A|C=c)$?'' without enumerating all settings of all variables in
    the model.
\end{itemize}

\vfill

\centerline{\Green{Graphical models = statistics $\times$ graph theory
    $\times$ computer science.}}



\end{frame}

%%%%%%%%%%%%%%%%%%%%%%%%%%%%%%%%%%%%%%%%%%%%%%%%%%%%%%%%%%%%%%%%%%%%%%
\begin{frame}
\frametitle{Conditional Independence}

{\bf Conditional Independence:} 

\[
X\ci Y|V \;\; \Leftrightarrow \;\; p(X|Y,V)=p(X|V)
\]
 when $p(Y,V)>0$. Also 
\[
X\ci Y|V \;\; \Leftrightarrow \;\; p(X,Y|V)=p(X|V) \, p(Y|V)
\] 

In general we can think of conditional independence between {\bf sets
  of variables}:
\[
{\cal X} \ci {\cal Y} | {\cal V} \;\; \Leftrightarrow \;\; 
p({\cal X},{\cal Y}|{\cal V})=p({\cal X}|{\cal V}) \, p({\cal Y}|{\cal
  V})
\]

{\bf Marginal Independence:}

\[
X\ci Y \;\; \Leftrightarrow \;\; X\ci Y|\emptyset \;\; \Leftrightarrow
\;\; p(X,Y)=p(X) \, p(Y)
\]



\end{frame}
%%%%%%%%%%%%%%%%%%%%%%%%%%%%%%%%%%%%%%%%%%%%%%%%%%%%%%%%%%%%%%%%%%%%%%
\begin{frame}
\frametitle{Conditional and Marginal Independence (Examples)}

\begin{itemize}
\item Amount of Speeding Fine $\ci$ Type of Car $|$ Speed 

\item Lung Cancer $\ci$ Yellow Teeth $|$ Smoking

\item (Position, Velocity)$_{t+1}$ $\ci$ (Position, Velocity)$_{t-1}$ $|$
(Position, Velocity)$_{t}$, Acceleration$_t$

\item Child's Genes $\ci$ Grandparents' Genes $|$ Parents' Genes
  

\item Ability of Team A $\ci$ Ability of Team B

\item \Red{not} ( Ability of Team A $\ci$ Ability of Team B $|$ Outcome of A
vs B Game )

\end{itemize}



\end{frame}
%%%%%%%%%%%%%%%%%%%%%%%%%%%%%%%%%%%%%%%%%%%%%%%%%%%%%%%%%%%%%%%%%%%%%%
\begin{frame}
\frametitle{Factor Graphs}

\parbox{65mm}{
\begin{center}
\begin{tabular}{cc}
\includegraphics[width=30mm]{factorgraphb} & 
\includegraphics[width=30mm]{factorgraph2b}\\
(a)  &  (b) 
\end{tabular}
\end{center}} \hspace{1.5mm}
\parbox{55mm}{Two types of nodes:
\begin{itemize}
\item The circles in a factor graph represent random variables (e.g.\ $A$). 
\item The filled dots represent factors in the joint distribution
  (e.g.\ $g_1(\cdot)$).
\end{itemize}}

\vspace{1ex}

\hspace{1em} 
(a) $P(A,B,C,D,E) = \frac{1}{Z} g_1(A,C) g_2(B,C,D) g_3(C,D,E) $ \\[1ex]
\hspace*{1em} 
(b) $P(A,B,C,D,E) = \frac{1}{Z} g_1(A,C) g_2(B,C) g_3(C,D) g_4(B,D)
g_5(C,E) g_6(D,E)$\\

The $g_i$ are non-negative functions of their arguments, and $Z$ is a
normalization constant.

E.g. in (a), if all variables are discrete and take values in ${\cal A} \times
{\cal B} \times {\cal C} \times {\cal D} \times {\cal E}$:
{\small 
\[
Z=\sum_{a \in {\cal A}} \sum_{b \in {\cal B}} \sum_{c \in {\cal C}}
\sum_{d \in {\cal D}} \sum_{e \in {\cal E}} g_1(A=a,C=c) g_2(B=b,C=c,D=d)
g_3(C=c,D=d,E=e)
\]}

Two nodes are \Red{neighbors} if they share a common factor. 


\end{frame}
%%%%%%%%%%%%%%%%%%%%%%%%%%%%%%%%%%%%%%%%%%%%%%%%%%%%%%%%%%%%%%%%%%%%%%
\begin{frame}
\frametitle{Factor Graphs}
\vspace{-3ex}
\begin{center}
\begin{tabular}{ccc}
\includegraphics[width=30mm]{factorgraphb} & \hspace{2em}
\includegraphics[width=30mm]{factorgraph2b}\\
(a)  &  (b) 
\end{tabular}
\end{center}
\vspace{-0.3ex}

The circles in a factor graph represent random variables. \\
The filled dots represent factors in the joint distribution. \\

\hspace{1em} 
(a) $P(A,B,C,D,E) = \frac{1}{Z} g_1(A,C) g_2(B,C,D) g_3(C,D,E) $ \\[1ex]
\hspace*{1em} 
(b) $P(A,B,C,D,E) = \frac{1}{Z} g_1(A,C) g_2(B,C) g_3(C,D) g_4(B,D)
g_5(C,E) g_6(D,E)$\\

Two nodes are \Red{neighbors} if they share a common factor. \\

{\bf Definition:} A \Blue{\em path} is a sequence of neighboring nodes.\\

{\bf Fact:} $X \ci Y | {\cal V}$ if \Blue{every path} between $X$ and $Y$
contains some node $V \in {\cal V}$ \\

{\bf Corollary:} Given the neighbors of $X$, the variable $X$ is \Red{conditionally
independent} of all other variables: $X \ci Y | \neigh(X)$, \ \ $\forall Y \notin
\{X \cup \neigh(X)\}$


\end{frame}
%%%%%%%%%%%%%%%%%%%%%%%%%%%%%%%%%%%%%%%%%%%%%%%%%%%%%%%%%%%%%%%%%%%%%%
\begin{frame}
\frametitle{Proving Conditional Independence}

Conditional independence:
\begin{equation}
X\ci Y|V \;\; \Leftrightarrow \;\; p(X|Y,V)=p(X|V)
\label{eq1}
\end{equation}
Assume:
\begin{equation}
P(X,Y,V) = \frac{1}{Z} g_1(X,V) g_2(Y,V) 
\label{eq2}
\end{equation}
Then summing (\ref{eq2}) over $X$ we get:
\begin{equation}
P(Y,V) = \frac{1}{Z} [ \sum_X g_1(X,V)] g_2(Y,V) 
\label{eq3}
\end{equation}
Dividing (\ref{eq2}) by (\ref{eq3}) we get:
\begin{equation}
P(X|Y,V) = \frac{g_1(X,V)}{\sum_X g_1(X,V)}
\label{eq4}
\end{equation}
Since the rhs.\ of (\ref{eq4}) doesn't depend on $Y$, it follows
that $X$ is independent of $Y$ given $V$. \\
Therefore factorizaton (\ref{eq2}) implies conditional independence (\ref{eq1}).


\end{frame}
%%%%%%%%%%%%%%%%%%%%%%%%%%%%%%%%%%%%%%%%%%%%%%%%%%%%%%%%%%%%%%%%%%%%%%
\begin{frame}
\frametitle{Undirected Graphical Models}

In an \Blue{Undirected Graphical Model}, the joint probability over all
variables can be written in a factored form:  
\[
P({\bf x}) = \frac{1}{Z} \prod_j g_j({\bf x}_{C_j})
\]
where ${\bf x}=(x_1, \ldots, x_K)$, and  
\[
C_j \subseteq \{ 1, \ldots, K\}
\] 
are subsets of the set of all variables, and ${\bf x}_S \equiv (x_k :
k \in S)$. \\

% This type of probabilistic model can be represented graphically.\\

{\bf Graph Specification:} Create a node for each variable. Connect nodes $i$
and $k$   if there exists a set $C_j$ such that  both $i \in C_j$ and
$k \in C_j$.  These sets form the \Blue{\em cliques} of the graph (fully
connected subgraphs).\\

{\bf Note:} Undirected Graphical Models are also called \Blue{\em Markov
Networks}. \\

Very similar to factor graphs.


\end{frame}

%%%%%%%%%%%%%%%%%%%%%%%%%%%%%%%%%%%%%%%%%%%%%%%%%%%%%%%%%%%%%%%%%%%%%%
\begin{frame}
\frametitle{Undirected Graphical Models}

\vspace*{-1.5ex}
\centerline{
% \begin{tabular}{cc}
\includegraphics[width=35mm]{undir2}
% &\parbox[b]{100mm}{Examples:\\ Boltzmann Machines\\ Markov Random
%  Fields{\vskip 20mm}} 
% \end{tabular}
}
\vspace*{0.1mm}

\[
P(A,B,C,D,E) = \frac{1}{Z} g_1(A,C) g_2(B,C,D) g_3(C,D,E)
\]

{\bf Fact:} $X \ci Y | {\cal V}$ if every path between $X$ and $Y$
contains some node $V \in {\cal V}$ \\

{\bf Corollary:} Given the neighbors of $X$, the variable $X$ is conditionally
independent of all other variables: $X \ci Y | \neigh(X)$, \ \ $\forall Y \notin
\{X \cup \neigh(X)\}$\\

{\bf Markov Blanket:} ${\cal V}$ is a Markov Blanket for $X$ iff $X\ci
Y|{\cal V}$ for all $Y \notin \{ X \cup {\cal V}\}$.\\

{\bf Markov Boundary:} minimal Markov Blanket $\equiv \neigh(X)$ for
undirected graphs and factor graphs




\end{frame}
%%%%%%%%%%%%%%%%%%%%%%%%%%%%%%%%%%%%%%%%%%%%%%%%%%%%%%%%%%%%%%%%%%%%%%
\begin{frame}
\frametitle{Comparing Undirected Graphs and Factor Graphs}

\begin{center}
\begin{tabular}{ccc}
\includegraphics[width=35mm]{undir2} &
\includegraphics[width=35mm]{factorgraphb} &
\includegraphics[width=35mm]{factorgraph2b} \\
(a) & (b) & (c) 
\end{tabular}
\end{center}

\vspace{1ex}

All nodes in (a), (b), and (c) have exactly the same neighbors and
therefore these three graphs represent exactly the same conditional
independence relationships. \\

(c) {\em also} represents the fact that the probability factors
into a product of pairwise functions. \\

Consider the case where each variables is discrete and can take on
$K$ possible values. Then the functions in (a) and (b) are tables with
${\cal O}(K^3)$ cells, whereas in (c) they are ${\cal O}(K^2)$. 



%%%%%%%%%%%%%%%%%%%%%%%%%%%%%%%%%%%%%%%%%%%%%%%%%%%%%%%%%%%%%%%%%%%%%%

\end{frame}

\begin{frame}
\frametitle{Problems with Undirected Graphs and Factor Graphs}

In UGs and FGs, many useful independencies are unrepresented---two
variables are  connected merely because some other variable depends on them:

\vspace*{4mm}
\centerline{\includegraphics[width=\textwidth]{undirvsdir}}
\vspace*{4mm}

This highlights the difference between {\bf marginal independence} and
{\bf conditional independence}.\\ 

$R$ and $S$ are marginally independent (i.e.\ given nothing), but they
are conditionally dependent given $G$.\\

\Blue{``Explaining Away''}: Observing that the spinkler is on, would
explain away the observation that the ground was wet, making it less
probable that it rained. 
 

\end{frame}
%%%%%%%%%%%%%%%%%%%%%%%%%%%%%%%%%%%%%%%%%%%%%%%%%%%%%%%%%%%%%%%%%%%%%%
\begin{frame}
\frametitle{Directed Acyclic Graphical Models (Bayesian Networks)}

\vspace{1em}
\centerline{\includegraphics[width=30mm]{dir2}}

A DAG Model / Bayesian network\footnote{``Bayesian networks'' can and
  often are learned using non-Bayesian (i.e. frequentist) methods;
  Bayesian networks (i.e.\ DAGs) do not require that parameter or structure
  learning use Bayesian methods.} corresponds to a factorization of the joint
probability distribution:
\[
p(A,B,C,D,E) = p(A) p(B) p(C|A,B) p(D|B,C) p(E|C,D)
\]
In general: \vspace*{-1em}
\[
p(X_1, \ldots, X_n) = \prod_{i=1}^n p(X_i|X_{{\mbox{pa}(i)}})
\]
where $\mbox{pa}(i)$ are the \Blue{parents} of node $i$.

\end{frame}
%%%%%%%%%%%%%%%%%%%%%%%%%%%%%%%%%%%%%%%%%%%%%%%%%%%%%%%%%%%%%%%%%%%%%%
\begin{frame}
\frametitle{Directed Acyclic Graphical Models (Bayesian Networks) }



% \frametitle{Bayesian Networks (Directed Graphical Models)}

\vspace*{-1ex} 
\hspace*{3in} \includegraphics[width=20mm]{dir2}

\vspace*{-3ex} 
{\bf Semantics:} $X\ci Y|{\cal V}$ if ${\cal V}$ \Red{d-separates} $X$
from $Y$\footnote{See also the ``Bayes Ball'' algorithm in the Appendix}.\\

{\bf Definition:} ${\cal V}$ \Red{d-separates} $X$ from $Y$ if {\em every} undirected
path\footnote{An undirected path ignores the direction of the edges.}
between $X$ and $Y$ is \Blue{\bf blocked} by ${\cal V}$. A path is blocked
by ${\cal V}$ if 
there is a node $W$ on the path such that either: 
%
\begin{enumerate}
\item $W$ has converging arrows along
the path ($\rightarrow W 
\leftarrow$)\footnote{Note that converging arrows {\em along the path}
only refers to what happens on that path.} and neither $W$ nor its
descendants are observed (in ${\cal V}$), or 
\item $W$ does not have converging arrows along the path ($\rightarrow W
\rightarrow$ or $\leftarrow W
\rightarrow$) and $W$ is observed ($W\in {\cal V}$).
\end{enumerate}

\vspace{1ex}

{\bf Corollary:} Markov Boundary for $X$: $\{\operatorname{parents}(X)\cup\operatorname{children}(X)\cup\operatorname{parents-of-children}(X)\}$.

\end{frame}
%%%%%%%%%%%%%%%%%%%%%%%%%%%%%%%%%%%%%%%%%%%%%%%%%%%%%%%%%%%%%%%%%%%%%%
\begin{frame}
\frametitle{Examples of D-Separation in DAGs}

\vspace*{-3ex} 
\hspace*{3in} \includegraphics[width=30mm]{dir2}

Examples:

\begin{itemize}
\item \PineGreen{$A \ci B $} since $A \rightarrow C \leftarrow B$ is
  blocked by $C$, $A
  \rightarrow C \rightarrow D \leftarrow B$ is blocked by $D$, etc.
\item \Red{not ($A \ci B | C$ )} since $A \rightarrow C \leftarrow B$ is not
blocked.
\item \PineGreen{$A \ci D | \{ B, C \}$} since $A \rightarrow C
  \rightarrow D$ is 
  blocked by $C$,  $A \rightarrow C \leftarrow B \rightarrow D$ is
  blocked by $D$, and $A \rightarrow C \rightarrow E \leftarrow D$ is
  blocked by $C$.
\item \Red{not ($A \ci B | E $)} since $A \rightarrow C \leftarrow B$
  is not blocked.
\end{itemize}


\end{frame}
%%%%%%%%%%%%%%%%%%%%%%%%%%%%%%%%%%%%%%%%%%%%%%%%%%%%%%%%%%%%%%%%%%%%%%
\begin{frame}
\frametitle{Directed Graphs for Statistical Models: \\ Plate Notation}

\centerline{\parbox[t!]{1.2in}{\includegraphics[scale=0.4]{gm-gaussian}}
\hspace{2ex} $\equiv$ \hspace{2ex}
\parbox[t!]{1.8in}{\includegraphics[scale=0.4]{gm-gaussian-noplate}}}

\vspace{3ex}

A data set of $N$ points generated from a Gaussian:
\[
p(x_1, \ldots, x_N, \mu, \sigma) = p(\mu) p(\sigma) \prod_{n=1}^N
p(x_n|\mu,\sigma)
\]


\end{frame}
%%%%%%%%%%%%%%%%%%%%%%%%%%%%%%%%%%%%%%%%%%%%%%%%%%%%%%%%%%%%%%%%%%%%%%
\begin{frame}
\frametitle{Summary}

\begin{itemize}
\item Three kinds of graphical models: directed, undirected, factor \\
  (there {\em are} other important classes, e.g.\ directed mixed graphs)
\item Marginal and conditional independence
\item Markov boundaries and d-separation
\item Plate notation
\end{itemize}


\end{frame}
%%%%%%%%%%%%%%%%%%%%%%%%%%%%%%%%%%%%%%%%%%%%%%%%%%%%%%%%%%%%%%%%%%%%%%
\begin{frame}
\frametitle{Appendix}



\end{frame}
%%%%%%%%%%%%%%%%%%%%%%%%%%%%%%%%%%%%%%%%%%%%%%%%%%%%%%%%%%%%%%%%%%%%%%
\begin{frame}
\frametitle{From Directed Trees to Undirected Trees}

\centerline{\includegraphics[width=1.7in]{btmt}}

\vspace*{-0.1in}
\begin{equation*}
\begin{split}
p(x_1,x_2,\ldots,x_7)\;=&\;\;p(x_3)p(x_1|x_3)p(x_2|x_3)p(x_4|x_3)p(x_5|x_4)p(x_6|x_4)p(x_7|x_4)\\[1ex]
=&\;\frac{p(x_1,x_3)p(x_2,x_3)p(x_3,x_4)p(x_4,x_5)p(x_4,x_6)p(x_4,x_7)}{p(x_3)p(x_3)p(x_4)p(x_4)p(x_4)}\\[1ex]
=&\;\frac{\mbox{product of cliques}}{\mbox{product of clique intersections}}\\[1ex]
=&\;g_1(x_1,x_3)g_2(x_2,x_3)g_3(x_3,x_4)g_4(x_4,x_5)g_5(x_4,x_6)g_6(x_4,x_7)=\\
=&\; \prod_ig_i(C_i)
\end{split}
\end{equation*}

Any directed tree can be converted into an undirected tree
representing the same conditional independence relationships, and
viceversa. 



\end{frame}
%%%%%%%%%%%%%%%%%%%%%%%%%%%%%%%%%%%%%%%%%%%%%%%%%%%%%%%%%%%%%%%%%%%%%%
\begin{frame}
\frametitle{Expressive Power of Directed and Undirected Graphs}

\centerline{
\begin{tabular}{cc}
\includegraphics[width=1.2in]{exm}&\qquad \parbox[b]{45mm}{No
  Directed Graph (Bayesian network) can represent these and only these independencies{\vskip 9mm}}
\end{tabular}
} \vspace{2mm}
%
No matter how we direct the arrows there will always be two
non-adjacent parents sharing a common child $\Longrightarrow$
dependence in Directed Graph but independence in Undirected 
Graph. \vspace{6mm}

\centerline{
\begin{tabular}{cc}
\includegraphics[width=1.2in]{exb}&\qquad \parbox[b]{45mm}{No
  Undirected Graph or Factor Graph can represent these and only these
  independencies{\vskip 2mm}} 
\end{tabular}
}

Directed graphs are better at expressing causal generative models,
undirected graphs are better at representing soft constraints between
variables. 


\end{frame}

%%%%%%%%%%%%%%%%%%%%%%%%%%%%%%%%%%%%%%%%%%%%%%%%%%%%%%%%%%%%%%%%%%%%%%
\begin{frame}
\frametitle{Appendix: Some Examples of Directed Graphical Models}
 
\centerline{
\begin{tabular}{cc}
\includegraphics[width=1.2in]{fa2col}&\qquad \parbox[b]{45mm}{factor analysis\\ probabilistic PCA{\vskip 5mm}}
\end{tabular}
}

\centerline{
\begin{tabular}{cc}
\includegraphics[width=1.8in]{ssm}&\qquad \parbox[b]{45mm}{hidden Markov models\\ linear dynamical systems{\vskip 2mm}}
\end{tabular}
}

\centerline{
\begin{tabular}{cc}
\includegraphics[width=1.8in]{swssmalt}&\qquad \parbox[b]{45mm}{switching state-space models{\vskip 2mm}}
\end{tabular}
}



\end{frame}
%%%%%%%%%%%%%%%%%%%%%%%%%%%%%%%%%%%%%%%%%%%%%%%%%%%%%%%%%%%%%%%%%%%%%%
\begin{frame}
\frametitle{Appendix: Examples of Undirected Graphical Models}

\begin{itemize}
\item Markov Random Fields (used in Computer Vision)
\hspace{12mm} \raisebox{-3ex}{\includegraphics[width=12mm]{MRF2}}
\item Exponential Language Models (used in Speech and Language Modelling) 
\[
p(s) = \frac{1}{Z} p_0(s) \exp \left\{ \sum_i \lambda_i f_i(s) \right\}
\]
\item Products of Experts (widely applicable)
\[
p({\bf x}) = \frac{1}{Z} \prod_j p_j({\bf x} | \theta_j)
\]
\item Boltzmann Machines (a kind of Neural Network/Ising Model)
\hspace{12mm} \raisebox{-3ex}{\includegraphics[width=12mm]{hopfield}}
\end{itemize}


\end{frame}

%%%%%%%%%%%%%%%%%%%%%%%%%%%%%%%%%%%%%%%%%%%%%%%%%%%%%%%%%%%%%%%%%%%%%%
\begin{frame}
\frametitle{Appendix: Clique Potentials and Undirected Graphs}

\vspace{-1ex}
{\bf Definition:} a \emph{clique} is a fully connected subgraph. By clique we usually mean maximal clique (i.e.\ not contained within another clique)\\[1ex]

\parbox{3.3in}{
$C_i$ denotes the set of variables in the $i^{th}$ clique.
\[ p(x_1,\ldots,x_K)=\frac{1}{Z}\prod_i g_i({\bf x}_{C_i})\] where
$ Z=\sum_{x_1\cdots x_K}\prod_i g_i({\bf x}_{C_i}) $ is the normalization.\\

Associated with each clique $C_i$ is a non-negative function $g_i({\bf
x}_{C_i})$  which measures ``compatibility'' between settings of  the
variables. \\

{\bf Example:} Let $C_1 = \{A,C\}, A \in \{0,1\}, C \in \{0,1\}$ \\
What does this mean?
}
\parbox{1.8in}{
\centerline{\includegraphics[width=15mm]{clique}}

\vspace{3ex}

\begin{center}
\begin{tabular}{cc|c}
$A$ & $C$ & $g_1(A,C)$ \\ \hline
0 & 0 & 0.2 \\
0 & 1 & 0.6 \\
1 & 0 & 0.0 \\
1 & 1 & 1.2 
\end{tabular}
\end{center}
}


\end{frame}
%%%%%%%%%%%%%%%%%%%%%%%%%%%%%%%%%%%%%%%%%%%%%%%%%%%%%%%%%%%%%%%%%%%%%%
\begin{frame}
\frametitle{Appendix: Hammersley--Clifford Theorem (1971)}

{\bf Theorem:} A probability function $p$ formed by a normalized product of
positive functions on cliques of $G$ is a Markov Field relative to
$G$.\\

{\bf Definition:} The distribution $p$ is a {\em Markov Field relative to
$G$} if all conditional independence relations represented by $G$ are true
of $p$. \\

$G$ represents the following CI relations: If $V \in \cal V$ lies on
\emph{all}  paths between $X$ and $Y$ in $G$, then $X\ci Y|{\cal
V}$. \\ 


{\bf Proof:} We need to show that if $p$ is a product of functions on
cliques of $G$ then a variable is conditionally independent of its
non-neighbors in $G$ given its neighbors in $G$. That is: 
$\neigh(x_{\ell})$ is a Markov Blanket for~$x_{\ell}$. Let $x_m \notin
\{x_{\ell} \cup \neigh(x_{\ell}) \}$
%
\begin{equation*}
\begin{split}
p(x_{\ell},x_m,\ldots) =& \frac{1}{Z}\prod_i g_i({\bf x}_{C_i})=\frac{1}{Z}\prod_{i:\ell \in C_i}
g_i({\bf x}_{C_i})\prod_{j:\ell \notin C_j}g_j({\bf x}_{C_j})\\
=& \frac{1}{Z'}f_1\big(x_{\ell},\neigh(x_\ell)\big) \; f_2\big(\neigh(x_{\ell}),x_m \big) =
\frac{1}{Z''} p(x_{\ell}|\neigh(x_{\ell}))\; p(x_m|\neigh(x_{\ell})) 
\end{split}
\end{equation*}
%
It follows  that: \hspace{1mm}
%
$ \displaystyle p(x_{\ell},x_m|\neigh(x_{\ell})) = p(x_{\ell}|\neigh(x_{\ell}))\; p(x_m|\neigh(x_{\ell})) \Leftrightarrow 
\Blue{x_{\ell} \ci x_m | \neigh(x_{\ell})}.$


\end{frame}
%%%%%%%%%%%%%%%%%%%%%%%%%%%%%%%%%%%%%%%%%%%%%%%%%%%%%%%%%%%%%%%%%%%%%%
\begin{frame}
\frametitle{Appendix: The ``Bayes-ball'' algorithm}

\centerline{\includegraphics[width=35mm]{dir2}}

\Red{Game:} can you get a ball from $X$ to $Y$ without being blocked
by ${\cal V}$? \\

Depending on the direction the ball came from and the type of node, the
ball can {\bf pass through} (from a parent to all children, from a
child to all parents), {\bf bounce back} (from any parent to all
parents, or from any child to all children), or be {\bf blocked}. 

\begin{itemize}
\item An unobserved (hidden) node ($W \notin \; {\cal V}$) passes balls through but also
  bounces back balls from children. 
\item An observed (given) node ($W \in {\cal V}$) bounces back balls from
  parents but blocks balls from children.
\end{itemize}
\end{frame}

\end{document}