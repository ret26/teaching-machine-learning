\subtitle{4F13: Machine Learning}
\author[Ghahramani \& Rasmussen]{Zoubin Ghahramani and Carl Edward Rasmussen}
\institute[CUED]{Department of Engineering, University of Cambridge}

\parskip=1.5ex

\newcommand{\maketilde}{\raisebox{0.4ex}{\tiny $\sim$}}
\newcommand{\bfa}{\mathbf a}
\newcommand{\bfb}{\mathbf b}
\newcommand{\bff}{\mathbf f}
\newcommand{\bfm}{\mathbf m}
\newcommand{\bfp}{\mathbf{p}}
\newcommand{\bfs}{\mathbf s}
\newcommand{\bfu}{\mathbf u}
\newcommand{\bfx}{\mathbf x}
\newcommand{\bfy}{\mathbf y}
\newcommand{\bfv}{\mathbf v}

\newcommand{\bmu}{{\boldsymbol{\mu}}}
\newcommand{\btheta}{\boldsymbol{\theta}}
\newcommand{\balpha}{\boldsymbol{\alpha}}
\newcommand{\bphi}{\boldsymbol{\phi}}
\newcommand{\bpi}{\boldsymbol{\pi}}
\newcommand{\blambda}{\boldsymbol{\lambda}}

\newcommand{\N}{\mathcal{N}}
\newcommand{\R}{\mathbb{R}}
\newcommand{\T}{{\scriptsize^{\top}}}
\newcommand{\D}{\mathcal{D}}
\newcommand{\F}{\mathcal{F}}
\newcommand{\E}{\mathbb{E}}
\newcommand{\V}{\mathbb{V}}
\newcommand{\M}{\mathcal{M}}
\newcommand{\KL}{\mathcal{KL}}
\newcommand{\cut}[1]{}
\newcommand{\trace}{\operatorname{trace}}

\newcommand{\argmax}{\operatorname{argmax}}
\newcommand{\argmin}{\operatorname{argmin}}
\newcommand{\ci}{{\bot\negthickspace\negthickspace\bot}} % conditional indep.
\newcommand{\neigh}{\operatorname{ne}}
\newcommand{\vectr}[2]{  \left[ \!\!\begin{array}{c} #1 \\
      #2 \end{array} \!\!\right]}
\newcommand{\deff}{\stackrel{\mathrm{def}}{=}}
\newcommand{\deldel}[2]{\frac{\partial #1}{\partial #2}}


\definecolor{mypine}{rgb}{0.05,0.45,0.05}
\definecolor{mycyan}{rgb}{0.0,0.4,0.4}
\newcommand{\Red}{\textcolor{red}}
\newcommand{\Blue}{\textcolor{blue}}
\newcommand{\Green}{\textcolor{mypine}}
\newcommand{\PineGreen}{\textcolor{mypine}}
\newcommand{\Magenta}{\textcolor{magenta}}


\begin{document}

\titleslide{Lecture 5: Graphical Models: Inference}{February 1st, 2008}


%%%%%%%%%%%%%%%%%%%%%%%%%%%%%%%%%%%%%%%%%%%%%%%%%%%%%%%%%%%%%%%%%%%%%%
\begin{frame}
\frametitle{Inference in a graphical model}

%\vspace*{-0.12in}
Consider the following graph: \includegraphics[width=25mm]{dir2}
which represents: \vspace*{-1ex}
\[
p(A,B,C,D,E) = p(A) p(B) p(C|A,B) p(D|B,C) p(E|C,D)
\]
\Red{\bf Inference}: evaluate the probability distribution over some set
of variables, given the values of another set of variables. \\

For example, how can we compute $P(A|C=c)$? Assume each variable is binary. \\

\Red{\bf Naive method:}
{\small
\begin{eqnarray*}
p(A,C=c) &=& \sum_{B,D,E} p(A,B,C=c,D,E) \hspace{2em} \mbox{[16 operations]} \\ 
p(C=c) &=& \sum_{A} p(A,C=c) \hspace{2em} \mbox{[2 operations]} \\
p(A|C=c) &=& \frac{p(A,C=c)}{p(C=c)} \hspace{2em} \mbox{[2 operations]} 
\end{eqnarray*}
}
\Blue{Total: 16+2+2 = 20 operations}

\end{frame}
%%%%%%%%%%%%%%%%%%%%%%%%%%%%%%%%%%%%%%%%%%%%%%%%%%%%%%%%%%%%%%%%%%%%%%
\begin{frame}
\frametitle{Inference in a graphical model}

% \vspace*{-0.12in}
Consider the following graph: \includegraphics[width=25mm]{dir2}
which represents: \vspace*{-1ex}
\[
p(A,B,C,D,E) = p(A) p(B) p(C|A,B) p(D|B,C) p(E|C,D)
\]
Computing $p(A|C=c)$.\\

\Red{\bf More efficient method:}
{\small 
\begin{eqnarray*}
p(A,C=c) &=& \sum_{B,D,E} p(A)p(B)p(C=c|A,B)p(D|B,C=c)p(E|C=c,D)  \\
&=& \sum_{B} p(A)p(B)p(C=c|A,B) \sum_D p(D|B,C=c) \sum_E p(E|C=c,D)  \\
&=& \sum_{B} p(A)p(B)p(C=c|A,B) \hspace{2em} \mbox{[4 operations]}
\end{eqnarray*}
}
\Blue{Total: 4+2+2 = 8 operations}\\

Belief propagation methods use the conditional independence
relationships in a graph to do efficient inference (for singly
connected graphs, \Red{exponential} gains in efficiency!).
\end{frame}
%%%%%%%%%%%%%%%%%%%%%%%%%%%%%%%%%%%%%%%%%%%%%%%%%%%%%%%%%%%%%%%%%%%%%%
\begin{frame}
\frametitle{Belief Propagation (in singly connected DAGs)}

\Red{\bf Definition:} A DAG is {\em singly connected} if its underlying
undirected graph is a tree, \emph{ie} there is only one undirected
path between any two nodes. 

\vspace*{1mm}
\centerline{
\includegraphics[scale=0.3]{singly-connected} \hspace*{2ex}
\includegraphics[scale=0.3]{multiply-connected}
}
\vspace*{1mm}

\Red{\bf Goal:} For some node $X$ we want to compute $p(X|E=e)$ given
observed variables (evidence) $e$. \\

Since we are considering singly connected graphs:
%
\begin{itemize}
\item every node $X$ divides the evidence into \Green{upstream $e^+_X$} and
\Blue{downstream $e^-_X$}
\item every edge $X\rightarrow Y$ divides the evidence into
  \Green{upstream $e^+_{XY}$} and \Blue{downstream $e^-_{XY}$}. 
\end{itemize}
\end{frame}
%%%%%%%%%%%%%%%%%%%%%%%%%%%%%%%%%%%%%%%%%%%%%%%%%%%%%%%%%%%%%%%%%%%%%%
\begin{frame}
\frametitle{Three key ideas behind Belief Propagation}

\centerline{\includegraphics[width=20mm]{bp}}

\Red{\bf Idea 1}: The probability of a variable $X$ can be found by
combining upstream and downstream evidence:  
\begin{equation*}
\begin{split}
p(X|e)\;=&\;\;\frac{p(X,e)}{p(e)}\;=\;\frac{p(X,e^+_X,e^-_X)}{p(e^+_X,e^-_X)}
\;\propto\;p(X|e^+_X)\quad\times\hspace{-2em}\underset{\Red{\mbox{$X$ d-separates $e_X^-$ from $e_X^+$}}}{\underbrace{p(e^-_X|X,e_X^+)}}\\
=&\;\;\Green{p(X|e_X^+)}\Blue{p(e^-_X|X)}\;=\;\Green{\pi(X)}\Blue{\lambda(X)}
\end{split}
\end{equation*} \vspace*{-2em}

\Red{\bf Idea 2}: The upstream and downstream evidence can be computed via a local message passing algorithm between the nodes in the graph. \\

\Red{\bf Idea 3}: ``Don't send back to a node (any part of) the message it
sent to you!''\\[2ex]

we will focus on factor graphs (simpler) instead of DAGs...

\end{frame}

%%%%%%%%%%%%%%%%%%%%%%%%%%%%%%%%%%%%%%%%%%%%%%%%%%%%%%%%%%%%%%%%%%%%%%
\begin{frame}
\frametitle{Factor Graphs}

In a factor graph, the joint probability distribution  is written as a
product of factors. Consider a vector of variables $\bfx=(x_1,\ldots,x_n)$
\[
p(\bfx)=p(x_1,\ldots,x_n) = \frac{1}{Z} \prod_{j} f_j(\bfx_{S_j})
\]
where $Z$ is the normalisation constant, $S_j$ denotes the subset of
$\{1,\ldots,n\}$ which participate in factor  $f_j$ and $\bfx_{S_j} =
\{ x_i : i \in S_j \}$.\\[2ex]

\centerline{\includegraphics[width=30mm]{factorgraphb}}

\Red{\bf variable nodes}: we draw open circles for each variable $x_i$ in the
distribution.\\[1ex]
\Red{\bf factor nodes}: we draw filled dots for each factor $f_j$ in
the distribution.
\end{frame}
%%%%%%%%%%%%%%%%%%%%%%%%%%%%%%%%%%%%%%%%%%%%%%%%%%%%%%%%%%%%%%%%%%%%%%
\begin{frame}
\frametitle{Factor graph propagation}

\centerline{Singly connected vs Multiply connected factor graphs:}

  \vspace{2em} 

\centerline{\includegraphics[width=0.4\textwidth]{factor4node}
  \hspace{2em} 
\includegraphics[width=0.24\textwidth]{factorgraphb}}

\end{frame}
%%%%%%%%%%%%%%%%%%%%%%%%%%%%%%%%%%%%%%%%%%%%%%%%%%%%%%%%%%%%%%%%%%%%%%
\begin{frame}
\frametitle{Propagation in Factor Graphs}

Let ${\mathrm n}(x)$ denote the set of factor nodes that are neighbors of
$x$.\\
Let ${\mathrm n}(f)$ denote the set of variable nodes that are
neighbors of $f$.  \\[2ex]

We can compute probabilities in a factor graph by propagating messages
from variable nodes to factor nodes and viceversa. \\[2ex]

\Red{\bf message from variable $x$ to  factor $f$}:
\[
\mu_{x\rightarrow f}(x) = \prod_{h \in {\mathrm n}(x)\setminus\{f\}}
\mu_{h\rightarrow x} (x)
\]

\Red{\bf message from  factor $f$ to variable $x$}:
\[
\mu_{f \rightarrow x}(x) = \sum_{\bfx\setminus x} \left( f(\bfx)
  \prod_{y\in {\mathrm n}(f)\setminus \{x\}} \mu_{y\rightarrow f}(y)
\right)
\]

\end{frame}
%%%%%%%%%%%%%%%%%%%%%%%%%%%%%%%%%%%%%%%%%%%%%%%%%%%%%%%%%%%%%%%%%%%%%%
\begin{frame}
\frametitle{Propagation in Factor Graphs}

\vspace*{-2ex}
${\mathrm n}(x)$ denotes the set of factor nodes that are neighbors of
$x$.\\
${\mathrm n}(f)$ denotes the set of variable nodes that are
neighbors of $f$.  \\[2ex]

\Red{\bf message from variable $x$ to  factor $f$}:
\[
\mu_{x\rightarrow f}(x) = \prod_{h \in {\mathrm n}(x)\setminus\{f\}}
\mu_{h\rightarrow x} (x)
\] \\[-2ex]

\Red{\bf message from  factor $f$ to variable $x$}:
\[
\mu_{f \rightarrow x}(x) = \sum_{\bfx\setminus x} \left( f(\bfx)
  \prod_{y\in {\mathrm n}(f)\setminus \{x\}} \mu_{y\rightarrow f}(y)
\right)
\]

If a variable has only one factor as a neighbor, it can initiate
message propagation.\\[2ex]

Once a variable has received all messages from its neighboring
factor nodes we can compute the probability of that variable by
multiplying all the messages and renormalising:
\[
p(x) \propto \prod_{h \in {\mathrm n}(x)}
\mu_{h\rightarrow x} (x)
\]
\end{frame}
%%%%%%%%%%%%%%%%%%%%%%%%%%%%%%%%%%%%%%%%%%%%%%%%%%%%%%%%%%%%%%%%%%%%%%
\begin{frame}
\frametitle{Propagation in Factor Graphs}

\centerline{\includegraphics[width=45mm]{factor4node}}

initialise all messages to be 1\\

an example schedule of messages resulting in computing $p(x_4)$:\\

\begin{tabular}{ll}
\hline message direction & message value \\ \hline
$x_1 \rightarrow f_1$ & $1(x_1)$ \\
$x_3 \rightarrow f_2$ & $1(x_3)$ \\
$f_1 \rightarrow x_2$ & $\sum_{x_1} f_1(x_1,x_2) 1(x_1)$ \\
$f_2 \rightarrow x_2$ & $\sum_{x_3} f_2(x_3,x_2) 1(x_3)$ \\
$x_2 \rightarrow f_3$ & $\left( \sum_{x_1} f_1(x_1,x_2) \right)
\left(\sum_{x_3} f_2(x_3,x_2) \right)$ \\ 
$f_3 \rightarrow x_4$ & $\sum_{x_2} f_3(x_2,x_4) \left( \sum_{x_1}
f_1(x_1,x_2) \right) \left(\sum_{x_3} f_2(x_3,x_2) \right)$ \\ \hline
\end{tabular}

\end{frame}
%%%%%%%%%%%%%%%%%%%%%%%%%%%%%%%%%%%%%%%%%%%%%%%%%%%%%%%%%%%%%%%%%%%%%%
\begin{frame}
\frametitle{Propagation in Factor Graphs}

\centerline{\includegraphics[width=45mm]{factor4node}}

initialise all messages to be 1\\

an example schedule of messages resulting in computing $p(x_4 | x_1 = a)$:\\

\begin{tabular}{ll}
\hline message direction & message value \\ \hline
$x_1 \rightarrow f_1$ & $\Green{\delta(x_1 = a)}$ \\
$x_3 \rightarrow f_2$ & $1(x_3)$ \\
$f_1 \rightarrow x_2$ & $\sum_{x_1} f_1(x_1,x_2) \Green{\delta(x_1 =
  a)} \,  = \,
\Green{f_1(x_1 = a ,x_2)} $ \\
$f_2 \rightarrow x_2$ & $\sum_{x_3} f_2(x_3,x_2) 1(x_3)$ \\
$x_2 \rightarrow f_3$ & $ \Green{f_1(x_1=a,x_2)} 
\left(\sum_{x_3} f_2(x_3,x_2) \right)$ \\ 
$f_3 \rightarrow x_4$ & $\sum_{x_2} f_3(x_2,x_4) 
\Green{f_1(x_1 = a,x_2)} \left(\sum_{x_3} f_2(x_3,x_2) \right)$ \\ \hline
\end{tabular}

\end{frame}
%%%%%%%%%%%%%%%%%%%%%%%%%%%%%%%%%%%%%%%%%%%%%%%%%%%%%%%%%%%%%%%%%%%%%%
\begin{frame}
\frametitle{Elimination Rules for Factor Graphs}

\begin{itemize}
\item \Red{\bf eliminating observed variables}

If a variable $x_i$ is {\bf observed}, i.e.\ its value is given, then
it is a {\em constant} in all factor that include $x_i$.\\

We can {\bf eliminate} $x_i$ from the graph by removing the
corresponding node and modifying all neighboring factors to treat it
as a constant.
\end{itemize}

\end{frame}
%%%%%%%%%%%%%%%%%%%%%%%%%%%%%%%%%%%%%%%%%%%%%%%%%%%%%%%%%%%%%%%%%%%%%%
\begin{frame}
\frametitle{Elimination Rules for Factor Graphs}

\vspace*{-2ex}
\begin{itemize}
\item \Red{\bf eliminating hidden variables}

If a variable $x_i$ is {\bf hidden} and we are not interested in it we
can eliminate it from the graph by summing over all its values. \\
\begin{eqnarray*}
\sum_{x_i} p(\bfx) & = & \frac{1}{Z} \sum_{x_i} \prod_{j}
f_j(\bfx_{S_j}) \\
&=& \frac{1}{Z} \prod_{j \notin {\mathrm n}(x_i)} 
f_j(\bfx_{S_j}) \Blue{
\left( \sum_{x_i} \prod_{k \in {\mathrm n}(x_i)}  f_k(\bfx_{S_k})
\right)}\\
&=& \frac{1}{Z} \prod_{j \notin {\mathrm n}(x_i)} 
f_j(\bfx_{S_j}) \;\; \; \; \Blue{f_{\mathrm{new}}(\bfx_{S_{\mathrm{new}}})}
\end{eqnarray*}
where $
f_{\mathrm{new}}(\bfx_{S_{\mathrm{new}}}) = \sum_{x_i} \prod_{k \in
  {\mathrm n}(x_i)}  f_k(\bfx_{S_k})
$
and $\displaystyle S_{\mathrm{new}} = \bigcup_{k \in {\mathrm n}(x_i)}
\!\! S_k \; \setminus \{i\}$. 

This causes all its neighboring factor nodes to merge into one new
factor node.
\end{itemize}
\end{frame}
%%%%%%%%%%%%%%%%%%%%%%%%%%%%%%%%%%%%%%%%%%%%%%%%%%%%%%%%%%%%%%%%%%%%%%
\begin{frame}
\frametitle{Inference in Hidden markov models and \\
Linear Gaussian state-space models}

\centerline{\includegraphics[width=2.2in]{ssm}}

\[
p(X_{1, \ldots, T}, Y_{1, \ldots, T}) = p(X_1) p(Y_1|X_1) \prod_{t=2}^T
\left[ p(X_t|X_{t-1}) p(Y_t|X_t) \right] 
\]
\begin{itemize}
\item In HMMs, the states $X_t$ are discrete.
\item In linear Gaussian SSMs, the states are real Gaussian vectors.
\item Both HMMs and SSMs can be represented as singly connected DAGs.
\item The \Blue{forward--backward algorithm} in hidden Markov models (HMMs),
  and the \Blue{Kalman smoothing algorithm} in SSMs are both instances
  of belief propagation / factor graph propagation. 
\end{itemize}

\end{frame}
%%%%%%%%%%%%%%%%%%%%%%%%%%%%%%%%%%%%%%%%%%%%%%%%%%%%%%%%%%%%%%%%%%%%%%
\begin{frame}
\frametitle{Inference in multiply connected DAGs}

\Blue{\bf The Junction Tree algorithm:} Form an undirected graph from your
directed graph such that no additional conditional independence
relationships have been created (this step is called
``moralization'').  Lump variables in cliques together
and form a tree of cliques---this may require a nasty step called
``triangulation''.  Do inference in this tree of cliques.\\ 

\Blue{\bf Cutset Conditioning:} or ``reasoning by assumptions''. Find a
small set of variables which, if they were given (i.e.\ known) would
render the remaining graph singly connected. For each value of these variables
run belief propagation on the singly connected network. Average the
resulting beliefs with the appropriate weights (given by normalizing
constants).\\ 

\Blue{\bf Loopy Belief Propagation:} just use BP although there are
loops. In this case the terms ``upstream'' and ``downstream'' are not
clearly defined. No guarantee of convergence, except for certain
special graphs, but often works well in practice (c.f.\
``turbo-decoding'' for error-correcting codes).

\end{frame}
%%%%%%%%%%%%%%%%%%%%%%%%%%%%%%%%%%%%%%%%%%%%%%%%%%%%%%%%%%%%%%%%%%%%%%
\begin{frame}
\frametitle{Summary}

\begin{itemize}
\item inference consists of the problem of computing \Blue{$p(\mbox{variables
    of interest}|\mbox{observed variables})$}
\item for singly connected DAGs, \Blue{belief propagation} solves this
  problem exactly. 
\item for factor graphs, the analogous algorithm is \Blue{factor graph
  propagation}.
\item well-known algorithms such as Kalman smoothing and
  forward-backward are special cases these general propagation
  algorithms.
\item for multiply connected graphs, the \Blue{junction tree algorithm}
  solves the exact inference problem, but can be {\em very} slow
  (exponential in the cardinality of the largest clique). 
\item one approximate inference algorithm is \Blue{``loopy belief
  propagation''}---we will see other approximate inference algorithms
  in a later lecture.
\end{itemize}

\end{frame}
%%%%%%%%%%%%%%%%%%%%%%%%%%%%%%%%%%%%%%%%%%%%%%%%%%%%%%%%%%%%%%%%%%%%%%
\begin{frame}
\frametitle{Appendix: Belief Propagation in Directed Graphs}

\vspace*{-2ex}
\centerline{
\begin{tabular}{cc}
\includegraphics[width=30mm]{bp}&\qquad \parbox[b]{45mm}{top-down
  upstream evidence:\\(message $U_i$ sends to $X$)\\[0.2cm]
\Green{$\pi_X(U_i) = p(U_i|e^+_{U_iX})$}\\[0.8cm]bottom-up downstream
evidence:\\(message $Y_j$ sends to $X$)\\[0.3cm]
\Blue{$\lambda_{Y_j}(X) = p(e^-_{XY_j}|X)$}{\vskip 0.1cm}}
\end{tabular}
}
To update the probability of $X$ given the observed data:
%
\begin{equation*}
\begin{split}
\operatorname{BEL}(X) = p(X|e) \;=&\;\;\frac{1}{Z}\lambda(X) \, \pi(X)\\
\lambda(X)\;=&\;\;\prod_j\lambda_{Y_j}(X)\\
\pi(X)\;=&\;\;\sum_{U_1\cdots U_n}p(X|U_1,\ldots,U_n)\prod_i\pi_X(U_i)
\end{split}
\end{equation*}
\end{frame}
%%%%%%%%%%%%%%%%%%%%%%%%%%%%%%%%%%%%%%%%%%%%%%%%%%%%%%%%%%%%%%%%%%%%%%
\begin{frame}
\frametitle{Belief Propagation (cont.)}

\vspace*{-2ex}
\centerline{
\begin{tabular}{cc}
\includegraphics[width=30mm]{bp}&\qquad \parbox[b]{45mm}{top-down
  upstream evidence:\\(message $U_i$ sends to $X$)\\[0.2cm]
\Green{$\pi_X(U_i) = p(U_i|e^+_{U_iX})$}\\[0.8cm]bottom-up downstream
evidence:\\(message $Y_j$ sends to $X$)\\[0.3cm]
\Blue{$\lambda_{Y_j}(X) = p(e^-_{XY_j}|X)$}{\vskip 0.1cm}}
\end{tabular}
}
Bottom-up propagation, message $X$ sends to $U_i$:
%
\begin{equation*}
\lambda_X(U_i)\;=\;\sum_X\lambda(X)\sum_{U_k:k\neq i}p(X|U_1,\ldots,U_n)\prod_{k\neq i}\pi_X(U_k)
\end{equation*}
%
Top-down propagation, message $X$ sends to $Y_j$:
%
\vspace*{-1.5ex}
\begin{equation*}
\pi_{Y_j}(X)\;=\;\frac{1}{Z}\big[\prod_{k\neq j}\lambda_{Y_k}(X)\big]\sum_{U_1\cdots U_n}p(X|U_1,\ldots,U_n)\prod_i\pi_X(U_i)\;=\;
\frac{1}{Z}\frac{\operatorname{BEL}(X)}{\lambda_{Y_j}(X)}
\end{equation*}
$Z$ is the normaliser ensuring $\sum_X \pi_{Y_j}(X) =1$ \hfill \PineGreen{Demo?}

% Fluffy and Moby Demo

\end{frame}
%%%%%%%%%%%%%%%%%%%%%%%%%%%%%%%%%%%%%%%%%%%%%%%%%%%%%%%%%%%%%%%%%%%%%%
\begin{frame}
\frametitle{Appendix: Understanding BP equations}

\centerline{\includegraphics[width=22mm]{bp}}

\begin{eqnarray}
p(X|e)  = \operatorname{BEL}(X)\;=&\;\;\frac{1}{Z}\lambda(X)\pi(X) =
p(e^-_X|X) p(X|e_X^+) \\
p(e^-_X|X) = \lambda(X)\;=&\;\;\prod_j\lambda_{Y_j}(X) =
\prod_j p(e^-_{XY_j}|X) \\
p(X|e^+_X)  = \pi(X)\;=&\;\;\sum_{U_1\cdots
  U_n}p(X|U_1,\ldots,U_n)\prod_i\pi_X(U_i)  \\
=& \;\;\sum_{U_1\cdots
  U_n}p(X|U_1,\ldots,U_n)\prod_i p(U_i|e^+_{U_iX}) 
\end{eqnarray}
$Z$ is a normalization constant.\\
All equations follow from the conditional independencies in the
graph. 

\end{frame}
%%%%%%%%%%%%%%%%%%%%%%%%%%%%%%%%%%%%%%%%%%%%%%%%%%%%%%%%%%%%%%%%%%%%%%
\begin{frame}
\frametitle{Appendix: The Junction Tree Algorithm}

\end{frame}
%%%%%%%%%%%%%%%%%%%%%%%%%%%%%%%%%%%%%%%%%%%%%%%%%%%%%%%%%%%%%%%%%%%%%%
\begin{frame}
\frametitle{The Junction Tree Algorithm 1}

\centerline{\includegraphics[scale=0.45]{jt1}}

starting with a DAG...

\end{frame}
%%%%%%%%%%%%%%%%%%%%%%%%%%%%%%%%%%%%%%%%%%%%%%%%%%%%%%%%%%%%%%%%%%%%%%
\begin{frame}
\frametitle{The Junction Tree Algorithm 2}

\centerline{\includegraphics[scale=0.45]{jt2}}

\Blue{\bf moralize} by marrying the parents of each node\\
remove edge directions\\
this results in an undirected graph with no additional C.I.\ relations
\end{frame}
%%%%%%%%%%%%%%%%%%%%%%%%%%%%%%%%%%%%%%%%%%%%%%%%%%%%%%%%%%%%%%%%%%%%%%
\begin{frame}
\frametitle{The Junction Tree Algorithm 3}

\centerline{\includegraphics[scale=0.45]{jt3}}

\Blue{\bf triangulate} so that there is no loop of length $>$ 3 without a
chord\\
this is necessary so that the final junction tree satisfies the
running intersection property
\end{frame}
%%%%%%%%%%%%%%%%%%%%%%%%%%%%%%%%%%%%%%%%%%%%%%%%%%%%%%%%%%%%%%%%%%%%%%
\begin{frame}
\frametitle{The Junction Tree Algorithm 4}

\vspace*{-3ex}
\centerline{\includegraphics[scale=0.35]{jt4}}

find cliques of the moralized, triangulated graph
\end{frame}
%%%%%%%%%%%%%%%%%%%%%%%%%%%%%%%%%%%%%%%%%%%%%%%%%%%%%%%%%%%%%%%%%%%%%%
\begin{frame}
\frametitle{The Junction Tree Algorithm 5}

\centerline{\includegraphics[scale=0.25]{jt5}}

\begin{itemize}
\item form \Blue{\bf junction tree}: tree of (overlapping) sets of variables
\item the \Blue{\bf running intersection property} means that if a variable
  appears in more than one clique (e.g.\ $C$), it appears in all
  intermediate cliques in the tree.
\item the junction tree propagation algorithm ensures that neighboring
cliques have consistent probability distribution
\item local consistency $\rightarrow$ global consistency  
\end{itemize}

\end{frame}
%%%%%%%%%%%%%%%%%%%%%%%%%%%%%%%%%%%%%%%%%%%%%%%%%%%%%%%%%%%%%%%%%%%%%%
\begin{frame}
\frametitle{Appendix: Fluffy and Moby: A Belief Propagation Demo}


\end{frame}
%%%%%%%%%%%%%%%%%%%%%%%%%%%%%%%%%%%%%%%%%%%%%%%%%%%%%%%%%%%%%%%%%%%%%%
\begin{frame}
\frametitle{1. Model Structure}

\phantom{.} \hfill 
\includegraphics[scale=0.450]{FluffyandMoby1}


\end{frame}
%%%%%%%%%%%%%%%%%%%%%%%%%%%%%%%%%%%%%%%%%%%%%%%%%%%%%%%%%%%%%%%%%%%%%%
\begin{frame}
\frametitle{2. Model Parameters}

\phantom{.} \hfill 
\includegraphics[scale=0.450]{FluffyandMoby2}

\end{frame}
%%%%%%%%%%%%%%%%%%%%%%%%%%%%%%%%%%%%%%%%%%%%%%%%%%%%%%%%%%%%%%%%%%%%%%
\begin{frame}
\frametitle{3. Propagating Evidence}

\phantom{.} \hfill 
\includegraphics[scale=0.450]{FluffyandMoby3} 


\vspace*{-5ex}
\begin{enumerate}
\item Observe \Red{``Moby is dead''}, i.e. $x_3 = 1$
\vspace*{-2ex}
\item Send $\lambda_{x_3}(x_1) \equiv p(e^-_{x_1 \rightarrow x_3}|x_1) = \vectr{0.1}{1.0}$  message $x_3
  \rightarrow x_1$
\vspace*{-2ex}
\item $BEL(x_1|x_3=1) = \frac{1}{Z} \vectr{0.99}{0.01} \odot
  \vectr{0.1}{1.0} = \vectr{0.91}{0.09}$ 
\end{enumerate}

\end{frame}
%%%%%%%%%%%%%%%%%%%%%%%%%%%%%%%%%%%%%%%%%%%%%%%%%%%%%%%%%%%%%%%%%%%%%%
\begin{frame}
\frametitle{4. Propagating Evidence}

\phantom{.} \hfill \includegraphics[scale=0.450]{FluffyandMoby4}

\vspace*{-8ex}
\begin{enumerate}
\addtocounter{enumi}{3}
\item Send $\pi_{x_4}(x_1) \equiv p(x_1|e^+_{x_1 \rightarrow x_4}) =
  \vectr{0.91}{0.09}$ 
\vspace*{-2ex}
\item Send $\pi_{x_4}(x_2) \equiv p(x_2|e^+_{x_2 \rightarrow x_4}) =
  p(x_2) = \vectr{0.99}{0.01}$ from $x_2 \rightarrow x_4$.
\vspace*{-2ex}
\item Compute $\pi(x_4) \equiv p(x_4|e^+_{x_4}) = \sum_{x_1,x_2} p(x_4|x_1,x_2)
  \pi_{x_4}(x_1)   \pi_{x_4}(x_2) = \vectr{0.18}{0.82}$
\vspace*{-2ex}
\item $BEL(x_4|x_3=1) = \vectr{0.18}{0.82}$, whereas before observing
  $x_3=1$, $BEL(x_4) = \vectr{0.1}{0.9}$.
\end{enumerate}

\end{frame}
%%%%%%%%%%%%%%%%%%%%%%%%%%%%%%%%%%%%%%%%%%%%%%%%%%%%%%%%%%%%%%%%%%%%%%
\begin{frame}
\frametitle{5. Propagating Evidence}

\phantom{.} \hfill \includegraphics[scale=0.45]{FluffyandMoby5}

\vspace*{-12ex}
\begin{enumerate}
\addtocounter{enumi}{7}
\item Observe \Red{``Fluffy's Food Bowl is Full'' $x_6=1$ !}
\vspace*{-2ex}
\item  Send $\lambda_{x_6}(x_4) = \vectr{0.9}{0.1}$  message $x_6
  \rightarrow x_4$
\vspace*{-2ex}
\item $BEL(x_4|x_3=1,x_6=1) = \frac{1}{Z} \vectr{0.18}{0.82} \odot
  \vectr{0.9}{0.1} = \vectr{0.66}{0.34}$
\vspace*{-2ex}
\item  Send $\lambda_{x_4}(x_1) = \sum_{x_4}   \lambda_{x_6}(x_4) \sum_{x_2}
  p(x_4|x_1,x_2) \pi_{x_4}(x_2) = \vectr{0.19}{0.82}$
\vspace*{-2ex}
\item $BEL(x_1|x_3=1,x_6=1)=\frac{1}{Z} \vectr{0.99}{0.01} \!\odot\!
  \vectr{0.1}{1.0} \!\odot\! \vectr{0.19}{0.82} \!=\! \vectr{0.70}{0.30} \Rightarrow$
   \Red{Fluffy still innocent!}
\end{enumerate}

\end{frame}
\end{document}