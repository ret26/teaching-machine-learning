\subtitle{4F13: Machine Learning}
\author[Ghahramani \& Rasmussen]{Zoubin Ghahramani and Carl Edward Rasmussen}
\institute[CUED]{Department of Engineering, University of Cambridge}

\parskip=1.5ex

\newcommand{\maketilde}{\raisebox{0.4ex}{\tiny $\sim$}}
\newcommand{\bfa}{\mathbf a}
\newcommand{\bfb}{\mathbf b}
\newcommand{\bff}{\mathbf f}
\newcommand{\bfm}{\mathbf m}
\newcommand{\bfp}{\mathbf{p}}
\newcommand{\bfs}{\mathbf s}
\newcommand{\bfu}{\mathbf u}
\newcommand{\bfx}{\mathbf x}
\newcommand{\bfy}{\mathbf y}
\newcommand{\bfv}{\mathbf v}

\newcommand{\bmu}{{\boldsymbol{\mu}}}
\newcommand{\btheta}{\boldsymbol{\theta}}
\newcommand{\balpha}{\boldsymbol{\alpha}}
\newcommand{\bphi}{\boldsymbol{\phi}}
\newcommand{\bpi}{\boldsymbol{\pi}}
\newcommand{\blambda}{\boldsymbol{\lambda}}

\newcommand{\N}{\mathcal{N}}
\newcommand{\R}{\mathbb{R}}
\newcommand{\T}{{\scriptsize^{\top}}}
\newcommand{\D}{\mathcal{D}}
\newcommand{\F}{\mathcal{F}}
\newcommand{\E}{\mathbb{E}}
\newcommand{\V}{\mathbb{V}}
\newcommand{\M}{\mathcal{M}}
\newcommand{\KL}{\mathcal{KL}}
\newcommand{\cut}[1]{}
\newcommand{\trace}{\operatorname{trace}}

\newcommand{\argmax}{\operatorname{argmax}}
\newcommand{\argmin}{\operatorname{argmin}}
\newcommand{\ci}{{\bot\negthickspace\negthickspace\bot}} % conditional indep.
\newcommand{\neigh}{\operatorname{ne}}
\newcommand{\vectr}[2]{  \left[ \!\!\begin{array}{c} #1 \\
      #2 \end{array} \!\!\right]}
\newcommand{\deff}{\stackrel{\mathrm{def}}{=}}
\newcommand{\deldel}[2]{\frac{\partial #1}{\partial #2}}


\definecolor{mypine}{rgb}{0.05,0.45,0.05}
\definecolor{mycyan}{rgb}{0.0,0.4,0.4}
\newcommand{\Red}{\textcolor{red}}
\newcommand{\Blue}{\textcolor{blue}}
\newcommand{\Green}{\textcolor{mypine}}
\newcommand{\PineGreen}{\textcolor{mypine}}
\newcommand{\Magenta}{\textcolor{magenta}}


\newcommand{\foilhead}[1]{\end{frame} 
\frametitle{#1}
 \begin{frame}}
\newcommand{\bea}{\begin{eqnarray*}}
\newcommand{\eea}{\end{eqnarray*}}

\begin{document}

\titleslide{Lecture 6: Graphical Models: Learning}{February 6th, 2008}

%%%%%%%%%%%%%%%%%%%%%%%%%%%%%%%%%%%%%%%%%%%%%%%%%%%%%%%%%%%%%%%%%%%%%%
\begin{frame}
\frametitle{Learning parameters}

\vspace*{-3ex}
\hspace{3ex} \includegraphics[width=0.7in]{dag07} \hspace{6ex} $P(x_1) P(x_2|x_1)
P(x_3|x_1) P(x_4|x_2)$ \hspace{6ex} \includegraphics[width=1.1in]{cpt1} \\[1ex]

Assume each variable $x_i$ is discrete and can take on $K_i$ values. \\[1ex]

The parameters of this model can be represented as 4 tables:
$\theta_1$ has $K_1$ entries, $\theta_2$ has $K_1 \times K_2$
entries, etc. \\[1ex]

These are called {\bf conditional probability tables} (CPTs) with the
following semantics:
\[
P(x_1 = k) = \theta_{1,k} \hspace{4ex} P(x_2 = k' | x_1 = k) = \theta_{2,k,k'} 
\]
If node $i$ has $M$ parents, $\theta_i$ can be represented either as
an $M+1$ dimensional table, or as a 2-dimensional table with $\left(\prod_{j
  \in \mathrm{pa}(i)} K_j \right) \times K_i $ entries by collapsing all
the states of the parents of node $i$. Note that $\sum_{k'}
\theta_{i,k,k'} = 1$.\\[2ex] 

Assume a data set $\D = \{ \bfx^{(n)} \}_{n=1}^N$. \hfill  \Red{\bf How
  do we learn $\btheta$ from $\D$?}


\end{frame}
%%%%%%%%%%%%%%%%%%%%%%%%%%%%%%%%%%%%%%%%%%%%%%%%%%%%%%%%%%%%%%%%%%%%%%
\begin{frame}
\frametitle{Learning parameters}

Assume a data set $\D = \{ \bfx^{(n)} \}_{n=1}^N$. How do we learn
$\btheta$ from $\D$? 

\vspace*{-7.5ex}

\hspace{3ex} $P(\bfx|\btheta) = P(x_1|\theta_1) P(x_2|x_1,\theta_2)
P(x_3|x_1,\theta_3) P(x_4|x_2,\theta_4) $ \hspace{8ex} \includegraphics[width=0.6in]{dag07}\\[1ex]


Likelihood:
\vspace*{-2ex}
\[
P(\D|\btheta) = \prod_{n=1}^N P(\bfx^{(n)}|\btheta) 
\] \\[-2ex]
Log Likelihood:
\vspace*{-2ex}
\[
\log P(\D|\btheta) = \sum_{n=1}^N \sum_i \log P(x_i^{(n)}|x_{\mathrm{pa}(i)}^{(n)}, \theta_i)
\]
This decomposes into sum of functions of $\theta_i$. Each $\theta_i$ can be
optimized separately:
\vspace*{-2ex}
\[
\hat{\theta}_{i,k,k'} = \frac{n_{i,k,k'}}{\sum_{k''} n_{i,k,k''}}
\] 
\vspace*{-.5ex}
where $n_{i,k,k'}$ is the number of times in $\D$ where $x_i=k'$ and
$x_{\mathrm{pa}(i)} = k$. \\[1ex]

ML  solution: \Red{Simply calculate frequencies!}
\hfill \includegraphics[width=1.5in]{cpt2} 
\end{frame}
%%%%%%%%%%%%%%%%%%%%%%%%%%%%%%%%%%%%%%%%%%%%%%%%%%%%%%%%%%%%%%%%%%%%%%
\begin{frame}
\frametitle{Maximum Likelihood Learning with Hidden Variables: \\ The EM
  Algorithm} 

\centerline{
\begin{tabular}{cc}
\includegraphics[width=1in]{x1y1} & 
\raisebox{0.4in}{
\parbox{2.7in}{
Assume a model parameterised by $\theta$ with observable variables $Y$
and hidden variables $X$}}
\end{tabular}
}

\vspace{2ex}

{\bf Goal:} maximize parameter log likelihood given observed data. 
\bea
{\cal L}(\theta) = \log p(Y|\theta) = \log \sum_X p(Y,X|\theta)
\eea
\end{frame}
%%%%%%%%%%%%%%%%%%%%%%%%%%%%%%%%%%%%%%%%%%%%%%%%%%%%%%%%%%%%%%%%%%%%%%
\begin{frame}
\frametitle{Maximum Likelihood Learning with Hidden Variables: \\ The EM
  Algorithm} 

\vspace*{-2ex}
{\bf Goal:} maximise parameter log likelihood given observables. 
\bea
{\cal L}(\theta) = \log p(Y|\theta) = \log \sum_X p(Y,X|\theta)
\eea

\Blue{The EM algorithm (intuition):} \\[2ex]

Iterate between applying the following two steps:

\begin{itemize}
\item {\bf The E step:} fill-in the hidden/missing variables
\item {\bf The M step:} apply complete data learning to filled-in data.
\end{itemize}

\end{frame}
%%%%%%%%%%%%%%%%%%%%%%%%%%%%%%%%%%%%%%%%%%%%%%%%%%%%%%%%%%%%%%%%%%%%%%
\begin{frame}
\frametitle{Maximum Likelihood Learning with Hidden Variables: \\ The EM
  Algorithm} 

\vspace*{-1ex}
{\bf Goal:} maximise parameter log likelihood given observables. 
\bea
{\cal L}(\theta) = \log p(Y|\theta) = \log \sum_X p(Y,X|\theta)
\eea

\vspace*{-2.5ex}
\Blue{The EM algorithm (derivation):} 
\bea 
{\cal L}(\theta) &=& \log \sum_X q(X) \frac{p(Y,X|\theta)}{q(X)} \ge \sum_X q(X) \log \frac{p(Y,X|\theta)}{q(X)} = \F(q(X),\theta)
\eea

\vspace*{-1ex}
\begin{itemize}
\item {\bf The E step:} maximize $\F(q(X),\theta^{[t]})$ wrt $q(X)$ holding
  $\theta^{[t]}$ fixed: \vspace*{-1ex}
\[
\Blue{q(X) = P(X|Y,\theta^{[t]})}
\]
\vspace*{-2.5ex}
\item {\bf The M step:} maximize $\F(q(X),\theta)$ wrt $\theta$ holding
  $q(X)$ fixed:
\[
\Blue{\theta^{[t+1]} \leftarrow \argmax_{\theta} \sum_X q(X) \log
p(Y,X|\theta) }
\]
\end{itemize}

The E-step requires solving the \Red{{\em inference}} problem, finding
the distribution over the hidden variables $p(X|Y,\theta^{[t]})$ given the
current model parameters. This can be done using \Red{\bf belief
  propagation} or the \Red{\bf junction tree algorithm}. 

\end{frame}
%%%%%%%%%%%%%%%%%%%%%%%%%%%%%%%%%%%%%%%%%%%%%%%%%%%%%%%%%%%%%%%%%%%%%%
\begin{frame}
\frametitle{Maximum Likelihood Learning with Hidden Variables: \\ The EM
  Algorithm} 

\PineGreen{\bf ML Learning with Complete Data (No Hidden Variables)}

Log likelihood decomposes into sum of functions of $\theta_i$. Each
$\theta_i$ can be optimized separately:
\PineGreen{
\[
\hat{\theta}_{ijk} \leftarrow \frac{n_{ijk}}{\sum_{k'} n_{ijk'}} 
\]}
where $n_{ijk}$ is the number of times in $\D$ where $x_i=k$ and
$x_{\mathrm{pa}(i)} = j$. \\

Maximum likelihood solution: \Red{Simply calculate frequencies!} \\[1ex]

\PineGreen{\bf ML Learning with Incomplete Data (i.e.\ with Hidden Variables)}

Iterative EM algorithm
\begin{itemize}

\item[] \Blue{\bf E step:} compute expected counts given previous settings of
parameters  $E[n_{ijk} | \D, \btheta^{[t]}]$.

% \vspace*{-1ex}
\item[] \Blue{\bf M step:} re-estimate parameters using these expected counts
\PineGreen{
\[
\theta^{[t+1]}_{ijk} \leftarrow
\frac{E[n_{ijk}|\D,\btheta^{[t]}]}{\sum_{k'} E[n_{ijk'}|\D,\btheta^{[t]}]}  
\]}
\end{itemize}

\end{frame}
%%%%%%%%%%%%%%%%%%%%%%%%%%%%%%%%%%%%%%%%%%%%%%%%%%%%%%%%%%%%%%%%%%%%%%
\begin{frame}
\frametitle{Bayesian parameter learning with {\bf no} hidden variables}

Let $n_{ijk}$ be the number of times ($x^{(n)}_i = k$ and
$x^{(n)}_{\mathrm{pa}(i)} = j$) in $\D$.

For each $i$ and $j$, $\btheta_{ij\cdot}$ is a probability vector of
length $K_i \times 1$. \\

Since $x_i$ is a discrete variable with probabilities given by
$\btheta_{i,j,\cdot}$, the likelihood is: 
\[
P(\D|\btheta) = \prod_n \prod_i P(x_i^{(n)}|x_{\mathrm{pa}(i)}^{(n)}, \btheta) =
\prod_i \prod_j \prod_k \theta_{ijk}^{n_{ijk}}
\] \\[-1ex]
If we choose a prior on $\btheta$ of the form: \vspace*{-1ex}
\[
P(\btheta) = c \prod_i \prod_j \prod_k
\theta_{ijk}^{\alpha_{ijk}-1}
\]
where $c$ is a normalization constant, and $\sum_k \theta_{ijk}=1$
$\forall i, j $, then the posterior distribution also has the same
form:
\[
P(\btheta|\D) =  c' \prod_i \prod_j \prod_k
\theta_{ijk}^{\tilde{\alpha}_{ijk}-1}
\] \\[-2ex]
where \Blue{$\tilde{\alpha}_{ijk}= \alpha_{ijk}+ n_{ijk}$}. \\[1ex]

This distribution is called the \Red{Dirichlet distribution}.

\end{frame}
%%%%%%%%%%%%%%%%%%%%%%%%%%%%%%%%%%%%%%%%%%%%%%%%%%%%%%%%%%%%%%%%%%
\begin{frame}
\frametitle{Dirichlet Distribution}

The \Red{Dirichlet distribution} is a distribution over the
$K$-dim probability simplex.\\

Let $\btheta$ be a $K$-dimensional vector s.t.\ $\forall j :\,
\theta_j \ge 0$ and 
$\sum_{j=1}^K \theta_j  = 1$
\[
P(\btheta|\balpha) = \mbox{Dir}(\alpha_1, \ldots, \alpha_K) \deff
\PineGreen{\frac{\Gamma(\sum_j \alpha_j)}{\prod_j \Gamma(\alpha_j)}}
\prod_{j=1}^K \theta_j^{\alpha_j - 1} 
\]
where the \PineGreen{first term} is a normalization
constant\footnote{$\Gamma(x) = (x-1) \Gamma(x-1) = \int_0^\infty
t^{x-1} e^{-t} dt$. For integer $n$, $\Gamma(n)=(n-1)!$} and $E(\theta_j) =
\alpha_j / (\sum_k \alpha_k)$\\

The Dirichlet is \Blue{conjugate to the multinomial distribution}. Let 
\vspace*{-1ex}
\[
x| \btheta \sim \mbox{Multinomial}(\cdot|\btheta)
\]
That is,  $P(x=j|\btheta) = \theta_j$. Then the posterior is also
Dirichlet:\vspace*{-1ex} 
\[
P(\btheta|x=j,\balpha) = \frac{P(x=j|\btheta)P(\btheta|\balpha)}{P(x=j|\balpha)} = \mbox{Dir}(\tilde{\balpha})
\]
where \Blue{$\tilde{\alpha}_j = \alpha_j + 1$}, and $\forall \ell \neq j : \,
\tilde{\alpha}_\ell = \alpha_\ell$

\end{frame}
%%%%%%%%%%%%%%%%%%%%%%%%%%%%%%%%%%%%%%%%%%%%%%%%%%%%%%%%%%%%%%%%%%
\begin{frame}\frametitle{Dirichlet Distributions}

\vspace*{-3ex}
Examples of Dirichlet distributions over $\btheta=(\theta_1, \theta_2,
\theta_3)$ which 
can be plotted in 2D since $\theta_3=1-\theta_1-\theta_2$:

\vspace{2ex}

\centerline{
\includegraphics[width=1.1in]{dir111}
\includegraphics[width=1.1in]{dir222}
\includegraphics[width=1.1in]{dir000}
}

\vspace{1ex}

\centerline{
\includegraphics[width=1.1in]{dir202}
\includegraphics[width=1.1in]{dir220}
\includegraphics[width=1.1in]{dir999}
}

\end{frame}
%%%%%%%%%%%%%%%%%%%%%%%%%%%%%%%%%%%%%%%%%%%%%%%%%%%%%%%%%%%%%%%%%%%%%%
\begin{frame}
\frametitle{Example}

\vspace*{-3ex}
\parbox{3.8in}{
Assume $\alpha_{ijk}=1$ $\forall i,j,k$. \\[2ex]
This corresponds to a \Red{\bf uniform} prior distribution over parameters
$\theta$. This is not a very strong/dogmatic prior, since any
parameter setting is assumed a priori possible.} 
 \hfill \includegraphics[width=0.8in]{dir111}

\vspace*{2ex}
After observed data $\D$, what are the parameter posterior
distributions? 
\[
P(\theta_{ij\cdot}|\D) = \mathrm{Dir}(n_{ij\cdot}+1)
\] \\

This distribution predicts, for future data:
\[
P(x_i=k|x_{\mathrm{pa}(i)}=j,\D) = \frac{n_{ijk}+1}{\sum_{k'} (n_{ijk'} +
  1)}
\]

Adding 1 to each of the counts is a form of smoothing called \Blue{``Laplace's Rule''.}

\end{frame}
%%%%%%%%%%%%%%%%%%%%%%%%%%%%%%%%%%%%%%%%%%%%%%%%%%%%%%%%%%%%%%%%%%%%%%
\begin{frame}
\frametitle{Bayesian parameter learning with hidden variables}

\Blue{\bf Notation:} let $\D$ be the observed data set, $X$ be hidden variables,
and $\btheta$ be model parameters. Assume discrete variables and
Dirichlet priors on $\btheta$\\[1ex]

\Blue{\bf Goal:} to infer $P(\btheta|\D) = \sum_X P(X,\btheta|\D)$\\[1ex]

\Blue{\bf Problem:} since (a) 
\[
P(\btheta|\D) = \sum_X P(\btheta|X,\D) P(X|\D),
\]
and (b) for every way of filling in the missing data, $P(\btheta|X,\D)$ is a
Dirichlet distribution, and (c) there are exponentially many ways of
filling in $X$, it follows that \Red{$P(\btheta|\D)$ is a mixture of
Dirichlets with exponentially many terms!}\\[2ex]

\Blue{\bf Solutions:}

\begin{itemize}
\item Find a single best (``Viterbi'') completion of $X$ (Stolcke and
Omohundro, 1993)
\item Markov chain Monte Carlo methods 
\item Variational Bayesian methods (Beal and Ghahramani, 2003)
\end{itemize}

\end{frame}
%%%%%%%%%%%%%%%%%%%%%%%%%%%%%%%%%%%%%%%%%%%%%%%%%%%%%%%%%%%%%%%%%%%%%%
\begin{frame}
\frametitle{Summary of parameter learning}

\vspace*{1ex}

\centerline{
\begin{tabular}{|l|ll|} \hline
\phantom{.} & Complete (fully observed) data & Incomplete (hidden
/missing) data \\  \hline
ML & calculate frequencies  &  EM \\
Bayesian & update Dirichlet distributions & MCMC / Viterbi  / VBEM  \\  \hline
\end{tabular}
}

\begin{itemize}
\item For complete data, Bayesian learning is not more costly than ML
\item For incomplete data, VBEM $\approx$ EM time complexity
\item Other parameter priors are possible but Dirichlet is 
  flexible and intuitive.
\item For binary data, other parametrizations include:
\begin{itemize}
\item Sigmoid:
\vspace*{-1ex}
\[
P(x_i=1|x_{\mbox{pa}(i)},\theta_i) = 1  / (1 + \exp \{  -\theta_{i0} - \sum_{j\in
  \mbox{pa}(i)} \theta_{ij} x_j \})
\] \vspace*{-2ex}
\item Noisy-or:
\vspace*{-1ex}
\[
P(x_i=1|x_{\mbox{pa}(i)},\theta_i) = 1  -  \exp \{- \theta_{i0}  - \sum_{j\in
  \mbox{pa}(i)} \theta_{ij} x_j \} 
\] \vspace*{-2ex}
\end{itemize}  
\item For non-discrete data, similar ideas but generally harder
  inference and learning.
\end{itemize}

\end{frame}
%%%%%%%%%%%%%%%%%%%%%%%%%%%%%%%%%%%%%%%%%%%%%%%%%%%%%%%%%%%%%%%%%%%%%%
\begin{frame}
\frametitle{Structure learning}

Given a data set of observations of $(A,B,C,D,E)$ can we learn the
structure of the graphical model?

\vspace{2ex}

\centerline{
\includegraphics[width=1in]{struct4} \hspace{2ex}
\includegraphics[width=1in]{struct3}\hspace{2ex}
\includegraphics[width=1in]{struct2}\hspace{2ex}
\includegraphics[width=1in]{struct1}
}

\vspace{6ex}

Let $m$ denote the graph structure = the set of edges.

\end{frame}
%%%%%%%%%%%%%%%%%%%%%%%%%%%%%%%%%%%%%%%%%%%%%%%%%%%%%%%%%%%%%%%%%%%%%%
\begin{frame}
\frametitle{Structure learning}

\vspace{2ex}

\centerline{
\includegraphics[width=1in]{struct4} \hspace{2ex}
\includegraphics[width=1in]{struct3}\hspace{2ex}
\includegraphics[width=1in]{struct2}\hspace{2ex}
\includegraphics[width=1in]{struct1}
}

\vspace{4ex}

\begin{itemize}
\item[] \Blue{\bf Constraint-Based Learning}: Use statistical tests of
  marginal and conditional independence. Find the set of DAGs whose
  d-separation relations match the results of conditional independence tests.

\item[] \Blue{\bf Score-Based Learning}: Use a global score such as
  the BIC score or Bayesian marginal likelihood. Find the structures
  that maximize this score.
\end{itemize}


\end{frame}
%%%%%%%%%%%%%%%%%%%%%%%%%%%%%%%%%%%%%%%%%%%%%%%%%%%%%%%%%%%%%%%%%%%%%%
\begin{frame}
\frametitle{Score-based structure learning for complete data}

Consider a graphical model with structure $m$, discrete observed data
$\D$, and parameters $\theta$. Assume Dirichlet priors.\\[2ex]

The Bayesian marginal likelihood score is easy to compute:
\vspace*{-1ex}
\[
\mbox{score}(m) = \log P(\D|m) = \log \int P(\D|\theta,m) P(\theta|m) d \theta
\]
\vspace*{-2ex}
\Green{\small \[
\mbox{score}(m) = \sum_i \sum_j \left[ 
\log \Gamma(\sum_k  \alpha_{ijk}) - \sum_k \log \Gamma(\alpha_{ijk}) 
- \log \Gamma(\sum_k  \tilde{\alpha}_{ijk}) + \sum_k \log
\Gamma(\tilde{\alpha}_{ijk})  
\right]
\]}
where $\tilde{\alpha}_{ijk} = \alpha_{ijk} + n_{ijk}$. \Red{\bf Note that the
score decomposes over $i$}.\\

One can incorporate structure prior information $P(m)$ as well:
\[
\mbox{score}(m) = \log P(\D|m)  + \log P(m)
\]

\Blue{\bf Greedy search algorithm:} Start with $m$. Consider modifications
$m \rightarrow m'$ (edge deletions, additions, reversals). Accept $m'$
if $\mbox{score}(m') >\mbox{score}(m)$. Repeat.\\[2ex]

\Blue{\bf Bayesian inference of model structure}: Run MCMC on $m$.

\end{frame}
%%%%%%%%%%%%%%%%%%%%%%%%%%%%%%%%%%%%%%%%%%%%%%%%%%%%%%%%%%%%%%%%%%%%%%
\begin{frame}
\frametitle{Bayesian Structural EM for {\em in}complete data}

% \vspace*{-1ex}
Consider a graphical model with structure $m$, observed data $\D$,
hidden variables $X$ and parameters $\theta$\\[2ex]

The Bayesian score is generally intractable to compute:
\[
\mbox{score}(m) = P(\D|m) = \int \sum_X P(X,\theta,\D|m) d \theta
\]

\Blue{\bf Bayesian Structure EM} (Friedman, 1998):
\begin{enumerate}
\item compute MAP parameters $\hat{\theta}$ for current model $m$ using EM
\item find hidden variable distribution $P(X|\D,\hat{\theta})$
\item for a small set of candidate structures compute or
approximate \[
\mbox{score}(m') = \sum_X P(X|\D,\hat{\theta}) \log
P(\D,X|m')
\]
\item $ m \leftarrow m'$ with highest score
\end{enumerate}

\end{frame}
%%%%%%%%%%%%%%%%%%%%%%%%%%%%%%%%%%%%%%%%%%%%%%%%%%%%%%%%%%%%%%%%%%%%%%
\begin{frame}
\frametitle{Directed Graphical Models and Causality}

Discovering causal relationships is fundamental to science and cognition.\\[2ex]

Although the independence relations are identical,
there is a \Blue{\bf causal} difference between \\[-1ex]
\begin{itemize}
\item ``smoking'' $\rightarrow$ ``yellow teeth''
\item ``yellow teeth''  $\rightarrow$ ``smoking''
\end{itemize}

\Blue{\bf Key idea:} interventions and the do-calculus:
\[
P(S|Y=y) \neq P(S|\mbox{do}(Y=y))
\]
\[
P(Y|S=s) = P(Y|\mbox{do}(S=s))
\] 
Causal relationships are robust to interventions on the parents.\\[1ex]

The \Red{\bf key difficulty} in learning causal relationships from
observational data is the presence of \Red{\bf hidden common causes}:

\centerline{\includegraphics[width=2.3in]{hidden-causes}}

\end{frame}
%%%%%%%%%%%%%%%%%%%%%%%%%%%%%%%%%%%%%%%%%%%%%%%%%%%%%%%%%%%%%%%%%%%%%%
\begin{frame}
\frametitle{Learning parameters and structure in undirected graphs}

\vspace*{0.7ex}
\centerline{\includegraphics[width=0.8in]{ustruct1} \hspace{3ex} \includegraphics[width=0.8in]{ustruct3}}

\vspace*{1ex}
\hspace{3ex} $P(\bfx|\btheta) = \frac{1}{Z(\btheta)} \prod_j g_j(\bfx_{C_j};
\btheta_j)$  where $Z(\btheta) = \sum_\bfx \prod_j g_j(\bfx_{C_j}; \btheta_j)$.\\[1ex]

\Red{\bf Problem:} computing $Z(\btheta)$ is computationally
intractable for general (non-tree-structured) undirected models.
Therefore, maximum-likelihood learning of parameters is generally
intractable, Bayesian scoring of structures is intractable, etc.\\[2ex]

\Red{\bf Solutions:} \\[-1ex]
\begin{itemize}
\item directly approximate $Z(\btheta)$ and/or its derivatives
  (cf.\ Boltzmann machine learning; contrastive divergence;
  pseudo-likelihood) 
% \vspace*{-1ex}
\item use approx inference methods (e.g.\ loopy belief
  propagation, bounding methods, EP).
\end{itemize}

{\footnotesize (Murray \& Ghahramani, 2004; Murray et al, 2006)
  for Bayesian learning in undirected models.}

\end{frame}
%%%%%%%%%%%%%%%%%%%%%%%%%%%%%%%%%%%%%%%%%%%%%%%%%%%%%%%%%%%%%%%%%%%%%%
\begin{frame}
\frametitle{Summary}

\begin{itemize}
\item Parameter learning in directed models: 
\begin{itemize}
\item complete and incomplete data; 
\item 
ML and Bayesian methods
\end{itemize}
\item Structure learning in directed models: complete and incomplete data
\item Causality
\item Parameter and Structure learning in undirected models
\end{itemize}

\end{frame}
\end{document}
%%%%%%%%%%%%%%%%%%%%%%%%%%%%%%%%%%%%%%%%%%%%%%%%%%%%%%%%%%%%%%%%%%
%%%%%%%%%%%%%%%%%%%%%%%%%%%%%%%%%%%%%%%%%%%%%%%%%%%%%%%%%%%%%%%%%%%%%%
