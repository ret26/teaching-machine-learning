\subtitle{4F13: Machine Learning}
\author[Ghahramani \& Rasmussen]{Zoubin Ghahramani and Carl Edward Rasmussen}
\institute[CUED]{Department of Engineering, University of Cambridge}

\parskip=1.5ex

\newcommand{\maketilde}{\raisebox{0.4ex}{\tiny $\sim$}}
\newcommand{\bfa}{\mathbf a}
\newcommand{\bfb}{\mathbf b}
\newcommand{\bff}{\mathbf f}
\newcommand{\bfm}{\mathbf m}
\newcommand{\bfp}{\mathbf{p}}
\newcommand{\bfs}{\mathbf s}
\newcommand{\bfu}{\mathbf u}
\newcommand{\bfx}{\mathbf x}
\newcommand{\bfy}{\mathbf y}
\newcommand{\bfv}{\mathbf v}

\newcommand{\bmu}{{\boldsymbol{\mu}}}
\newcommand{\btheta}{\boldsymbol{\theta}}
\newcommand{\balpha}{\boldsymbol{\alpha}}
\newcommand{\bphi}{\boldsymbol{\phi}}
\newcommand{\bpi}{\boldsymbol{\pi}}
\newcommand{\blambda}{\boldsymbol{\lambda}}

\newcommand{\N}{\mathcal{N}}
\newcommand{\R}{\mathbb{R}}
\newcommand{\T}{{\scriptsize^{\top}}}
\newcommand{\D}{\mathcal{D}}
\newcommand{\F}{\mathcal{F}}
\newcommand{\E}{\mathbb{E}}
\newcommand{\V}{\mathbb{V}}
\newcommand{\M}{\mathcal{M}}
\newcommand{\KL}{\mathcal{KL}}
\newcommand{\cut}[1]{}
\newcommand{\trace}{\operatorname{trace}}

\newcommand{\argmax}{\operatorname{argmax}}
\newcommand{\argmin}{\operatorname{argmin}}
\newcommand{\ci}{{\bot\negthickspace\negthickspace\bot}} % conditional indep.
\newcommand{\neigh}{\operatorname{ne}}
\newcommand{\vectr}[2]{  \left[ \!\!\begin{array}{c} #1 \\
      #2 \end{array} \!\!\right]}
\newcommand{\deff}{\stackrel{\mathrm{def}}{=}}
\newcommand{\deldel}[2]{\frac{\partial #1}{\partial #2}}


\definecolor{mypine}{rgb}{0.05,0.45,0.05}
\definecolor{mycyan}{rgb}{0.0,0.4,0.4}
\newcommand{\Red}{\textcolor{red}}
\newcommand{\Blue}{\textcolor{blue}}
\newcommand{\Green}{\textcolor{mypine}}
\newcommand{\PineGreen}{\textcolor{mypine}}
\newcommand{\Magenta}{\textcolor{magenta}}


\newcommand{\xx}{\bfx}
\usepackage{algorithmic}
\newcommand{\qi}{q_{{\scriptscriptstyle\backslash} \!i}(\btheta)}

\begin{document}

\titleslide{Lecture 10, 11: Variational Approximations}{February 20th, 22nd, 2008}


%%%%%%%%%%%%%%%%%%%%%%%%%%%%%%%%%%%%%%%%%%%%%%%%%%%%%%%%%%%%%%%%%%%%%%
\begin{frame}
\frametitle{Motivation}

Many statistical inference problems result in \Red{intractable
computations}...\\

$\bullet\, $ 
Bayesian posterior over model parameters:
\[
P(\theta|\D) = \frac{P(\D|\theta) P(\theta)}{P(\D)}
\]

$\bullet\, $ 
Computing posterior over hidden variables (e.g.\ for E step of EM):
\[
P(H|V,\theta) = \frac{P(V|H,\theta) P(H|\theta)}{P(V|\theta)}
\]

\parbox{3.5in}{
$\bullet\, $ 
Computing marginals in a multiply-connected graphical models:
\[
P(x_i|x_j = e) = \sum_{\xx \backslash \{ x_i, x_j\}} P(\xx|x_j = e)
\]
\Green{Solutions:} Markov chain Monte Carlo, variational approximations
}
\parbox{1.1in}{\centerline{\includegraphics[width=1in]{multiply}}}

\end{frame}
%%%%%%%%%%%%%%%%%%%%%%%%%%%%%%%%%%%%%%%%%%%%%%%%%%%%%%%%%%%%%%%%%%%%%%
\begin{frame}
\frametitle{Example: Binary latent factor model}
\begin{tabular}{cc}
\includegraphics[width=1.6in]{multicause} & \raisebox{0.45in}{\parbox{2.9in}{
Model with $K$ binary latent variables, $s_i\in\{0,1\}$, \\[1ex]
organised into a vector $\bfs=(s_1, \ldots, s_K)$ \\
real-valued observation vector $\bfy$ \\
parameters $\btheta=\{ \{ \bmu_i, \pi_i\}_{i=1}^K, \sigma^2\}$ \\[2ex]
$\bfs \sim$ Bernoulli \\ 
$\bfy | \bfs  \sim$ Gaussian  
}} 
\end{tabular}
\vspace*{-2.5ex}
\begin{eqnarray*}
p(\bfs|\bpi) = p(s_1, \ldots, s_K|\bpi) &=& \prod_{i=1}^K p(s_i|\pi_i) = \prod_{i=1}^K
\pi_i^{s_i} (1-\pi_i)^{(1-s_i)}  \\
p(\bfy|s_1, \ldots, s_K,\bmu,\sigma^2) &=& {\cal N}\left(\sum_{i=1}^K s_i \bmu_i,
\sigma^2 I\right)  \\[-2ex]
\end{eqnarray*}
EM optimizes bound on likelihood: \hfill $ \displaystyle
{\cal F}(q,\btheta) = \langle \log p(\bfs, \bfy|\btheta)\rangle_{q(\bfs)} -
\langle \log q(\bfs) \rangle_{q(\bfs)} 
$ \\
where $\langle \rangle_q$ is expectation under $q$:  \hspace{2em} $\langle
f(\bfs)\rangle_q \deff \sum_{\bfs} f(\bfs) q(\bfs)$ \\[1.5ex]

\Blue{\bf Exact E step:} $q(\bfs) = p(\bfs|\bfy,\btheta)$ distribution
over $2^K$ states: \Red{\bf intractable} for large $K$ 

\end{frame}
%%%%%%%%%%%%%%%%%%%%%%%%%%%%%%%%%%%%%%%%%%%%%%%%%%%%%%%%%%%%%%%%%%%%%%
\begin{frame}
\frametitle{Example: Binary latent factor model}
\begin{tabular}{cc}
\includegraphics[width=1.6in]{multicause} & \raisebox{0.45in}{\parbox{2.9in}{
Model with $K$ binary latent variables, $s_i\in\{0,1\}$, \\[1ex]
organised into a vector $\bfs=(s_1, \ldots, s_K)$ \\
real-valued observation vector $\bfy$ \\
parameters $\btheta=\{ \{ \bmu_i, \pi_i\}_{i=1}^K, \sigma^2\}$ \\[2ex]
$\bfs \sim$ Bernoulli \\ 
$\bfy | \bfs  \sim$ Gaussian 
}} 
\end{tabular}

\centerline{\includegraphics[width=4.1in]{lu-results}}

\centerline{\small from Lu et al (2004)}
\end{frame}
%%%%%%%%%%%%%%%%%%%%%%%%%%%%%%%%%%%%%%%%%%%%%%%%%%%%%%%%%%%%%%%%%%%%%%
\begin{frame}
\frametitle{Review: The EM algorithm}

Given a set of observed (visible) variables $V$, a set of unobserved
(hidden / latent / missing) variables $H$, and model parameters
$\theta$, optimize the log likelihood:  
%
\begin{equation*}
{\cal L}(\theta)=\log p(V|\theta)=\log \int p(H,V|\theta)dH,
\end{equation*}
% \\

Using Jensen's inequality, for \Red{any distribution} of hidden
variables $q(H)$ we have:
%
\begin{equation*}
{\cal L}(\theta)=\log\int q(H)\frac{p(H,V|\theta)}{q(H)}\; dH
\geq\int q(H)\log\frac{p(H,V|\theta)}{q(H)} \; dH={\cal F}(q,\theta),
\end{equation*}
%
defining the ${\cal F}(q,\theta)$ functional, which is a lower bound
on the log likelihood.\\

In the EM algorithm, we alternately optimize ${\cal F}(q,\theta)$ wrt
$q$ and $\theta$, and we can prove that this will never decrease
${\cal L}$.


\end{frame}
%%%%%%%%%%%%%%%%%%%%%%%%%%%%%%%%%%%%%%%%%%%%%%%%%%%%%%%%%%%%%%%%%%%%%%
\begin{frame}
\frametitle{The E and M steps of EM}

The lower bound on the log likelihood:
%
\begin{equation*}
{\cal F}(q,\theta)=\int q(H)\log\frac{p(H,V|\theta)}{q(H)}dH=
\int q(H)\log p(H,V|\theta)dH+\Green{{\cal H}(q)},
\end{equation*}
%
where $\Green{{\cal H}(q)=-\displaystyle\int q(H)\log q(H)dH}$ is the \Green{entropy} of
$q$. We iteratively alternate:
%
\begin{description}
\item[E step:] maximize ${\cal F}(q,\theta)$ wrt the distribution over hidden
variables given the parameters:
\begin{equation*}
\Blue{q^{[k]}(H)}:=\underset{q(H)}{\argmax}\;\;{\cal F}\big(\Blue{q(H)},\Red{\theta^{[k-1]}}\big) = p(H|V,\Red{\theta^{[k-1]}}).
\end{equation*}
\item[M step:] maximize ${\cal F}(q,\theta)$ wrt the parameters given
the hidden distribution:
\begin{equation*}
\Blue{\theta^{[k]}}:=\underset{\theta}{\argmax}\;\;
{\cal F}\big(\Red{q^{[k]}(H)},\Blue{\theta}\big)=
\underset{\theta}{\argmax}\;\;\int q^{[k]}(H)\log p(H,V|\theta)dH,
\end{equation*}
which is equivalent to optimizing the expected complete-data log likelihood
$\log p(H,V|\theta)$, since the \Green{entropy of $q(H)$} does not depend on 
$\theta$.
\end{description}

\end{frame}
\cut{
%%%%%%%%%%%%%%%%%%%%%%%%%%%%%%%%%%%%%%%%%%%%%%%%%%%%%%%%%%%%%%%%%%%%%%
\begin{frame}
\frametitle{EM as Coordinate Ascent in ${\cal F}$}

\centerline{\includegraphics[width=9in]{fqtheta}}
\end{frame}
}
%%%%%%%%%%%%%%%%%%%%%%%%%%%%%%%%%%%%%%%%%%%%%%%%%%%%%%%%%%%%%%%%%%%%%%
\begin{frame}
\frametitle{Variational Approximations to the EM algorithm}

Often $p(H|V,\theta)$ is computationally \Red{intractable}, so an
exact E step is out of the question. \\[1ex]

\Blue{\bf Assume some simpler form for $q(H)$}, e.g.\ $q \in {\cal Q}$, the set of
fully-factorized distributions over the hidden variables: $q(H) =
\prod_i q(H_i)$   

\vspace{1ex}

\begin{description}
\item[E step] (approximate):
maximize ${\cal F}(q,\theta)$ wrt the distribution over hidden
variables given the parameters:
\begin{equation*}
\Blue{q^{[k]}(H)}:=\underset{\Green{q(H)\in {\cal Q}}}{\argmax}\;\;{\cal
F}\big(\Blue{q(H)},\Red{\theta^{[k-1]}}\big).
\end{equation*}
\item[M step]: maximize ${\cal F}(q,\theta)$ wrt the parameters given
the hidden distribution:
\begin{equation*}
\Blue{\theta^{[k]}}:=\underset{\theta}{\argmax}\;\;
{\cal F}\big(\Red{q^{[k]}(H)},\Blue{\theta}\big)=
\underset{\theta}{\argmax}\;\;\int q^{[k]}(H)\log p(H,V|\theta)dH,
\end{equation*}
\end{description}

This maximizes a lower bound on the log likelihood.\\[1ex]

Using the fully-factorized $q$ is sometimes called a \Blue{\bf mean-field
approximation}. \\[1ex]

\end{frame}

%%%%%%%%%%%%%%%%%%%%%%%%%%%%%%%%%%%%%%%%%%%%%%%%%%%%%%%%%%%%%%%%%%%%%%
\begin{frame}
\frametitle{Example: Binary latent factor model}
\begin{tabular}{cc}
\includegraphics[width=1.6in]{multicause} & \raisebox{0.45in}{\parbox{2.9in}{
Model with $K$ binary latent variables, $s_i\in\{0,1\}$, \\[1ex]
organised into a vector $\bfs=(s_1, \ldots, s_K)$ \\
real-valued observation vector $\bfy$ \\
parameters $\btheta=\{ \{ \bmu_i, \pi_i\}_{i=1}^K, \sigma^2\}$ \\[2ex]
$\bfs \sim$ Bernoulli \\ 
$\bfy | \bfs  \sim$ Gaussian  
}} 
\end{tabular}
\vspace*{-2.5ex}
\begin{eqnarray*}
p(\bfs|\bpi) = p(s_1, \ldots, s_K|\bpi) &=& \prod_{i=1}^K p(s_i|\pi_i) = \prod_{i=1}^K
\pi_i^{s_i} (1-\pi_i)^{(1-s_i)}  \\
p(\bfy|s_1, \ldots, s_K,\bmu,\sigma^2) &=& {\cal N}\left(\sum_{i=1}^K s_i \bmu_i,
\sigma^2 I\right)  \\[-2ex]
\end{eqnarray*}
EM optimizes bound on likelihood: \hfill $ \displaystyle
{\cal F}(q,\btheta) = \langle \log p(\bfs, \bfy|\btheta)\rangle_{q(\bfs)} -
\langle \log q(\bfs) \rangle_{q(\bfs)} 
$ \\
where $\langle \rangle_q$ is expectation under $q$:  \hspace{2em} $\langle
f(\bfs)\rangle_q \deff \sum_{\bfs} f(\bfs) q(\bfs)$ \\[1.5ex]

\Blue{\bf Exact E step:} $q(\bfs) = p(\bfs|\bfy,\btheta)$ distribution
over $2^K$ states: \Red{\bf intractable} for large $K$ 

\end{frame}
%%%%%%%%%%%%%%%%%%%%%%%%%%%%%%%%%%%%%%%%%%%%%%%%%%%%%%%%%%%%%%%%%%%%%%
\begin{frame}
\frametitle{Example: Binary latent factors model (cont.)}
\begin{tabular}{cc}
\includegraphics[width=1.4in]{multicause} & \raisebox{0.45in}{\parbox{2.4in}{
\[
{\cal F}(q,\btheta) = \langle \Red{\log p(\bfs, \bfy|\btheta)} \rangle_{q(\bfs)} -
\langle \log q(\bfs) \rangle_{q(\bfs)} 
\]
}}
\end{tabular}

{\small
\begin{eqnarray*}
\Red{\log} & \Red{p(\bfs,\bfy|\btheta)} +c & \\
=&  \sum_{i=1}^K \Blue{s_i} \log \pi_i & + (1-\Blue{s_i}) \log
(1-\pi_i) - D \log \sigma - \frac{1}{2 \sigma^2} 
(\bfy - \sum_i \Blue{s_i} \bmu_i)\T (\bfy - \sum_i \Blue{s_i} \bmu_i) \\
=& \sum_{i=1}^K \Blue{s_i} \log \pi_i & + (1-\Blue{s_i}) \log
(1-\pi_i) - D \log \sigma \\ & & - \frac{1}{2 \sigma^2} 
\left(\bfy\T \bfy - 2 \sum_i \Blue{s_i} \bmu_i\T \bfy + \sum_i \sum_j \Blue{s_i  s_j}
\bmu_i\T \bmu_j\right) 
\end{eqnarray*}
}
we therefore need $\langle \Blue{s_i} \rangle$  and $\langle \Blue{s_i s_j} \rangle$ to
compute $\cal F$. \\[0.1ex]

These are the expected {\em sufficient statistics} of the hidden variables. 

\end{frame}
%%%%%%%%%%%%%%%%%%%%%%%%%%%%%%%%%%%%%%%%%%%%%%%%%%%%%%%%%%%%%%%%%%%%%%
\begin{frame}
\frametitle{Example: Binary latent factors model (cont.)}

{\bf Variational approximation}: 
\[
q(\bfs)= \prod_i q_i(s_i) = \prod_{i=1}^K \lambda_i^{s_i}
(1-\lambda_i)^{(1-s_i)} 
\]
where $\lambda_i$ is a parameter of the variational approximation
modelling the posterior mean of $s_i$ (compare to $\pi_i$ which models
the {\em prior} mean of $s_i$). \\[1ex]
Under this approximation we know $\langle \Blue{s_i} \rangle = \lambda_i$
and $\langle \Blue{s_i s_j} \rangle = \lambda_i \lambda_j + \delta_{ij}
(\lambda_i - \lambda^2_i)$.
\[
\begin{split}
{\cal F}(\blambda,\btheta)  = &
\sum_i \lambda_i \log \frac{\pi_i}{\lambda_i} +
(1-\lambda_i) \log \frac{(1-\pi_i)}{(1-\lambda_i)} \\
& - D \log \sigma  - \frac{1}{2 \sigma^2}  
(\bfy - \sum_i \lambda_i \bmu_i)\T (\bfy - \sum_i \lambda_i \bmu_i)  \\
& -\frac{1}{2\sigma^2} \sum_i (\lambda_i - 
\lambda_i^2) \bmu_i\T \bmu_i  - \frac{D}{2} \log
(2 \pi)
\end{split}
\]

\end{frame}
%%%%%%%%%%%%%%%%%%%%%%%%%%%%%%%%%%%%%%%%%%%%%%%%%%%%%%%%%%%%%%%%%%%%%%
\begin{frame}
\frametitle{Fixed point equations for the binary latent factors model}

Taking derivatives w.r.t.\ $\lambda_i$:
\[
\deldel{\cal F}{\lambda_i} = \log \frac{\pi_i}{1-\pi_i} - \log
\frac{\lambda_i}{1-\lambda_i} + \frac{1}{\sigma^2} (\bfy - \sum_{j\neq
i} \lambda_j \bmu_j)\T \bmu_i  - \frac{1}{2\sigma^2} \bmu_i\T \bmu_i
\]
Setting to zero we get fixed point equations:
\[
\lambda_i = f \left( \log \frac{\pi_i}{1-\pi_i} + \frac{1}{\sigma^2}
(\bfy - \sum_{j\neq i} \lambda_j \bmu_j)\T \bmu_i  -
\frac{1}{2\sigma^2} \bmu_i\T \bmu_i \right)  
\]
where $f(x) = 1/(1+\exp(-x))$ is the logistic (sigmoid)
function. \includegraphics[width=0.7in]{sigm} \\[1ex]

\Blue{\bf Learning algorithm:} \\[1ex]
\Blue{\bf E step:} run fixed point equations until convergence of
$\blambda$ {\em for each data point}. \\
\Blue{\bf M step:} re-estimate $\btheta$ given $\blambda$s.


\end{frame}
%%%%%%%%%%%%%%%%%%%%%%%%%%%%%%%%%%%%%%%%%%%%%%%%%%%%%%%%%%%%%%%%%%%%%%
\begin{frame}
\frametitle{KL divergence}

Note that 
\begin{description}
\item[E step]
maximize ${\cal F}(q,\theta)$ wrt the distribution over hidden
variables, given the parameters:
\begin{equation*}
\Blue{q^{[k]}(H)}:=\underset{q(H)\in {\cal Q}}{\argmax}\;\;{\cal
F}\big(\Blue{q(H)},\Red{\theta^{[k-1]}}\big).
\end{equation*}
\end{description}
is equivalent to:
\begin{description}
\item[E step]
minimize ${\cal KL}(q \| p(H|V,\theta))$ wrt the distribution over
hidden variables, given the parameters:
\begin{equation*}
\Blue{q^{[k]}(H)}:=\underset{q(H)\in {\cal Q}}{\argmin} \int \Blue{q(H)} \log
\frac{\Blue{q(H)}}{p(H|V,\Red{\theta^{[k-1]}})} dH
\end{equation*}
\end{description}


So, in each E step, the algorithm tries to find
the best approximation to $p$ in ${\cal Q}$. \\[1ex]

This is related to ideas in {\em information geometry}.

\end{frame}
%%%%%%%%%%%%%%%%%%%%%%%%%%%%%%%%%%%%%%%%%%%%%%%%%%%%%%%%%%%%%%%%%%%%%%
\begin{frame}
\frametitle{Variational Approximations to Bayesian Learning}

\begin{eqnarray*}
\log p(V) &=& \log \int \int p(V,H|\btheta) p(\btheta) \; dH \; d\btheta \\
&\ge & \int \int q(H,\btheta) \log \frac{p(V,H,\btheta)}{q(H,\btheta)}
\; dH \; d\btheta 
\end{eqnarray*} \\[1ex]

Constrain $q \in {\cal Q}$ s.t. $q(H,\btheta) = q(H) q(\btheta)$. \\[2ex]

This results in the \Green{\bf variational Bayesian EM algorithm}. \\[2ex]

More about this later (when we study model selection).

\end{frame}
%%%%%%%%%%%%%%%%%%%%%%%%%%%%%%%%%%%%%%%%%%%%%%%%%%%%%%%%%%%%%%%%%%%%%%
\begin{frame}
\frametitle{Variational Approximations and Graphical Models I}

Let $q(H) = \prod_i q_i(H_i)$. \\[2ex]

Variational approximation maximises $\cal F$:
\[
{\cal F}(q) = \int q(H) \log p(H,V) dH - \int q(H) \log q(H) dH
\]
Focusing on one term, $q_j$, we can write this as:
\[
{\cal F}(q_j) = \int q_j(H_j) \left\langle \log p(H,V)
\right\rangle_{\sim q_j(H_j)} dH_j + \int q_j(H_j) \log q_j(H_j) dH_j+
\mbox{const} 
\]
Where $\left\langle \cdot \right\rangle_{\sim q_j(H_j)}$ denotes
averaging w.r.t.\ $q_i(H_i)$ for all $i \neq j$ \\[2ex]

Optimum  occurs when: 
\[
\Blue{q^*_j(H_j) = \frac{1}{Z} \exp \left\langle \log p(H,V)
\right\rangle_{\sim q_j(H_j)}}
\]

\end{frame}
%%%%%%%%%%%%%%%%%%%%%%%%%%%%%%%%%%%%%%%%%%%%%%%%%%%%%%%%%%%%%%%%%%%%%%
\begin{frame}
\frametitle{Variational Approximations and Graphical Models II}


\parbox{3in}{
Optimum  occurs when: 
\[
\Blue{q^*_j(H_j) = \frac{1}{Z} \exp \left\langle \log p(H,V)
\right\rangle_{\sim q_j(H_j)}}
\]

Assume graphical model: $p(H,V) = \prod_i p(X_i|\mbox{pa}_i)$
}
\parbox{1.8in}{
\centerline{\includegraphics[width=1.8in]{gmxx}}
}

\begin{eqnarray*}
\log q^*_j(H_j) &=& \Big\langle \sum_i \log p(X_i|\mbox{pa}_i)
\Big\rangle_{\sim q_j(H_j)} + \mbox{const} \\
&=&  \left\langle \log p(H_j|\mbox{pa}_j) \right\rangle_{\sim
q_j(H_j)} + \sum_{k \in \mbox{ch}_j } \left\langle \log
p(X_k|\mbox{pa}_k) \right\rangle_{\sim q_j(H_j)}
+ \mbox{const} 
\end{eqnarray*}

This defines messages that get passed between nodes in the graph. Each
node receives messages from its \Blue{Markov boundary}: parents,
children and parents of children. \\[1ex]

\centerline{\Green{\small Variational Message Passing (Winn and Bishop, 2004)}}

\end{frame}
%%%%%%%%%%%%%%%%%%%%%%%%%%%%%%%%%%%%%%%%%%%%%%%%%%%%%%%%%%%%%%%%%%
\begin{frame}
\frametitle{Expectation Propagation (EP)}

% \newcommand{\D}{\mathcal{D}}
% \newcommand{\Magenta}{\textcolor{magenta}}

\vspace*{-1ex}
Data (iid) $\D=\{ \bfx^{(1)} \ldots, \bfx^{(N)} \}$, model
$p(\bfx|\btheta)$, with parameter prior $p(\btheta)$. \\

The parameter posterior is: \hfill $ \displaystyle
p(\btheta|\D) = \frac{1}{p(\D)} p(\btheta) \prod_{i=1}^N
p(\bfx^{(i)}|\btheta) $\\

We can write this as product of factors over $\btheta$: \hfill $\displaystyle
p(\btheta) \prod_{i=1}^N p(\bfx^{(i)}|\btheta) = \PineGreen{\prod_{i=0}^N f_i(\btheta)}
$\\
where $f_0(\btheta) \deff p(\btheta)$ and $f_i(\btheta) \deff
p(\bfx^{(i)}|\btheta)$ and we will ignore the constants. \\

We wish to approximate this by a product of {\em simpler} terms: \hfill $\displaystyle
q(\btheta) \deff \Magenta{\prod_{i=0}^N \tilde{f}_i (\btheta)}$ \\

\begin{center}
\begin{tabular}{ll}
$\displaystyle \min_{q(\btheta)} \KL\left(\PineGreen{\prod_{i=0}^N f_i(\btheta) }\right\| \left. \Magenta{\prod_{i=0}^N
\tilde{f}_i(\btheta)}\right)$ & \Red{(intractable)} \\[2.5ex]
$\displaystyle \min_{\tilde{f}_i(\btheta)} \KL\left( \PineGreen{f_i(\btheta)} \| \Magenta{\tilde{f}_i(\btheta)}\right)$ & \Red{(simple, non-iterative, inaccurate)} \\[2.5ex]
$\displaystyle \min_{\tilde{f}_i(\btheta)} \KL \Big( \PineGreen{f_i(\btheta)} \Magenta{\prod_{j\neq i}
  \tilde{f}_j(\btheta)} \Big\| 
\Magenta{\tilde{f}_i(\btheta) \prod_{j\neq i} \tilde{f}_j(\btheta)}
\Big)$ & \Red{(simple, iterative, accurate)} $\leftarrow$ \Blue{EP} 
\end{tabular}
\end{center}

\end{frame}
%%%%%%%%%%%%%%%%%%%%%%%%%%%%%%%%%%%%%%%%%%%%%%%%%%%%%%%%%%%%%%%%%%
\begin{frame}
\frametitle{Expectation Propagation II}

% \vspace*{-1ex}

\centerline{
\fbox{\parbox{4.1in}{\begin{flushleft}
\begin{algorithmic}
\STATE Input  $f_0(\btheta) \ldots f_N(\btheta)$
\STATE Initialize $\tilde{f}_0(\btheta)=f_0(\btheta)$,
$\tilde{f}_i(\btheta)=1$ for $i>0$, $q(\btheta)=\prod_i
\tilde{f}_i(\btheta)$ 
\REPEAT 
\FOR{$i=0 \ldots N$}
\STATE \Blue{Deletion:} $\displaystyle \qi \leftarrow
\frac{q(\btheta)}{\tilde{f}_i(\btheta)} = \prod_{j \neq i}
\tilde{f}_j(\btheta)$ \\
\STATE \Blue{Projection:} $\displaystyle \tilde{f}^{\mathrm{new}}_i(\btheta) \leftarrow
\arg\min_{f(\btheta)} \, \KL(f_i(\btheta) \qi \|
f(\btheta) \qi)$ \\
\STATE \Blue{Inclusion:} $\displaystyle q(\btheta) \leftarrow
\tilde{f}^{\mathrm{new}}_i(\btheta) \, \qi $\\
\ENDFOR
\UNTIL{convergence}
\end{algorithmic}
\end{flushleft}}}}

\Red{\bf The EP algorithm.} Some variations are possible: here we assumed
  that $f_0$ is in the exponential family, and we updated sequentially over
  $i$. % The names for the steps (deletion,  projection, inclusion)
       % are not the same as in~(Minka, 2001) 
\begin{itemize}
\item Tries to minimize the opposite KL to variational methods
\item $\tilde{f}_i(\btheta)$ in exponential family $\rightarrow$
projection step is \Green{moment matching} 
% \item Loopy belief propagation and assumed density filtering are special cases 
\item No convergence guarantee (although convergent forms can be developed) 
\end{itemize}

\end{frame}
%%%%%%%%%%%%%%%%%%%%%%%%%%%%%%%%%%%%%%%%%%%%%%%%%%%%%%%%%%%%%%%%%%%%%%
\begin{frame}
\frametitle{Some Further Readings}

{\small 
\begin{itemize}
\item MacKay, D.J.C.\ (2003) Information Theory, Inference, and
Learning Algorithms. Chapter~33.
\item Bishop, C.M.\ (2006) Pattern Recognition and Machine Learning. 
\item Winn, J.\ and Bishop, C.M.\ (2005) Variational Message Passing. 
{\em J.\ Machine Learning
  Research}. {\tt http://johnwinn.org/Publications/papers/VMP2005.pdf}
\item Lu, X., Hauskrecht, M., and Day, R.S. (2004) Modeling cellular processes with
variational Bayesian cooperative vector quantizer. In the Proceedings
of the Pacific Symposium on Biocomputing (PSB) 9:533-544.
{\tt http://psb.stanford.edu/psb-online/proceedings/psb04/lu.pdf}
\item Minka, T.P.\ (2004) Roadmap to EP: \\
{\tt http://research.microsoft.com/$\sim$minka/papers/ep/roadmap.html}
\item Ghahramani, Z. (1995) Factorial learning and the EM algorithm.
In Adv Neur Info Proc Syst 7. \\
{\tt http://learning.eng.cam.ac.uk/zoubin/zoubin/factorial.abstract.html}
\item Jordan, M.I., Ghahramani, Z., Jaakkola, T.S. and Saul, L.K. (1999) 
     An Introduction to Variational Methods for Graphical Models.
Machine Learning 37:183-233. Available at:
{\tt http://learning.eng.cam.ac.uk/zoubin/papers/varintro.pdf}
\end{itemize}
}
\end{frame}
%%%%%%%%%%%%%%%%%%%%%%%%%%%%%%%%%%%%%%%%%%%%%%%%%%%%%%%%%%%%%%%%%%%%%%
\begin{frame}
\frametitle{Appendix: The binary latent factors model for an i.i.d.\ data set}

\vspace*{-1ex}
Assume data set ${\cal D}=\{ \bfy^{(1)} \ldots, \bfy^{(N)}\}$ of $N$
points and params $\btheta=\{ \{ \bmu_i, \pi_i\}_{i=1}^K,
\sigma^2\}$ \\[1ex]
Use a factorised distribution: \\[1ex]
\centerline{$\displaystyle q(\bfs)=\prod_{n=1}^N q_n(\bfs^{(n)}) = 
\prod_{n=1}^N \prod_{i=1}^K q_n(s_{i}^{(n)}) = \prod_{n} \prod_{i} (\lambda^{(n)}_i)^{s_i^{(n)}} (1-\lambda^{(n)}_i)^{(1-s_i^{(n)})}$}

\vspace*{-2ex}
\begin{eqnarray*}
p({\cal D}|\btheta)&=&\prod_{n=1}^N p(\bfy^{(n)}|\btheta)\\
p(\bfy^{(n)}|\btheta) &=& \sum_{\bfs} p(\bfy^{(n)}|\bfs,\bmu,\sigma)
p(\bfs|\bpi)\\
{\cal F}(q(\bfs),\btheta) &=& \sum_n {\cal
  F}_n(q_n(\bfs^{(n)}),\btheta) \le \log p({\cal D}|\btheta) \\
{\cal F}_n(q_n(\bfs^{(n)}),\btheta) &=& \left\langle \log p(\bfs^{(n)}, \bfy^{(n)}|\btheta) \right\rangle_{q_n(\bfs^{(n)})} -
\left\langle \log q_n(\bfs^{(n)}) \right\rangle_{q_n(\bfs^{(n)})}
\end{eqnarray*} 

We need to optimise w.r.t.\ $q_n(\bfs^{(n)})$ for
{\em each data point}, so \\

\Blue{\bf E step:} optimize $q_n(\bfs^{(n)})$ (i.e.\ $\blambda^{(n)}$) for each $n$.\\
\Blue{\bf M step:} re-estimate $\btheta$ given  $q_n(\bfs^{(n)})$'s. 

\end{frame}
%%%%%%%%%%%%%%%%%%%%%%%%%%%%%%%%%%%%%%%%%%%%%%%%%%%%%%%%%%%%%%%%%%%%%%
\begin{frame}
\frametitle{Appendix: How tight is the lower bound?}
 
It is hard to compute a nontrivial general upper bound. \\[1ex]

To determine how tight the bound is, one can approximate the true
likelihood by a variety of other methods. \\[1ex]

One approach is to use the variational approximation as as a proposal
distribution for {\bf importance sampling}.

\begin{center}
\includegraphics[width=3in]{importance}
% \put(-340, 30){\mbox{$q(x)$}}
% \put(-220, 105){\mbox{$p(x)$}}
\end{center}

\vspace{0.1in}
But this will generally not work well. See exercise 33.6 in David
MacKay's textbook. 

\end{frame}
\end{document}
%%%%%%%%%%%%%%%%%%%%%%%%%%%%%%%%%%%%%%%%%%%%%%%%%%%%%%%%%%%%%%%%%%%%%%
%%%%%%%%%%%%%%%%%%%%%%%%%%%%%%%%%%%%%%%%%%%%%%%%%%%%%%%%%%%%%%%%%%%%%%
%%%%%%%%%%%%%%%%%%%%%%%%%%%%%%%%%%%%%%%%%%%%%%%%%%%%%%%%%%%%%%%%%%%%%%
