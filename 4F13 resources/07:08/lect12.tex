\subtitle{4F13: Machine Learning}
\author[Ghahramani \& Rasmussen]{Zoubin Ghahramani and Carl Edward Rasmussen}
\institute[CUED]{Department of Engineering, University of Cambridge}

\parskip=1.5ex

\newcommand{\maketilde}{\raisebox{0.4ex}{\tiny $\sim$}}
\newcommand{\bfa}{\mathbf a}
\newcommand{\bfb}{\mathbf b}
\newcommand{\bff}{\mathbf f}
\newcommand{\bfm}{\mathbf m}
\newcommand{\bfp}{\mathbf{p}}
\newcommand{\bfs}{\mathbf s}
\newcommand{\bfu}{\mathbf u}
\newcommand{\bfx}{\mathbf x}
\newcommand{\bfy}{\mathbf y}
\newcommand{\bfv}{\mathbf v}

\newcommand{\bmu}{{\boldsymbol{\mu}}}
\newcommand{\btheta}{\boldsymbol{\theta}}
\newcommand{\balpha}{\boldsymbol{\alpha}}
\newcommand{\bphi}{\boldsymbol{\phi}}
\newcommand{\bpi}{\boldsymbol{\pi}}
\newcommand{\blambda}{\boldsymbol{\lambda}}

\newcommand{\N}{\mathcal{N}}
\newcommand{\R}{\mathbb{R}}
\newcommand{\T}{{\scriptsize^{\top}}}
\newcommand{\D}{\mathcal{D}}
\newcommand{\F}{\mathcal{F}}
\newcommand{\E}{\mathbb{E}}
\newcommand{\V}{\mathbb{V}}
\newcommand{\M}{\mathcal{M}}
\newcommand{\KL}{\mathcal{KL}}
\newcommand{\cut}[1]{}
\newcommand{\trace}{\operatorname{trace}}

\newcommand{\argmax}{\operatorname{argmax}}
\newcommand{\argmin}{\operatorname{argmin}}
\newcommand{\ci}{{\bot\negthickspace\negthickspace\bot}} % conditional indep.
\newcommand{\neigh}{\operatorname{ne}}
\newcommand{\vectr}[2]{  \left[ \!\!\begin{array}{c} #1 \\
      #2 \end{array} \!\!\right]}
\newcommand{\deff}{\stackrel{\mathrm{def}}{=}}
\newcommand{\deldel}[2]{\frac{\partial #1}{\partial #2}}


\definecolor{mypine}{rgb}{0.05,0.45,0.05}
\definecolor{mycyan}{rgb}{0.0,0.4,0.4}
\newcommand{\Red}{\textcolor{red}}
\newcommand{\Blue}{\textcolor{blue}}
\newcommand{\Green}{\textcolor{mypine}}
\newcommand{\PineGreen}{\textcolor{mypine}}
\newcommand{\Magenta}{\textcolor{magenta}}


\newcommand{\by}{\mathbf{y}}
\newcommand{\bx}{\mathbf{x}}
\newcommand{\yy}{\mathbf{y}}
\newcommand{\im}{\item}
\newlength{\oin}
\setlength{\oin}{0.45in}
\newcommand{\cF}{\mathcal{F}}
\newcommand{\nn}{\nonumber}
\newcommand{\bw}{\mathbf{w}}
\newcommand{\bv}{\mathbf{v}}



\begin{document}

\titleslide{Lecture 12: Model Comparison}{February 27th, 2008}

%%%%%%%%%%%%%%%%%%%%%%%%%%%%%%%%%%%%%%%%%%%%%%%%%%%%%%%%%%%%%%%%%%
\begin{frame}
\frametitle{Model complexity and overfitting: \\ a simple example}

\vspace{0.5\oin}
\centerline{\includegraphics[width=6\oin]{ofitnew}}
\vspace{0.3\oin}

\end{frame}
%%%%%%%%%%%%%%%%%%%%%%%%%%%%%%%%%%%%%%%%%%%%%%%%%%%%%%%%%%%%%%%%%%
\begin{frame}
\frametitle{Learning Model Structure}

\vspace*{-5ex}
How many clusters in the data? \hfill \parbox{1.5\oin}{\includegraphics[width=1.5\oin]{clusters}}

\vspace*{-2.5ex}
What is the intrinsic dimensionality of the data? \hfill \parbox{1.5\oin}{\includegraphics[width=1.5\oin]{surface}}

Is this input relevant to predicting that output?
\hfill \parbox{1.5\oin}{\includegraphics[width=1.5\oin]{bonds}} 
% Japanese Govt Bond 

\vspace*{-2.5ex}
What is the order of a dynamical system?
\hfill \parbox{1.5\oin}{\includegraphics[width=1.5\oin]{steelpic}}

How many states in a hidden Markov model? 

\vspace*{-2.5ex}
\phantom{.} \hfill {\small \tt SVYDAAAQLTADVKKDLRDSWKVIGSDKKGNGVALMTTY}

\vspace*{-2.5ex}
How many auditory sources in the input? \hfill
\parbox{1.5\oin}{\includegraphics[width=1.5\oin]{bss}}


\end{frame}
%%%%%%%%%%%%%%%%%%%%%%%%%%%%%%%%%%%%%%%%%%%%%%%%%%%%%%%%%%%%%%%%%%
\begin{frame}
\frametitle{Using Occam's Razor to Learn Model Structure}
\vspace*{-0.3\oin}
Compare model classes $m$ using their posterior probability given the
data:  
\vspace*{-0.3\oin}
\[
P(m|\by) = \frac{\Red{P(\by|m)} P(m)}{P(\by)}, \hspace*{0.5\oin}
\Red{P(\by|m)} = \int_{\Theta_m} P(\by|\btheta_m, m)
P(\btheta_m|m) \; d {\btheta_m}
\]
{\bf Interpretation of $P(\by|m)$:} The probability that {\em
randomly selected}\ parameter values from the model class would
generate data set $\by$. \\[1ex]

Model classes that are \Green{too simple} are unlikely to
generate the data set. \\[1ex]

Model classes that are \Blue{too complex} can generate many possible
data sets, so again, they are unlikely to generate that particular
data set at random.\\[3ex]

\centerline{\includegraphics[width=0.4\textwidth]{ockham}}

\end{frame}
%%%%%%%%%%%%%%%%%%%%%%%%%%%%%%%%%%%%%%%%%%%%%%%%%%%%%%%%%%%%%%%%%%
\begin{frame}
\frametitle{Bayesian Model Comparison: Terminology}

\begin{itemize}
\item A \Blue{model class} $m$ is a set of models parameterised by
$\btheta_m$, e.g. the set of all possible mixtures of $m$ Gaussians. 

\item  The \Blue{marginal likelihood} of model class $m$:
\[
\Red{P(\by|m)} = \int_{\Theta_m} P(\by|\btheta_m, m)
P(\btheta_m|m) \; d {\btheta_m}
\]
is also known as the \Blue{Bayesian evidence} for model $m$.

\item The ratio of two marginal likelihoods is known as the
\Blue{Bayes factor}: 
\[
\frac{P(\by|m)}{P(\by|m')}
\]

\item The \Blue{Occam's Razor} principle is, roughly speaking, that
one should prefer simpler explanations than more complex explanations.

\item Bayesian inference formalises and {\em automatically} implements
the Occam's Razor principle.
\end{itemize}

\end{frame}
%%%%%%%%%%%%%%%%%%%%%%%%%%%%%%%%%%%%%%%%%%%%%%%%%%%%%%%%%%%%%%%%%%
\begin{frame}

\frametitle{Bayesian Model Comparison: \normalsize Occam's Razor at Work }

\vspace{0.1\oin}
\centerline{\includegraphics[width=5\oin,angle=90]{ofitsamps2} 
\raisebox{36pt}[0pt][36pt]{
\includegraphics[width=3\oin,angle=90]{ofitpost2}}}

\vspace*{-4ex}
e.g. for quadratic (M=2): $ y = a_0 + a_1 x + a_2 x^2 + \epsilon$,
where $\epsilon \sim {\cal   N}(0,\tau)$ and $\btheta_2 = [a_0 \; a_1
\; a_2 \; \tau]$ \\[1ex]
{\tt demo: polybayes}
\end{frame}
%%%%%%%%%%%%%%%%%%%%%%%%%%%%%%%%%%%%%%%%%%%%%%%%%%%%%%%%%%%%%%%%%%
\begin{frame}
\frametitle{Practical Bayesian approaches}
\vspace{8pt}

\begin{itemize}
 \item \Red{Laplace approximations}: 
        \begin{itemize}
         \item Makes a Gaussian approximation about the maximum {\em a
                posteriori} parameter estimate. \end{itemize}   
    \item \Red{Bayesian Information Criterion} (BIC)
        \begin{itemize}
	\item  an  asymptotic approximation. \end{itemize}
    \item \Red{Markov chain Monte Carlo methods} (MCMC):
    \begin{itemize}
         \item In the limit are guaranteed to converge, but:
        \item Many samples required to ensure accuracy.
        \item Sometimes hard to assess convergence. \end{itemize}
    \item  \Red{Variational approximations}
\end{itemize}

\vspace*{0.6\oin}
Note: other deterministic approximations have been developed more
recently and can be applied in this context:  e.g.\ Bethe
approximations and Expectation Propagation.

\vfill

\end{frame}
%%%%%%%%%%%%%%%%%%%%%%%%%%%%%%%%%%%%%%%%%%%%%%%%%%%%%%%%%%%%%%%%%%
\begin{frame}
\frametitle{Laplace Approximation}

\vspace*{-0.4\oin}
{\small data set: $\yy$ \hspace{5mm} models: $m=1 \ldots,M$
\hspace{5mm} 
parameter sets: $\btheta_1 \ldots, \btheta_M$} \\

Model Comparison: \hspace{1\oin} $P(m|\yy) \propto P(m)
P(\yy|m) $ \\[1ex]

For large amounts of data (relative to number of parameters, $d$) the
parameter posterior is approximately Gaussian around the MAP estimate
$\hat{\btheta}_m$: 
\[
P(\btheta_m|\yy,m) \approx (2 \pi)^{\frac{-d}{2}}
|A|^{\frac{1}{2}} \exp \left \{ -\frac{1}{2} (\btheta_m - \hat{\btheta}_m)^\T A
(\btheta_m - \hat{\btheta}_m) \right\}
\]
\[
P(\yy|m) = \frac{P(\btheta_m, \yy| m)}{P(\btheta_m|\yy,m)}
\]
Evaluating the above for $\ln P(\yy|m)$ at
$\hat{\btheta}_m$ we get the Laplace approximation:
\Blue{
\[
\ln P(\yy|m) \approx \ln P(\hat{\btheta}_m|m) + \ln
P(\yy|\hat{\btheta}_m,m) + \frac{d}{2} \ln 2 \pi - \frac{1}{2} \ln
|A|
\]}
$-A$ is the $d\times d$ Hessian matrix of $\log
P(\btheta_m|\yy,m)$: 
$A_{k \ell }= -\frac{\partial^2}{\partial \theta_{m k}
  \partial \theta_{m \ell}} \ln P(\btheta_m|\yy,m) |_{\hat{\btheta}_m}$.\\[1ex]

Can also be derived from $2^{nd}$ order Taylor expansion of log
posterior. \\[1ex]
The Laplace approximation can be used for model comparison. 

\end{frame}
%%%%%%%%%%%%%%%%%%%%%%%%%%%%%%%%%%%%%%%%%%%%%%%%%%%%%%%%%%%%%%%%%%
\begin{frame}
\frametitle{Bayesian Information Criterion (BIC) }

BIC can be obtained from the Laplace approximation:
\begin{equation}
\ln P(\yy|m) \approx \ln P(\hat{\btheta}_m|m) + \ln
P(\yy|\hat{\btheta}_m,m) + \frac{d}{2} \ln 2 \pi - \frac{1}{2} \ln
|A| \nonumber
\end{equation}
in the large sample limit $(N\rightarrow \infty)$ where $N$ is the
number of data points, $A$ grows as $N A_0$ for some fixed matrix
$A_0$, so $\ln |A| \rightarrow \ln |NA_0| = \ln (N^d |A_0|) = d \ln N +
\ln |A_0|$. Retaining only terms that grow in $N$ we get:
\Blue{
\begin{equation}
\ln P(\yy|m) \approx \ln P(\yy|\hat{\btheta}_m,m) - \frac{d}{2} \ln 
N  \nonumber
\end{equation} } 

\vspace*{-5ex}
Properties:
\begin{itemize}
\item Quick and easy to compute, and does not depend on the prior
\im We can use the ML estimate of $\theta$ instead of the MAP estimate
\im It assumes that in the large sample limit, all the parameters are
well-determined (i.e.\ the model is \Red{identifiable}; otherwise, $d$
should be the number of \Red{well-determined} parameters)
\im \Red{\bf Danger:} counting parameters can be deceiving! (c.f.\ sinusoid)
\im It is equivalent to the ``Minimum Description Length'' (MDL) criterion 
\end{itemize}

\end{frame}
%%%%%%%%%%%%%%%%%%%%%%%%%%%%%%%%%%%%%%%%%%%%%%%%%%%%%%%%%%%%%%%%%%
\begin{frame}
\frametitle{Sampling Approximations}

\vspace*{-1ex}
Let's consider a non-Markov chain method, {\bf Importance Sampling:}\\
\vspace*{-2ex}
\begin{eqnarray*}
\ln P(\yy|m) &=& \ln \int_{\Theta_m} P(\yy|\btheta_m,m)
P(\btheta_m|m) \; d \btheta_m\\
&=& \ln \int_{\Theta_m} P(\yy|\btheta_m,m)
\frac{P(\btheta_m|m)}{Q(\btheta_m)} Q(\btheta_m) \; d \btheta_m\\
&\approx & \ln \frac{1}{K} \sum_k P(\yy|\btheta_m^{(k)},m)
\frac{P(\btheta_m^{(k)}|m)}{Q(\btheta_m^{(k)})} 
\end{eqnarray*}
where $\btheta_m^{(k)}$ are i.i.d.\ draws from  
$Q(\btheta_m)$. Assumes we can {\bf sample from} and {\bf evaluate} $Q(\btheta_m)$
(incl.\ normalization!) and we can {\bf compute the likelihood}
$P(\yy|\btheta_m^{(k)},m)$. \\[2ex]

Although importance sampling does not work well in high dimensions,
it inspires the following approach:
Create a {\bf Markov chain,} $Q_k \rightarrow Q_{k+1} \ldots$ for which:
\begin{itemize}
\item $Q_k(\btheta)$ can be evaluated including normalization \vspace*{-1ex}
\item $\lim_{k\rightarrow \infty} Q_k(\btheta) = P(\btheta|\yy,m)$
\end{itemize}

\end{frame}
%%%%%%%%%%%%%%%%%%%%%%%%%%%%%%%%%%%%%%%%%%%%%%%%%%%%%%%%%%%%%%%%%%
\begin{frame}
\frametitle{Variational Bayesian Learning \\[0.1ex] \normalsize Lower
Bounding the Marginal Likelihood} 

Let the hidden latent variables be $\bx$, data $\by$ and the parameters
$\btheta$.\\[1ex]
\Blue{Lower bound} the \PineGreen{marginal likelihood (Bayesian
model evidence)} using Jensen's inequality:
\begin{eqnarray*}
\ln \PineGreen{P(\by)} & = & \ln \int d\bx \, d\btheta \;
P(\by,\bx,\btheta) \hspace{20ex} ||m\\
&= & \ln \int d\bx \, d\btheta \; \Red{Q(\bx,\btheta)} \frac{P(\by,\bx,\btheta)}{\Red{Q(\bx,\btheta)}} \\
&\geq & \int d\bx \, d\btheta \; \Red{Q(\bx,\btheta)} \ln
\frac{P(\by,\bx,\btheta)}{\Red{Q(\bx,\btheta)}} .
\end{eqnarray*}
\vspace{-2pt}
Use a simpler, factorised approximation to $Q(\bx,\btheta)$: 
\begin{eqnarray*}
\hspace*{-0.2\oin}
\ln P(\by) & \geq & \int d\bx \, d\btheta \; \Red{ Q_\bx(\bx)
Q_{\btheta}(\btheta) } \ln \frac{P(\by,\bx,\btheta)}{\Red{ Q_\bx(\bx)
Q_{\btheta}(\btheta) } } \\
&=& \cF(\Red{Q_\bx(\bx)},\Red{Q_{\btheta}(\btheta)},\by) .
\end{eqnarray*}
Maximize this lower bound.
\end{frame}
%%%%%%%%%%%%%%%%%%%%%%%%%%%%%%%%%%%%%%%%%%%%%%%%%%%%%%%%%%%%%%%%%%
\begin{frame}
\frametitle{Variational Bayesian Learning \ldots}

Maximizing this \Blue{lower bound}, $\cF$, leads to {\bf EM-like} updates:
\begin{eqnarray*}
  \Red{Q_\bx^*(\bx)} &\propto& \exp \left \langle \ln P(\bx,\!\by|\btheta) \right \rangle_{\Red{Q_{\btheta}(\btheta)}} \hspace{36pt} {\it E\!-\!like \ step} \nn \\
  \Red{Q_{\btheta}^*(\btheta)} &\propto& P(\btheta) \exp \left \langle \ln P(\bx,\!\by|\btheta) \right \rangle_{\Red{Q_\bx(\bx)}} \hspace{11pt} {\it M\!-\!like \ step} \nn
\end{eqnarray*}

%\vspace{-5pt}
Maximizing $\cF$ is equivalent to minimizing KL-divergence between the
{\em approximate posterior}, $Q(\btheta)Q(\bx)$ and the {\em true posterior}, 
$P(\btheta,\bx|\by)$.

\begin{eqnarray*}
\ln P(\by) - \cF(Q_\bx(\bx),Q_{\btheta}(\btheta),\by) &=& \\
\ln P(\by) - \int d\bx \, d\btheta \; Q_\bx(\bx)
Q_{\btheta}(\btheta)  \ln \frac{P(\by,\bx,\btheta)}{ Q_\bx(\bx)
Q_{\btheta}(\btheta)  } &=&  \\ 
\int d\bx \, d\btheta \; Q_\bx(\bx)
Q_{\btheta}(\btheta)  \ln \frac{ Q_\bx(\bx)
Q_{\btheta}(\btheta)  }{P(\bx,\btheta|\by)} &=&  KL(Q||P)
\end{eqnarray*}
\end{frame}
%%%%%%%%%%%%%%%%%%%%%%%%%%%%%%%%%%%%%%%%%%%%%%%%%%%%%%%%%%%%%%%%%%
\begin{frame}
\frametitle{Mixture of Factor Analysers}

\centerline{
\includegraphics[width=3\oin]{mfacdots}}

\vspace*{-2.5em}
Goal:
\begin{itemize}
\item Infer number of clusters
\item Infer intrinsic dimensionality of each cluster
\end{itemize}
Under the assumption that each cluster is Gaussian \\

{\tt embed\_demo}

\end{frame}
%%%%%%%%%%%%%%%%%%%%%%%%%%%%%%%%%%%%%%%%%%%%%%%%%%%%%%%%%%%%%%%%%%
\begin{frame}

\frametitle{Mixture of Factor Analysers}

\Blue{True data:} 6 Gaussian clusters with dimensions: (1 7 4 3 2 2)
embedded in 10-D \\[1ex]

\Red{Inferred structure:}\\[1ex]
\centerline{\includegraphics[width=8\oin]{table}}

\begin{itemize}
\item Finds the clusters and dimensionalities efficiently.
      \item The model complexity reduces in line with the lack of data
support.   
\end{itemize}

{\tt demos: run\_simple} and {\tt ueda\_demo}

\end{frame}
%%%%%%%%%%%%%%%%%%%%%%%%%%%%%%%%%%%%%%%%%%%%%%%%%%%%%%%%%%%%%%%%%%%%%%
\begin{frame}
\frametitle{Hidden Markov Models}

\centerline{\includegraphics[width=0.7\textwidth]{hmm}} 

\vspace*{1ex}

Discrete hidden states, $\bfs_t$.\\

Observations $\by_t$. \\[2ex]

How many hidden states? \\
What structure state-transition matrix?\\[2ex]

{\tt demo: vbhmm\_demo}
\end{frame}
%%%%%%%%%%%%%%%%%%%%%%%%%%%%%%%%%%%%%%%%%%%%%%%%%%%%%%%%%%%%%%%%%%
\begin{frame}
\frametitle{Linear Dynamical Systems}
\centerline{
\includegraphics[width=5\oin]{ssm}
} 
\begin{itemize}
\item Assumes \Blue{$\by_t$} generated from a sequence of Markov {\em
hidden} state variables \Blue{$\bx_t$}
\im If transition and output functions are linear, time-invariant, and
noise distributions are Gaussian, this is a \Red{linear-Gaussian
state-space model}: 
\[ \bx_t = A \bx_{t-1} + \bw_t, \hspace{0.4\oin}  \by_t = C \bx_t + \bv_t  \]
% \im This is a dynamic generalization of factor analysis.
\im Three levels of inference: 
\begin{enumerate} 
\item[I] Given data, structure and parameters, {\bf Kalman smoothing}
$\rightarrow$ hidden state; \\
\item[II] Given data and structure, {\bf EM} $\rightarrow$ hidden
state and parameter point estimates; \\
\item[III] Given data only, {\bf VEM} $\rightarrow$ {\bf model
structure} {\em and} {\bf distributions over parameters} {\em and}
{\bf hidden state}. 
\end{enumerate}
\end{itemize}
\end{frame}
%%%%%%%%%%%%%%%%%%%%%%%%%%%%%%%%%%%%%%%%%%%%%%%%%%%%%%%%%%%%%%%%%%
\cut{
\begin{frame}
\frametitle{Linear Dynamical System Results}

{\bf Inferring model structure:} \\[1ex] 

\parbox{2.7\oin}{\begin{center}
a) SSM(0,3) i.e. FA \\ \ \\ 
\centerline{
\includegraphics[width=55pt]{exp1}
}
 \end{center}
}
\parbox{2.2\oin}{\begin{center}
b) SSM(3,3) \\ \ \\ 
\centerline{
\includegraphics[width=55pt]{exp2}}
 \end{center}
}
\parbox{2.2\oin}{\begin{center}
c) SSM(3,4) \\ \ \\ 
\centerline{
\includegraphics[width=115pt]{exp3}
}
 \end{center}
}
\vspace{1ex}

{\bf Inferred model complexity reduces with less data:} \\[1ex]

True model: \ \ SSM(6,6) \hspace{10pt} $\bullet$
10-dim observation vector. 
\vspace{2ex}

\centerline{
\includegraphics[width=45pt]{degrad400} 
\put(-21,45){400}
\includegraphics[width=45pt]{degrad350} 
\put(-21,45){350}
\includegraphics[width=45pt]{degrad250} 
\put(-21,45){250}
\includegraphics[width=45pt]{degrad100} 
\put(-21,45){100}
\includegraphics[width=45pt]{degrad030} 
\put(-21,45){30}
\includegraphics[width=45pt]{degrad020} 
\put(-21,45){20}
\includegraphics[width=45pt]{degrad010} 
\put(-21,45){10}
}

{\tt demo: bayeslds}
\end{frame}
}

\cut{
%%%%%%%%%%%%%%%%%%%%%%%%%%%%%%%%%%%%%%%%%%%%%%%%%%%%%%%%%%%%%%%%%%
\begin{frame}

\frametitle{Independent Components Analysis}

Blind Source Separation: 5 $\times$ 100 msec speech and music sources
linearly mixed to produce 11 signals (microphones)

\includegraphics[width=9.5\oin]{attiasbss} 

from Attias (2000)



\end{frame}
}
%%%%%%%%%%%%%%%%%%%%%%%%%%%%%%%%%%%%%%%%%%%%%%%%%%%%%%%%%%%%%%%%%%
\begin{frame}
\frametitle{Summary \& Conclusions}

\begin{itemize} 
\item Bayesian learning avoids overfitting and can be
used to do model comparison / selection. \vspace*{-0.2ex} 
\item But we need approximations: 
\begin{itemize}
\item Laplace
\item BIC
\item Sampling
\item Variational
\end{itemize}
\end{itemize}

\end{frame}
\end{document}
%%%%%%%%%%%%%%%%%%%%%%%%%%%%%%%%%%%%%%%%%%%%%%%%%%%%%%%%%%%%%%%%%%
% APPENDIX
%%%%%%%%%%%%%%%%%%%%%%%%%%%%%%%%%%%%%%%%%%%%%%%%%%%%%%%%%%%%%%%%%%
%%%%%%%%%%%%%%%%%%%%%%%%%%%%%%%%%%%%%%%%%%%%%%%%%%%%%%%%%%%%%%%%%%
%%%%%%%%%%%%%%%%%%%%%%%%%%%%%%%%%%%%%%%%%%%%%%%%%%%%%%%%%%%%%%%%%%
%%%%%%%%%%%%%%%%%%%%%%%%%%%%%%%%%%%%%%%%%%%%%%%%%%%%%%%%%%%%%%%%%%
%%%%%%%%%%%%%%%%%%%%%%%%%%%%%%%%%%%%%%%%%%%%%%%%%%%%%%%%%%%%%%%%%%
%%%%%%%%%%%%%%%%%%%%%%%%%%%%%%%%%%%%%%%%%%%%%%%%%%%%%%%%%%%%%%%%%%
\end{document}


\input{texhead}

\usepackage{amssymb}

\newcommand{\nn}{\nonumber}
% cal
% bold 
\newcommand{\bs}{\mathbf{s}}
\newcommand{\bt}{\mathbf{t}}
\newcommand{\bz}{\mathbf{z}}
\newcommand{\bn}{\mathbf{n}}
\newcommand{\bu}{\mathbf{u}}
\renewcommand{\bm}{\mathbf{m}}
\newcommand{\bc}{\mathbf{c}}
\newcommand{\ba}{\mathbf{a}}
% bold greek
\newcommand{\bpi}{\boldsymbol{\pi}}
% \newcommand{\bmu}{\boldsymbol{\mu}}
\newcommand{\bSigma}{\boldsymbol{\Sigma}}
\newcommand{\balpha}{\boldsymbol{\alpha}}
\newcommand{\brho}{\boldsymbol{\rho}}
\newcommand{\bnu}{\boldsymbol{\nu}}
\newcommand{\bbeta}{\boldsymbol{\beta}}
\newcommand{\bepsilon}{\boldsymbol{\epsilon}}
% \newcommand{\btheta}{\boldsymbol{\theta}}
\newcommand{\bphi}{\boldsymbol{\phi}}
\newcommand{\bnut}{\tilde{\boldsymbol{\nu}}}
\newcommand{\etat}{\tilde{\eta}}
\newcommand{\bub}{\overline{\mathbf{u}}}
\newcommand{\bphib}{\overline{\bphi}}
\newcommand{\bthetat}{\tilde{\boldsymbol{\theta}}}
% other
\newcommand{\diag}{{\rm diag}}
\newcommand{\Qtt}{\PineGreen{Q_{\theta}(\theta)}}
\newcommand{\Qxx}{\Blue{Q_{X}(X)}}

\definecolor{mypine}{rgb}{0.05,0.45,0.05}
\renewcommand{\Green}{\textcolor{mypine}}
\newcommand{\Cyan}{\textcolor{cyan}}
\renewcommand{\PineGreen}{\textcolor{mypine}}
\newcommand{\DarkGreen}{\textcolor{mypine}}

% \graphicspath{{/home/zoubin/talks/PSFILES/}{/home/zoubin/talks/cmu02inf/forzoubin/}{/nfs/home/zoubin/talks/PSFILES/}{/nfs/home/zoubin/talks/cmu02inf/forzoubin/}}

