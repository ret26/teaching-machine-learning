\subtitle{4F13: Machine Learning}
\author[Ghahramani \& Rasmussen]{Zoubin Ghahramani and Carl Edward Rasmussen}
\institute[CUED]{Department of Engineering, University of Cambridge}

\parskip=1.5ex

\newcommand{\maketilde}{\raisebox{0.4ex}{\tiny $\sim$}}
\newcommand{\bfa}{\mathbf a}
\newcommand{\bfb}{\mathbf b}
\newcommand{\bff}{\mathbf f}
\newcommand{\bfm}{\mathbf m}
\newcommand{\bfp}{\mathbf{p}}
\newcommand{\bfs}{\mathbf s}
\newcommand{\bfu}{\mathbf u}
\newcommand{\bfx}{\mathbf x}
\newcommand{\bfy}{\mathbf y}
\newcommand{\bfv}{\mathbf v}

\newcommand{\bmu}{{\boldsymbol{\mu}}}
\newcommand{\btheta}{\boldsymbol{\theta}}
\newcommand{\balpha}{\boldsymbol{\alpha}}
\newcommand{\bphi}{\boldsymbol{\phi}}
\newcommand{\bpi}{\boldsymbol{\pi}}
\newcommand{\blambda}{\boldsymbol{\lambda}}

\newcommand{\N}{\mathcal{N}}
\newcommand{\R}{\mathbb{R}}
\newcommand{\T}{{\scriptsize^{\top}}}
\newcommand{\D}{\mathcal{D}}
\newcommand{\F}{\mathcal{F}}
\newcommand{\E}{\mathbb{E}}
\newcommand{\V}{\mathbb{V}}
\newcommand{\M}{\mathcal{M}}
\newcommand{\KL}{\mathcal{KL}}
\newcommand{\cut}[1]{}
\newcommand{\trace}{\operatorname{trace}}

\newcommand{\argmax}{\operatorname{argmax}}
\newcommand{\argmin}{\operatorname{argmin}}
\newcommand{\ci}{{\bot\negthickspace\negthickspace\bot}} % conditional indep.
\newcommand{\neigh}{\operatorname{ne}}
\newcommand{\vectr}[2]{  \left[ \!\!\begin{array}{c} #1 \\
      #2 \end{array} \!\!\right]}
\newcommand{\deff}{\stackrel{\mathrm{def}}{=}}
\newcommand{\deldel}[2]{\frac{\partial #1}{\partial #2}}


\definecolor{mypine}{rgb}{0.05,0.45,0.05}
\definecolor{mycyan}{rgb}{0.0,0.4,0.4}
\newcommand{\Red}{\textcolor{red}}
\newcommand{\Blue}{\textcolor{blue}}
\newcommand{\Green}{\textcolor{mypine}}
\newcommand{\PineGreen}{\textcolor{mypine}}
\newcommand{\Magenta}{\textcolor{magenta}}


\newcommand{\vv}{{\mathbf v}}
\newcommand{\B}{{\cal B}}
\newcommand{\cP}{{\mathcal P}}
\newcommand{\cR}{{\mathcal R}}
\newcommand{\xx}{{\mathbf x}}
\newcommand{\uu}{{\mathbf u}}

\begin{document}

\titleslide{Lecture 13, 14, 15: Reinforcement Learning}{February 29th, March 5th and 7th, 2008}

\begin{frame}
\frametitle{Intelligent Behaviour?}

Imagine a creature/agent (human/animal/machine) which receives sensory
inputs and can take some actions in an environment:

\centerline{\includegraphics[width=2.1in]{caricature}}

Assume that the creature also receives rewards (or penalties/losses)
from the environment.\\

The goal of the creature is to maximise the rewards it receives (or
equivalently minimise the losses). \\

\Green{A theory for choosing actions that minimize losses is a theory for how
to behave optimally... }
\end{frame}

\begin{frame}
\frametitle{Bayesian Decision Theory}

{\bf Bayesian decision theory} deals with the problem of making
optimal decisions---that is, decisions or actions that minimize an
expected loss. 

\begin{itemize}
\item Let's say we have a choice of taking one of $k$ possible
\Red{\bf actions} $a_1 \ldots a_k$.

\item Assume that the world can be in one of $m$ different
\Red{\bf states} $s_1,\ldots, s_m$.

\item If we take action $a_i$ and the world is in state $s_j$ we incur a
\Red{\bf loss} $\ell_{ij}$

\item Given all the observed data $\D$ and prior background knowledge $\B$ ,
our \Red{\bf beliefs} about the state of the world are summarized by
$p(s|\D,\B)$.


\item \PineGreen{\em The optimal action is the one which is expected to minimize 
loss (or maximize utility)}:
\[
a^*  = \underset{a_i}{\argmin} \sum_{j=1}^m \ell_{ij}\, p(s_j|\D,\B) 
\]
\end{itemize}

\begin{center}
\begin{tabular}{ll}
Bayesian sequential decision theory & \Blue{(statistics)} \\
Optimal control theory & \Blue{(engineering)} \\
Reinforcement learning & \Blue{(computer science / psychology)}
\end{tabular}
\end{center}
\end{frame}

\begin{frame}
\frametitle{A simple example}

Assume we have two actions:

$a_1$ : play \\
$a_2$ : don't play 

And two outcomes:

$s_1$ : win lottery \\
$s_2$ : don't win lottery

Optimal action:
\[
a^*  = \underset{a_i}{\argmin} \sum_{j=1}^m \ell_{ij}\, p(s_j|a_i,\B) 
\]

\[
\begin{array}{ll} \hline
p(s_1|a_1,\B) = 0.000001 & \ell_{11} = -100000 \\
p(s_2|a_1,\B) = 0.999999 & \ell_{12} = +0.9 \\
p(s_1|a_2,\B) = 0 & \ell_{21}= 0 \\
p(s_2|a_2,\B) = 1 & \ell_{22} = 0 \\ \hline
\end{array}
\]
{\em What is the optimal action for this decision problem?}
\end{frame}

\begin{frame}
\frametitle{Comments about the above framework}

\PineGreen{\em The optimal action is the one which is expected to minimize 
loss (or maximize utility)}:
\[
a^*\;=\;\underset{a_i}{\argmin} \sum_{j=1}^m \ell_{ij}\, p(s_j|\D,\B) 
\]

\begin{itemize}
\item This is a theory for how to make a \Blue{\em single decision}. How do
  we make a \Blue{\em sequence of decisions} in order to achieve some long-term
  goals/rewards?
\item This assumes that we \Blue{\em know} what the losses are for each
  action-state pair. The losses may in fact have to be learned from experience.
\item We need a model for how the observed data $\D$ relates to the
  states of the world $s$.
\item It may be impossible to \Blue{\em enumerate} all possible
  actions and states. What about continuous state and action spaces?
\end{itemize}
\end{frame}

\begin{frame}
\frametitle{Markov Decision Problems (MDPs)}

\Blue{States}: $s_t$\\
\Blue{Actions}: $a_t$\\
\Blue{Rewards}: $r_t$

The variable $s_t$ is the state of the world and agent at time $t$

The agent takes action $a_t$ and receives reward $r_t$ (or loss, if
you like to think negatively...)

The reward is assumed to depend on the state and the action.

\centerline{\includegraphics[width=2.1in]{mdp}}

\Green{Markov property: \hfill $
p(s_{t+1},r_{t}|s_t,a_t,s_{t-1},a_{t-1},r_{t-1}, \ldots) = 
p(s_{t+1},r_{t}|s_t, a_t) $ \hfill}
\end{frame}

\begin{frame}
\frametitle{Markov Decision Problems (MDPs)}

\centerline{\includegraphics[width=2.1in]{mdp2}}

The world is characterized by

\Blue{Transition Probabilities:} \hspace{2em}
 $ \cP_{ss'}^a = p(s_{t+1}=s'|s_t=s,a_t=a) $ \\[1ex]

\Blue{Expected rewards:} \hspace{2em}
 $ \cR_{ss'}^a = \E[r_{t+1}|s_t=s,a_t=a,s_{t+1}=s'] $ \\[2ex]

The agent is characterized by

\Blue{Policy:} \hspace{2em} $\pi(s,a) = p(a_t=a|s_t=s)$ \\[1ex]

\emph{Why is the action at time $t$ only dependent on the state at time $t$?}
\end{frame}

\begin{frame}
\frametitle{Markov Decision Problems (MDPs)}

{\em Why is the action at time $t$ only dependent on the state at time $t$?}

\centerline{\includegraphics[width=2.1in]{mdp2}}

Actions $a_t$ should be chosen to maximize sum of (discounted)
future rewards $R_t$.

By the Markov properties in the graph (i.e.\ conditional
independence), future rewards and states are independent of past
rewards, actions, and states given $s_t$ and $a_t$:
\[
p(s_{t+1}, r_{t+1}, s_{t+2}, r_{t+2}, \ldots | s_{t}, a_{t}, s_{t-1},
a_{t-1}, \ldots )\;=\;p(s_{t+1}, r_{t+1}, s_{t+2}, r_{t+2}, \ldots |
s_{t}, a_{t}) 
\]
If $s_t$ is known, the {\em expected} value of the return $R_t$ depends only
on $a_t$, so previous states and actions are irrelevant.
\end{frame}

\begin{frame}
\frametitle{Value Functions}

\Blue{Value Function:} how good is it to be in a given state? This obviously depends on the agent's policy:
\[
V^\pi(s)\;=\;\E_{\pi}[R_t|s_t=s]\;=\;
\E_\pi\Big[\sum_{k=0}^\infty\gamma^k r_{t+k+1}\Big| s_t=s \Big].
\]

\Blue{State-action value function:} how good is it to be in a given
state and take a given action, and then follow policy $\pi$:
\[
Q^\pi(s,a)\;=\;\E_\pi[R_t|s_t=s,a_t=a]\;=\;
\E_\pi\Big[\sum_{k=0}^\infty \gamma^k r_{t+k+1}\Big| s_t=s, a_t=a\Big].
\]

The relation between the state value function and the state-action value
function:
\[
V^\pi(s)\;=\;\sum_a\pi(s,a) Q^\pi(s,a)
\]
\end{frame}

\begin{frame}
\frametitle{Self-Consistency of Value Functions}

A fundamental property of value functions is that they satisfy a set
of recursive consistency equations. $V^\pi$ is the unique solution to
these equations. 
\[
\begin{split}
V^\pi(s) \;&=\;\E_\pi[R_t|s_t=s]\\
&=\;\E_\pi\Big[r_{t+1} + \gamma \sum_{k=0}^\infty \gamma^k
r_{t+k+2}\Big| s_t=s\Big]\\ 
&=\;\sum_a \pi(s,a) \sum_{s'} \cP_{ss'}^a \Big( \cR_{ss'}^a + \gamma \E_\pi
\Big[\sum_{k=0}^\infty \gamma^k r_{t+k+2} \Big| s_{t+1}=s'\Big] \Big) \\
&=\;\sum_a\pi(s,a)\sum_{s'}\cP_{ss'}^a\Big(\cR_{ss'}^a+\gamma V^\pi(s')\Big) 
\end{split}
\]
We can solve them using a ``backup operation'' from $s'\rightarrow s$ (or
other means).  Linear system of $N\equiv|s|$ equations in $N$ unknowns.
\[
\vv = \big(I-\gamma\sum_a\mbox{diag}(\bpi_a)\cP^a\big)^{-1}
\big(\sum_a\bpi_a\odot\mbox{diag}(\cP^a\cR^{a\top})\big) 
\]
There is a similar equation for $Q^{\pi}(s,a)$
\end{frame}

\begin{frame}
\frametitle{Optimal Policies and Values}

{\bf Optimal Policy:} $\pi^*$ such that $V^{\pi^*}(s) \ge V^\pi(s) \;
\forall s$.

There may be more than one optimal policy.

Question: Is there always at least one optimal policy? YES

{\bf Optimal state value function}: $V^*(s) = \max_\pi V^\pi(s) \;
\forall s$ 

{\bf Optimal state-action value function}: $Q^*(s,a) = \max_\pi
Q^\pi(s,a) \; \forall s$. This is the expected return of action $a$ in
state $s$, thereafter following optimal policy. 
\[
Q^*(s,a)\;=\;\E[r_{t+1} + \gamma V^*(s_{t+1}) |s_t=s, a_t=a]. 
\]
\end{frame}

\begin{frame}
\frametitle{Bellman Optimality Equation}
\[
\begin{split}
V^*(s)\;&=\;\max_a Q^{\pi^*}(s,a)\\
&=\;\max_a\E_{\pi^*}\Big[\sum_{k=0}^\infty \gamma^k r_{t+k+1}\Big|s_t=s, a_t=a \Big]\\ 
&=\;\max_a\E_{\pi^*}\Big[r_{t+1} + \gamma V^*(s_{t+1}) \Big|s_t=s, a_t=a \Big]\\
&=\;\max_a \sum_{s'} \cP_{ss'}^a \big(\cR_{ss'}^a + \gamma V^*(s')\Big)
\end{split}
\]
$N$ {\em nonlinear} equations in $N$ unknowns for $V^*$.

\[
\begin{split}
Q^*(s,a)\;&=\;\E\Big[r_{t+1} + \gamma \max_{a'} Q^*(s_{t+1},a')
\Big| s_t=s, a_t=a \Big]\\
&=\;\sum_{s'} \cP_{ss'}^a \big( \cR_{ss'}^a + \gamma \max_{a'} Q^*(s',a')\big)
\end{split}
\]
$NA$ {\em nonlinear} equations in $NA$ unknowns for $Q^*$
\end{frame}

\begin{frame}
\frametitle{Solving MDPs}

Given the optimal value function, $V^*$, it is easy to get optimal
policy $\pi^*$: be {\bf greedy} w.r.t.\ $V^*$.

If you have $V^*$, the actions that appear best after a  one-step
search will be optimal.

$V^*$ turns a long-term reward into a quantity that is locally and
immediately available.

Using $Q^*$ it is even easier to get the optimal policy:
\[
\pi^*(s,a) = 0 \;\; \forall a \;\; s.t. \;\; Q^*(s,a) \neq \max_{a'} Q^*(s,a')
\]
\end{frame}

\begin{frame}
\frametitle{Policy Improvement Theorem}

{\bf Policy Evaluation} \hfill $ \displaystyle
V_{k+1}^\pi(s) \leftarrow \sum_a \pi(s,a) \sum_{s'} \cP_{ss'}^a
\big(\cR_{ss'}^a + \gamma V_k^\pi(s') \big)$ \hfill\\
assumes $\cP$ known, $\cR$ known, and a full backup (we can also sweep in
place) \vspace{0.2in}

{\bf Policy Improvement Theorem}
\[
Q^\pi(s,\pi'(s))\;\ge\;V^\pi(s)\;\forall s\;\;\Longrightarrow\;\;V^{\pi'}(s)
\;\ge\;V^{\pi}(s) 
\]
{\bf Proof}: 
\vspace*{-3ex}
\[
\begin{split} 
V^{\pi}(s)\;&\le\;Q^\pi(s,\pi'(s))\\
&=\;\E_{\pi'}[r_{t+1} + \gamma V^\pi(s_{t+1})|s_t=s]\\
&\le\;\E_{\pi'}[r_{t+1} + \gamma Q^\pi(s_{t+1},\pi'(s_{t+1}))|s_t=s]\\
&=\;\E_{\pi'}\big[r_{t+1} + \gamma \E_{\pi'}[r_{t+2}+\gamma V^{\pi}(s_{t+2})]|s_t=s\big]\\
&=\;\E_{\pi'}[r_{t+1} + \gamma r_{t+2}+\gamma^2V^{\pi}(s_{t+2})|s_t=s]\\
&\;\;\vdots\\
&\le\;V^{\pi'}(s)
\end{split}
\]
\end{frame}

\begin{frame}
\frametitle{Policy Iteration}

The policy improvement theorem suggests a way of improving policies:
\[
\begin{split}
\pi'(s)\;&\leftarrow\;\arg \max_a Q^\pi(s,a) \;\; \forall s\\
&=\;\arg \max_a \E[r_{t+1} + \gamma V^\pi(s_{t+1})|s_t=s,a_t=a]
\end{split}
\]
This procedure converges to an optimal policy by policy improvement
theorem and Bellman optimality.
\[
V^{\pi'}(s)\;\ge\;\arg \max_a Q^{\pi}(s,a)\;\ge\;\sum_a \pi(s,a)
Q^{\pi}(s,a)\;=\;V^{\pi}(s)
\]
{\bf Policy Iteration:} Iterates between evaluation and improvement
\[
\pi_0 \stackrel{E}{\longrightarrow} 
V^{\pi_0} \stackrel{I}{\longrightarrow} 
\pi_1 \stackrel{E}{\longrightarrow} 
V^{\pi_1} \stackrel{I}{\longrightarrow} 
\pi_2 \ldots \pi^*
\]
Problem with Policy Iteration: Evaluation step can be really slow...
\end{frame}

\begin{frame}
\frametitle{Value Iteration}

Do we really need to wait until convergence of the evaluation step?

In fact, we can improve after {\bf one} sweep of evaluation!

\[
\begin{split}
V_{k+1}(s)\;&\leftarrow\;\max_a \E[ r_{t+1}+ \gamma V_k(s_{t+1})|s_t=s,a_t=a] \\
&=\;\max_a \sum_{s'} \cP_{ss'}^a \big( \cR_{ss'}^a + \gamma V_k(s')\big)
\end{split}
\]

converges: $V_k \longrightarrow V^*$. At each step we also have a
policy. 

Problem: it is still not feasible to update the
value of every single state.

E.g. backgammon has $10^{20}$ states!

Bellman called this the {\bf curse of dimensionality}
\end{frame}

\begin{frame}
\frametitle{Asynchronous dynamic programming}

These are in-place iterated dynamic programming (DP) algorithms that
are not organized in terms of systematic sweeps over all the states.

{\em States are backed-up in order visited or randomly.}

To converge the algorithms must continue to visit every state.

Key idea in RL: We can run the DP algorithm at the same time as the
agent is {\em actually experiencing} the MDP.

This leads to an {\bf exploration vs exploitation tradeoff}: act so as to visit
new parts of state space or exploit already visited part of state-space?

An example of a simple exploration strategy are $\epsilon$-greedy
policies: 
\[
\pi_{\epsilon}(s,a) = (1-\epsilon) \pi(s,a) + \epsilon u(a)
\] 
where $u(a)$ is a uniform distribution over actions.

{\em Can you think of anything wrong with this?}
\end{frame}

\begin{frame}
\frametitle{Monte Carlo and TD}

{\bf Monte Carlo methods} solve RL problems by averaging sample returns.

Question: how do you trade off length of sampled trajectory, vs
previously estimated values?
\begin{eqnarray}
V(s_t) &\leftarrow &V(s_t) + \alpha [ r_{t+1} + \gamma V(s_{t+1}) -
V(s_{t})] \label{one} \\
V(s_t) &\leftarrow &V(s_t) + \alpha [ r_{t+1} + \gamma r_{t+2} +
\gamma^2 V(s_{t+2}) -
V(s_{t})] \nonumber \\
V(s_t) &\leftarrow &V(s_t) + \alpha [ r_{t+1} + \gamma r_{t+2} +
\gamma^2 r_{t+3} + \gamma^3 V(s_{t+3}) -
V(s_{t})] \nonumber \\
& \vdots& \nonumber \\
V(s_t) &\leftarrow &V(s_t) + \alpha [ R_t - V(s_{t})]  \label{two}
\end{eqnarray}

Equation~(\ref{one}) is {\bf Temporal Difference learning,
TD(0)}. TD($\lambda$) approximates the range eqn
(\ref{one})--(\ref{two}), where higher $\lambda$ is closer to the full
MC method.

TD($\lambda$) has been proven to converge

These are general methods for controlling the {\bf bias-variance tradeoff}.
\end{frame}

\begin{frame}
\frametitle{SARSA and Q Learning}

Definitions: {\em on-policy} methods evaluate or improve the
current policy used for control. {\em Off-policy} methods evaluate or improve
one policy, while acting using another (behavior) policy.

In off-policy methods, the behavior policy must have non-zero
probability for each state-action the evaluated policy does.


{\bf SARSA:} on-policy greedy control
\[
Q(s_t,a_t)\;\leftarrow\;Q(s_t,a_t) + \alpha [ r_{t+1} + \gamma 
\; Q(s_{t+1},a_{t+1}) - Q(s_t,a_t) ]
\]

{\bf Q Learning}: off-policy greedy control
\[
Q(s_t,a_t)\;\leftarrow\;Q(s_t,a_t) + \alpha [ r_{t+1} + \gamma \max_a
Q(s_{t+1},a) - Q(s_t,a_t) ]
\]
Converges if $\forall a,s$ are visited and updated infinitely often.

We can also combine the bias-variance ideas with $Q$ and SARSA, to get
$Q(\lambda)$ and SARSA($\lambda$).
\end{frame}

\begin{frame}
\frametitle{Function Approximation}

For very large or continuous state spaces it is hopeless to store a
table with all the state values or state-action values.

\parbox{2.7in}{
It makes sense to use {\bf function approximation}\\[-1ex]
\[
V(s) = f_\theta(s)
\]
e.g. basis function representation: \\[-1ex]
\[
V(s) = \sum_i \theta_i \phi_i(s)
\]
Similarly for $Q(s,a)$.\\[1ex]
This should hopefully lead to better {\bf generalization}}
\parbox{2.0in}{
\centerline{\includegraphics[width=2.0in]{value}}}

\vspace{1.5ex}

Gradient descent methods:\\[-1ex]
\[
\btheta(t+1) = \btheta(t) + \alpha [v_t - V_t(s_t)] \frac{\partial
V_t(s_t)}{\partial {\btheta}}
\]
where $\alpha$ is a learning rate and $v_t$ is a measured/estimated
value. See chapter 8 of Sutton and Barto.
\end{frame}

\begin{frame}
\frametitle{Optimal Control}

{\bf Optimal Control}: The engineering field of optimal control covers
exactly the same topics as RL, except the state and action space is
usually assumed to be continuous, and the model is often known.

The {\bf Hamilton-Jacobi-Bellman} optimality conditions
are the continuous state generalization of the Bellman equations.

A typical elementary problem in optimal control is the linear
quadratic Gaussian control {\bf LQG} problem. Here the cost function
is quadratic in states $\xx_t$  and actions $\uu_t$, and the system is a
linear-Gaussian state-space model. 
\[
\xx_{t+1} = A \xx_t + B \uu_t + \epsilon_t
\]
For this model the optimal policy
can be computed from the estimated state. It's a linear feedback
controller:
\[
\uu_t = L \hat{\xx}_t
\]
The optimal policy here happens not depend on the uncertainty in
$\xx_t$. This is not generally the case.
\end{frame}

\begin{frame}
\frametitle{Influence Diagrams}

\parbox{2.7in}{You can extend the framework of directed acyclic probabilistic
graphical models (a.k.a. Bayesian networks) to include \Red{\bf decision
nodes} and \Blue{\bf value nodes}. These are called {\bf influence
diagrams}.

Solving an influence diagram corresponds to finding the settings of the
decision nodes that maximize the expectation of the value node.}
\parbox{2in}{\centerline{\includegraphics[width=2in]{influence-diagram-c}}}

It is possible to convert the problem of solving an influence diagram
into the problem of doing inference in a (usually
multiply connected)  graphical model (Shachter and Peot, 1992). Exact
solutions can be computationally intractable.

Like other graphical models, influence diagrams can contain both
observed and {\bf hidden} variables...
\end{frame}

\begin{frame}
\frametitle{POMDPs}

\parbox{2.4in}{
POMDP = Partially-observable Markov decision problem.

The agent does not observe the full state of the environment.

What is the optimal policy? 
}
\parbox{2.4in}{
\centerline{\includegraphics[width=2.4in]{pomdp}}}

\begin{itemize}
\item If the agent has the correct model of the world, it turns out that the
optimal policy is a (piece-wise linear) function of the {\bf belief
state}, $P(s_t| a_1, \ldots, a_{t-1}, r_1, \ldots, r_{t}, o_1,
\ldots, o_{t})$.\\
\Blue{Unfortunately, the belief state can grow exponentially complex. }

\item Equivalently, we can view the optimal policy as being a function
of the entire sequence of past actions and observations (this is the
usual way the policy in influence diagrams is represented).\\ 
\Blue{Unfortunately, the set of possible such sequences grows
exponentially. }
\end{itemize}

Efficient methods for approximately solving POMDPs is an active research area.
\end{frame}

\begin{frame}
\frametitle{RL summary}

Central ideas in reinforcement learning
\begin{itemize}
\item the difference between \emph{reward} and \emph{value}
\item Bellman consistency of values $V^\pi(s)=\sum_a\pi(s,a)\sum_{s'}\cP_{ss'}^a\Big(\cR_{ss'}^a+\gamma V^\pi(s')\Big)$
\item Policy Iteration $=$ Policy Evaluation $+$ Policy Improvement
\item Requires knowledge of and represenatation of transitions and rewards
\item Key idea: run the algorithms as we're experiencing the MDP
\item on-policy and off-policy methods (SARSA and Q-learning) don't
require knowledge of transition probabilities and rewards
\item exploration vs. exploitation
\end{itemize}
\end{frame}


\begin{frame}
\frametitle{Some References on Reinforcement Learning}

\begin{itemize}
\item{}
Kaelbling, L.P.,  Littman, M.L. and Moore, A.W. (1996) Reinforcement
Learning: A Survey. Journal of Aritificial Intelligence Research
4:237-285. 

\item{}
Sutton, R.S.  and Barto, A.G. (2000) Reinforcement Learning: An
Introduction. MIT Press. \texttt{http://www.cs.ualberta.ca/\maketilde sutton/book/ebook}

\item{}
Bertsekas, D.P. and Tsitsiklis, J.N. (1996) Neuro-Dynamic Programming.
Athena Scientific.

\item{} Bryson, A.E. and Ho, Y.-C. (1975) Applied Optimal
Control. Hemisphere Publishing Corp. Washington DC. 

\end{itemize}
\end{frame}

\end{document}